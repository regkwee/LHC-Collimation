\section{Additional plots for Run I and II simulations\label{run1run2app}}

\subsection{Halo distributions in Run I}

\begin{figure}%[!htb]
\begin{center}
\includegraphics[width=0.49\textwidth]{figures/4TeV/haloB1_20MeV/PhiNDist_BH_4TeV_B1_20MeV.pdf}
\includegraphics[width=0.49\textwidth]{figures/4TeV/haloB1_20MeV/RadNDist_BH_4TeV_B1_20MeV.pdf}
\includegraphics[width=0.49\textwidth]{figures/4TeV/haloB1_20MeV/RadEnDist_BH_4TeV_B1_20MeV.pdf}
\includegraphics[width=0.49\textwidth]{figures/4TeV/haloB1_20MeV/XYNMuons_BH_4TeV_B1_20MeV.pdf}
\end{center}
\vspace{-0.6cm}
 \caption{Halo Beam 1 distributions at the interface plane. 
  \label{dist4TeVB1}}
\end{figure}

\begin{figure}%[!htb]
\begin{center}
%\includegraphics[width=0.4\textwidth]{figures/4TeV/compB1B2/perTCThit/ratioEkinMuons.pdf}
\includegraphics[width=0.4\textwidth]{figures/4TeV/compB1B2/perTCThit/ratioPhiEnAll.pdf}
\includegraphics[width=0.4\textwidth]{figures/4TeV/compB1B2/perTCThit/ratioPhiEnMuons.pdf}
\includegraphics[width=0.4\textwidth]{figures/4TeV/compB1B2/perTCThit/ratioPhiEnProtons.pdf}
\end{center}
\vspace{-0.6cm}
\caption{Comparison of B1 and B2 halo shower distributions at the interface plane. The numbers in the ratio plot is the ratio of both integrals of the top distributions to indicate by how much nominator or denominator is larger. The error bars indicate statistical uncertainties.
  \label{comp4TeVB1B2}}
\end{figure}


\begin{figure}%[!htb]
\begin{center}
\includegraphics[width=0.49\textwidth]{figures/4TeV/bs_20MeV/RadNDist_BG_4TeV_20MeV_bs.pdf}
\includegraphics[width=0.49\textwidth]{figures/4TeV/bs_20MeV/RadEnDist_BG_4TeV_20MeV_bs.pdf}
\end{center}
\vspace{-0.6cm}
 \caption{Beam-gas induced background shown per BG interaction (flat pressure).
  \label{dist4TeVBGbs}}
\end{figure}




\begin{figure}%[!htb]
\begin{center}
  \includegraphics[width=0.411\textwidth]{figures/4TeV/beamsizeRatio/ratioPhiNProtonsE100.pdf}
  \includegraphics[width=0.411\textwidth]{figures/4TeV/beamsizeRatio/ratioPhiEnProtons.pdf}
  \includegraphics[width=0.411\textwidth]{figures/4TeV/beamsizeRatio/ratioPhiNMuonsE100.pdf}
  \includegraphics[width=0.411\textwidth]{figures/4TeV/beamsizeRatio/ratioPhiEnMuons.pdf}
\end{center}
\vspace{-0.6cm}
 \caption{Effect of beam size in 4 TeV beam-gas: azimuthal distributions of multiplicity (left) of high energy protons (top) and muons (bottom) and all proton and muon energies. 
  \label{bsRatioPhiMP}}
\end{figure}

\begin{figure}%[!htb]
\begin{center}
  \includegraphics[width=0.24\textwidth]{figures/4TeV/beamsizeRatio/ratioPhiEnPrZ1.pdf}
  \includegraphics[width=0.24\textwidth]{figures/4TeV/beamsizeRatio/ratioPhiEnPrZ2.pdf}
  \includegraphics[width=0.24\textwidth]{figures/4TeV/beamsizeRatio/ratioPhiEnPrZ3.pdf}
  \includegraphics[width=0.24\textwidth]{figures/4TeV/beamsizeRatio/ratioPhiEnPrZ4.pdf}
\end{center}
\vspace{-0.6cm}
 \caption{Azimuthal energy distributions of protons in different s-regions.
  \label{bsZPr}}
\end{figure}

\begin{figure}%[!htb]
\begin{center}
  \includegraphics[width=0.411\textwidth]{figures/4TeV/beamsizeRatio/ratioEkinAll.pdf}
  \includegraphics[width=0.411\textwidth]{figures/4TeV/beamsizeRatio/ratioEkinMuons.pdf}
  \includegraphics[width=0.411\textwidth]{figures/4TeV/beamsizeRatio/ratioEkinNeutrons.pdf}
  \includegraphics[width=0.411\textwidth]{figures/4TeV/beamsizeRatio/ratioEkinProtons.pdf}
\end{center}
\vspace{-0.6cm}
 \caption{Effect of beam size in 4 TeV beam-gas: kinetic energy of particles.
  \label{bsRatioEkin}}
\end{figure}

\begin{figure}%[!htb]
\begin{center}
  \includegraphics[width=0.411\textwidth]{figures/4TeV/beamsizeRatio/ratioRadNAll.pdf}
  \includegraphics[width=0.411\textwidth]{figures/4TeV/beamsizeRatio/ratioRadNMuons.pdf}
  \includegraphics[width=0.411\textwidth]{figures/4TeV/beamsizeRatio/ratioRadNNeutrons.pdf}
  \includegraphics[width=0.411\textwidth]{figures/4TeV/beamsizeRatio/ratioRadNProtons.pdf}
\end{center}
\vspace{-0.6cm}
 \caption{Effect of beam size in 4 TeV beam-gas: radius for different particle types.
  \label{bsRatioRadN}}
\end{figure}

\begin{figure}%[!htb]
\begin{center}
  \includegraphics[width=0.411\textwidth]{figures/4TeV/beamsizeRatio/ratioRadEnAll.pdf}
  \includegraphics[width=0.411\textwidth]{figures/4TeV/beamsizeRatio/ratioRadEnMuons.pdf}
  \includegraphics[width=0.411\textwidth]{figures/4TeV/beamsizeRatio/ratioRadEnNeutrons.pdf}
  \includegraphics[width=0.411\textwidth]{figures/4TeV/beamsizeRatio/ratioRadEnProtons.pdf}
\end{center}
\vspace{-0.6cm}
 \caption{Effect of beam size in 4 TeV beam-gas: energy in $r$ of different particle types.
  \label{bsRatioRadEn}}
\end{figure}


\newpage

% -----------------------------------------------------------------------------------------------------
% beamgas before and after reweightening


\begin{figure}
\begin{center}
  \includegraphics[width=0.75\textwidth]{figures/4TeV/reweighted/cv81_OrigZMuon_BG_4TeV_20MeV_bs}
  \includegraphics[width=0.75\textwidth]{figures/4TeV/reweighted/cv81_OrigZPhotons_BG_4TeV_20MeV_bs}
  \includegraphics[width=0.75\textwidth]{figures/4TeV/reweighted/cv81_OrigZProtons_BG_4TeV_20MeV_bs}
\end{center}
\vspace{-0.6cm}
 \caption{Origin of muon (top), photon (middle) and proton (bottom) production along s, in black before and in pink after re-normalising to the 2012 pressure profile. 
  \label{fig:OrigZ4TeV2}}
\end{figure}

\begin{figure}
\begin{center}
  %% \includegraphics[width=0.45\textwidth]{figures/4TeV/reweighted/cv81_EkinAll_BG_4TeV_20MeV_bs}
  %% \includegraphics[width=0.45\textwidth]{figures/4TeV/reweighted/cv81_EkinMuons_BG_4TeV_20MeV_bs}
  %% \includegraphics[width=0.45\textwidth]{figures/4TeV/reweighted/cv81_PhiEnAll_BG_4TeV_20MeV_bs}
  %% \includegraphics[width=0.45\textwidth]{figures/4TeV/reweighted/cv81_PhiEnMuons_BG_4TeV_20MeV_bs}
  \includegraphics[width=0.45\textwidth]{figures/4TeV/reweighted/cv81_RadNAll_BG_4TeV_20MeV_bs}
  \includegraphics[width=0.45\textwidth]{figures/4TeV/reweighted/cv81_RadNMuons_BG_4TeV_20MeV_bs}
  \includegraphics[width=0.45\textwidth]{figures/4TeV/reweighted/cv81_RadEnAll_BG_4TeV_20MeV_bs}
  \includegraphics[width=0.45\textwidth]{figures/4TeV/reweighted/cv81_RadEnMuons_BG_4TeV_20MeV_bs}
\end{center}
\vspace{-0.6cm}
 \caption{From top to bottom for all particles (left) and muons (right): transverse radius $r$ and energy in $r$ before reweightening (black) and reweighted to pressure profile (pink). The black curve are scaled up to better compare the shapes. 
  \label{fig:cv81EkinPhiEn4TeV2}} 
\end{figure}


\begin{figure}
\begin{center}
  \includegraphics[width=0.45\textwidth]{figures/4TeV/reweighted/cv81_EkinProtons_BG_4TeV_20MeV_bs}
  \includegraphics[width=0.45\textwidth]{figures/4TeV/reweighted/cv81_EkinNeutrons_BG_4TeV_20MeV_bs}
  \includegraphics[width=0.45\textwidth]{figures/4TeV/reweighted/cv81_PhiEnProtons_BG_4TeV_20MeV_bs}
  \includegraphics[width=0.45\textwidth]{figures/4TeV/reweighted/cv81_PhiEnNeutrons_BG_4TeV_20MeV_bs}
  \includegraphics[width=0.45\textwidth]{figures/4TeV/reweighted/cv81_RadNProtons_BG_4TeV_20MeV_bs}
  \includegraphics[width=0.45\textwidth]{figures/4TeV/reweighted/cv81_RadNNeutrons_BG_4TeV_20MeV_bs}
  \includegraphics[width=0.45\textwidth]{figures/4TeV/reweighted/cv81_RadEnProtons_BG_4TeV_20MeV_bs}
  \includegraphics[width=0.45\textwidth]{figures/4TeV/reweighted/cv81_RadEnNeutrons_BG_4TeV_20MeV_bs}
\end{center}
\vspace{-0.6cm}
 \caption{Similar as previous figure but for protons (left) and neutrons (right): energy spectrum, energy in $\phi$, transverse radius $r$, and energy in $r$ before reweightening (black) and reweighted to pressure profile. The black curve are scaled up to better compare the shapes. 
   \label{fig:cv81ProtNeut4TeV}}
\end{figure}

% -----------------------------------------------------------------------------------------------------
% offmomentum


\begin{figure}
  \begin{center}
    \includegraphics[width=0.49\textwidth]{figures/4TeV/offmom/20MeV/Ekin_offplus500Hz_4TeV_B2_20MeV.pdf}
    \includegraphics[width=0.49\textwidth]{figures/4TeV/offmom/20MeV/PhiEnDist_offplus500Hz_4TeV_B2_20MeV.pdf}
  \includegraphics[width=0.49\textwidth]{figures/4TeV/offmom/20MeV/RadNDist_offplus500Hz_4TeV_B2_20MeV.pdf}
  \includegraphics[width=0.49\textwidth]{figures/4TeV/offmom/20MeV/RadEnDist_offplus500Hz_4TeV_B2_20MeV.pdf}

\end{center}
\vspace{-0.6cm}
 \caption{Off-momentum induced particle distributions.
  \label{offmom4TeV2}}
\end{figure}




\begin{figure}%[!htb]
\begin{center}
  \includegraphics[width=0.30\textwidth]{figures/4TeV/offmom/comppm500Hz/ratioEkinMuons.pdf}
  \includegraphics[width=0.30\textwidth]{figures/4TeV/offmom/comppm500Hz/ratioEkinProtons.pdf}
  \includegraphics[width=0.30\textwidth]{figures/4TeV/offmom/comppm500Hz/ratioEkinNeutrons.pdf}
  \includegraphics[width=0.30\textwidth]{figures/4TeV/offmom/comppm500Hz/ratioEkinPhotons.pdf}
  \includegraphics[width=0.30\textwidth]{figures/4TeV/offmom/comppm500Hz/ratioEkinElecPosi.pdf}
\end{center}
\vspace{-0.6cm}
 \caption{Comparison of IR3 cleaning induced showers generated with a positive and negative frequency shift.
  \label{compPM_ekin}}
\end{figure}

\begin{figure}%[!htb]
\begin{center}
  \includegraphics[width=0.30\textwidth]{figures/4TeV/offmom/comppm500Hz/ratioPhiNAll.pdf}
  \includegraphics[width=0.30\textwidth]{figures/4TeV/offmom/comppm500Hz/ratioPhiNProtons.pdf}
  \includegraphics[width=0.30\textwidth]{figures/4TeV/offmom/comppm500Hz/ratioPhiNMuons.pdf}

  \includegraphics[width=0.30\textwidth]{figures/4TeV/offmom/comppm500Hz/ratioPhiEnProtons.pdf}
  \includegraphics[width=0.30\textwidth]{figures/4TeV/offmom/comppm500Hz/ratioPhiEnMuons.pdf}
  \includegraphics[width=0.30\textwidth]{figures/4TeV/offmom/comppm500Hz/ratioRadNAll.pdf}
  \includegraphics[width=0.30\textwidth]{figures/4TeV/offmom/comppm500Hz/ratioRadNProtons.pdf}
  \includegraphics[width=0.30\textwidth]{figures/4TeV/offmom/comppm500Hz/ratioRadNMuons.pdf}
  \includegraphics[width=0.30\textwidth]{figures/4TeV/offmom/comppm500Hz/ratioRadEnAll.pdf}
  \includegraphics[width=0.30\textwidth]{figures/4TeV/offmom/comppm500Hz/ratioRadEnProtons.pdf}
  \includegraphics[width=0.30\textwidth]{figures/4TeV/offmom/comppm500Hz/ratioRadEnMuons.pdf}
\end{center}
\vspace{-0.6cm}
\caption{Comparisons of particle distributions in the ``plus'' and ``minus'' cases showing all particles (left colum), protons (middle) and muons (right). 
The first two rows highlight multiplicity and energy in $\phi$, and the last two rows shows also multiplicity and energy in $r$. Since the radii are not inclusive distributions (everything with $r~>~600~$cm is cut off) unlike the $\phi$-distributions the integral ratios - all being below 1 - indicate that the ``plus-case'' gives higher contributions to smaller radii or differently said, the ``minus-case'' produces more energetic particles only for $r > 600~$cm. 
  \label{compPM_phien}}
\end{figure}
\newpage


\begin{figure}%[!htb]
\centering
\includegraphics[width=0.49\textwidth]{figures/BH_run2/b2/RadNDist_BH_6500GeV_haloB2_20MeV.pdf}
\includegraphics[width=0.49\textwidth]{figures/BH_run2/b2/RadEnDist_BH_6500GeV_haloB2_20MeV.pdf}
\includegraphics[width=0.49\textwidth]{figures/BH_run2/b2/OrigYZMuons_BH_6500GeV_haloB2_20MeV.pdf}
\includegraphics[width=0.49\textwidth]{figures/BH_run2/b2/OrigXYMuons_BH_6500GeV_haloB2_20MeV.pdf}
 \caption{B2 halo induced background at the interface plane. 
  \label{dist6500GeVB22}}
\end{figure}

\begin{figure}%[!htb]
\begin{center}
  \includegraphics[width=0.411\textwidth]{figures/BH_run2/perTCThit/ratioEkinMuons.pdf}
  \includegraphics[width=0.411\textwidth]{figures/BH_run2/perTCThit/ratioPhiNMuons.pdf}
  \includegraphics[width=0.411\textwidth]{figures/BH_run2/perTCThit/ratioPhiEnMuons.pdf}
  \includegraphics[width=0.411\textwidth]{figures/BH_run2/perTCThit/ratioRadEnMuons.pdf}
\end{center}
\vspace{-0.6cm}
 \caption{Comparison of B1/B2 halo induced distributions per TCT hit for muons.
  \label{compBHB1B2run2}}
\end{figure}

\newpage

\begin{figure}%[!htb]
\begin{center}
%  \includegraphics[width=0.49\textwidth]{figures/6500GeV/20MeV/Ekin_BG_6500GeV_flat_20MeV_bs.pdf}
%  \includegraphics[width=0.49\textwidth]{figures/6500GeV/20MeV/PhiEnDist_BG_6500GeV_flat_20MeV_bs.pdf}
  \includegraphics[width=0.49\textwidth]{figures/6500GeV/20MeV/RadNDist_BG_6500GeV_flat_20MeV_bs.pdf}
  \includegraphics[width=0.49\textwidth]{figures/6500GeV/20MeV/RadEnDist_BG_6500GeV_flat_20MeV_bs.pdf}
\end{center}
\vspace{-0.6cm}
 \caption{Characteristic beam-gas induced distributions at 6.5~TeV per BG interaction using the more realistic model of the beam size.
  \label{bg65002}}
\end{figure}

\begin{figure}
\begin{center}
  \includegraphics[width=0.75\textwidth]{figures/6500GeV/reweighted/cv81_OrigZAll_BG_6500GeV_flat_20MeV_bs.pdf}
%  \includegraphics[width=0.75\textwidth]{figures/6500GeV/reweighted/cv81_OrigZMuon_BG_6500GeV_flat_20MeV_bs.pdf}
  \includegraphics[width=0.75\textwidth]{figures/6500GeV/reweighted/cv81_OrigZProtons_BG_6500GeV_flat_20MeV_bs.pdf}
  \includegraphics[width=0.75\textwidth]{figures/6500GeV/reweighted/cv81_OrigZPhotons_BG_6500GeV_flat_20MeV_bs.pdf}
\end{center}
\vspace{-0.6cm}
 \caption{Origin of all particles (top), protons (middle) and photons (bottom) production along s, in black before and in gold after re-normalising to the 2015 pressure profile. 
  \label{fig:OrigZ6p52}}
\end{figure}


\begin{figure}
\begin{center}
  \includegraphics[width=0.45\textwidth]{figures/6500GeV/reweighted/cv81_EkinAll_BG_6500GeV_flat_20MeV_bs}
  \includegraphics[width=0.45\textwidth]{figures/6500GeV/reweighted/cv81_PhiEnAll_BG_6500GeV_flat_20MeV_bs}
  \includegraphics[width=0.45\textwidth]{figures/6500GeV/reweighted/cv81_RadNAll_BG_6500GeV_flat_20MeV_bs}
%  \includegraphics[width=0.45\textwidth]{figures/6500GeV/reweighted/cv81_RadNMuons_BG_6500GeV_flat_20MeV_bs}
  \includegraphics[width=0.45\textwidth]{figures/6500GeV/reweighted/cv81_RadEnAll_BG_6500GeV_flat_20MeV_bs}
 % \includegraphics[width=0.45\textwidth]{figures/6500GeV/reweighted/cv81_RadEnMuons_BG_6500GeV_flat_20MeV_bs}
\end{center}
\vspace{-0.6cm}
 \caption{The distributions 
  \label{fig:EkinPhiEn6p52}}
\end{figure}



\begin{figure}
\begin{center}
  \includegraphics[width=0.45\textwidth]{figures/6500GeV/reweighted/cv81_EkinProtons_BG_6500GeV_flat_20MeV_bs}
  \includegraphics[width=0.45\textwidth]{figures/6500GeV/reweighted/cv81_EkinNeutrons_BG_6500GeV_flat_20MeV_bs}
  \includegraphics[width=0.45\textwidth]{figures/6500GeV/reweighted/cv81_PhiEnProtons_BG_6500GeV_flat_20MeV_bs}
  \includegraphics[width=0.45\textwidth]{figures/6500GeV/reweighted/cv81_PhiEnNeutrons_BG_6500GeV_flat_20MeV_bs}
  \includegraphics[width=0.45\textwidth]{figures/6500GeV/reweighted/cv81_RadNProtons_BG_6500GeV_flat_20MeV_bs}
  \includegraphics[width=0.45\textwidth]{figures/6500GeV/reweighted/cv81_RadNNeutrons_BG_6500GeV_flat_20MeV_bs}
  \includegraphics[width=0.45\textwidth]{figures/6500GeV/reweighted/cv81_RadEnProtons_BG_6500GeV_flat_20MeV_bs}
  \includegraphics[width=0.45\textwidth]{figures/6500GeV/reweighted/cv81_RadEnNeutrons_BG_6500GeV_flat_20MeV_bs}
\end{center}
\vspace{-0.6cm}
 \caption{The distributions 
  \label{fig:ProtNeut6p52}} 
\end{figure}

% ----------------------------------
\newpage
\clearpage

