\section{Simulation Results for HL--LHC\label{hllhcResults}}

%Several cases were simulated in order to characterise the cleaning efficiency for baseline settings of HL-LHC, deploying different collimator layouts (in IR1/5 TCT4s only and TCT4s + TCT5s) and alternative collimator settings, so called \twosigmaret~settings to quantify the effect of the TCT5s for incoming beams (B1 and B2). Inelastic interactions with beam protons are forced in FLUKA at initial conditions given by SixTrack on the TCT4s and (when included) TCT5s. These interactions generate a particle flux towards the experiment. All shower particles are recorded at the machine-detector interface plane at 22.6~m from the IP using in FLUKA a production and transportation cut-off at 20 MeV.

This section describes the several case studies for HL, focussing on beam-halo. As mentioned earlier, the HL beam-gas simulations as described in Ref.~\cite{ipac2014_rkh} are still the most recent of this kind.

\subsection{Beam-Halo}

The aim was to study loss locations for different collimator layouts and settings and evaluate quantitatively on halo background in IR1. The simulation technique as described in Sec.~\ref{simSetup} was performed in detail for the more realistic collimator settings scenario, so-called \twosigmaret~settings, and all cases are listed in Tab.~\ref{hlscenario}. The baseline collimator layout contains the TCT4s and TCT5s, although the position may not be fixed to the meter precise for the moment due to lack of sufficient space at the positions as shown in Fig.~\ref{hllhc_layout}. 

Two different HL optics (version HLLHCv1.0\footnote{The FLUKA geometry contained layout updates of v1.1, but v1.0 collimator settings were used assuming the uncertainties from the version differences are negligable.}) were studied:
The current baseline of ATS optics for $\beta^{*}$ of 15~cm for round beams is used in the simulations.
Another ATS optics scenario was also studied with the purpose of validating a new HL collimation layout. To maintain flexibility in low-$\beta^*$ reach, e.g~in the case of crab cavity failures in the experimental IRs, flat beam optics were developed. While in round optics, $\beta^*$ is 15~cm in IP1/5 in the horizontal and vertical plane, it is for flat beams 7.5~cm in the horizontal plane and 30~cm in the vertical plane at IP1 (vice versa for IP5)~\cite{opticsWebRef}. 


\begin{table}%[hbt]
   \centering
   \caption{Beam-halo simulation cases in SixTrack (HLLHC v1.0).}\vskip2mm
   \begin{tabular}{|l|l|l|l|}
       \hline
       collimator settings & TCT5s & beam halo & optics \\
       \hline\hline
       updated nominal  & out & h+v B1 & round \\
       updated nominal  & in & h+v B1 & round \\\hline
       \twosigmaret & out & h+v B1 & round \\ 
       \twosigmaret & in  & h+v B1 & round \\ 
       \twosigmaret & out & h+v B2 & round \\
       \twosigmaret & in  & h+v B2 & round \\ \hline
       \twosigmaret & out  & h+v B1 & flat \\
       \twosigmaret & in  & h+v B1 & flat \\ 
%       \twosigmaret & in  & h+v B1 & sround \\ 
%       \twosigmaret & in  & h+v B1 & sflat \\ 

       \hline

   \end{tabular}
   \label{hlscenario}
\end{table}

\subsubsection{Losses at IR1/IR5 tertiary collimators in HL--LHC}

A zoom into loss locations in the experimental IRs of ATLAS and CMS are shown in Fig.~\ref{IR15_roundB1_nomSett} for nominal collimator settings. The similar set of zooms are shown and discussed for the \twosigmaret~settings in the Appendix~\ref{lossmapszooms}. The beam direction is from left to right. The two black bars at upstream of the IP are the losses on the pair of tertiary collimators (TCT4s and TCT5s), while the black bars downstream are losses on TCLs, debris collimators. These losses are normalised to total number of lost particles. One can see, IR5 losses are generally lower than in IR1, an expected feature known from Run I and II. A more direct comparison of the losses is made in Fig.~\ref{compTCT5INOUT}. The losses are normalised to the number of simulated primary per simulation case, then the sum per collimator is shown, i.e. as example the bin of a specific TCTH is set to $\big(\frac{\mathrm{hits\,on\,TCTH}}{\#\mathrm{primary}}\big)_{\mathrm{h}} + \big(\frac{\mathrm{hits\,on\,TCTH}}{\#\mathrm{primary}}\big)_{\mathrm{v}}$. The simulation cases shown are in Fig.~\ref{compTCT5INOUT}~(a) for round, and Fig.~\ref{compTCT5INOUT}~(b) for flat beams.

\begin{figure} [!htb]
\begin{center}

\includegraphics[width=0.48\textwidth]{figures/lossmaps/coll_loss_H5_HL_nomSett_hHalo_b1_IR1}
\includegraphics[width=0.48\textwidth]{figures/lossmaps/coll_loss_H5_HL_nomSett_vHalo_b1_IR1}
\includegraphics[width=0.48\textwidth]{figures/lossmaps/coll_loss_H5_HL_nomSett_hHalo_b1_IR5}
\includegraphics[width=0.48\textwidth]{figures/lossmaps/coll_loss_H5_HL_nomSett_vHalo_b1_IR5}
\end{center}
\vspace{-0.3cm}
 \caption{Zoom into IR1 (top) and IR5 (bottom, IP5 is at 133,300~m) when the TCT5s were inserted using round optics. Horizontal beam 1 is on the left, vertical beam 1 on the right.
  \label{IR15_roundB1_nomSett}}
\end{figure}



\begin{figure}[!htb]
\begin{center}
\includegraphics[width=0.495\textwidth]{figures/inelposition_sum_tct5otrd.pdf}
\includegraphics[width=0.495\textwidth]{figures/inelposition_sum_tcotrdb2.pdf}

\end{center}
\vspace{-0.6cm}
 \caption{Positions of inelastic interactions as given by SixTrack within the collimator jaws.
  \label{inelHLtct5in}}
\end{figure}


\begin{figure}[!htb]
\begin{center}
\includegraphics[width=0.495\textwidth]{figures/inelposition_sum_tct5inrd.pdf}
\includegraphics[width=0.495\textwidth]{figures/inelposition_sum_tcinrdb2.pdf}
\end{center}
\vspace{-0.6cm}
 \caption{Positions of inelastic interactions as given by SixTrack within the collimator jaws.
  \label{inelHLtct5in}}
\end{figure}



\begin{figure}[!htb]
\begin{center}
\includegraphics[width=0.495\textwidth]{figures/compTCT5LINOUT_roundthin_B1_IR1IR5}
\includegraphics[width=0.495\textwidth]{figures/compTCT5LINOUT_flatthin_B1_IR1IR5}
\end{center}
\begin{picture} (0.,0.)
\setlength{\unitlength}{1.0cm}
\small{
    \put ( 4.,1.){(a)}
    \put ( 12.4,1.){(b)}
}
\end{picture}
\vspace{-0.6cm}
 \caption{Summary of hits load of the single TCTs for the cases TCT5 are out and in for round (a, $\beta^*_{\textrm{x,y}}$ is 15~cm) and flat (b, $\beta^*_{\textrm{x/y}}$ = 7.5~cm, $\beta^*_{\textrm{y/x}}$ = 30~cm) beam optics.
  \label{compTCT5INOUT}}
\end{figure}

\begin{figure}[tbh]
    \centering
    \includegraphics[width=0.5\textwidth]{figures/TUPTY067f3}
    \vspace{-0.5cm}
    \caption{Comparison of losses on TCTs for round and flat beam optics.}
    \label{compOptics}
\end{figure}

\subsubsection{Comparison of round and flat beam optics}
We focus on the different loads of TCT4s and TCT5s in IR1 and IR5 and are interested in how the TCT hits are shared amongst them. Both cases, having TCT5s in and out, are compared for B1 round beam optics in IR1 and IR5 in Fig.~\ref{compTCT5INOUT}~(a). One can observe as expected that the TCT5s take over a large fraction of halo protons when they are included. In return, nearly a factor 5 less load on the TCT4s in IR1 can be expected. However, considering the losses of TCTH.4 from Fig.~\ref{compTCT5INOUT}~(a), there are a factor 4 less hits when TCT5s are in IR1, and about a factor 6 in IR5. This gain will be decreased by contributions from TCT5s. Even more significant is the difference in IR5 with about 25 times less losses at the TCT4s. We also find that B1 losses are smaller in IR5 than in IR1. This can be expected for B1 since the halo protons that leak through the cleaning system of IR7 have a much shorter distance to travel to IR1 than to IR5. We also note that a slight increase of about 8~\% of intercepted losses is found when both TCT4s and TCT5s are deployed. Since more particles are intercepted, more shower particles can be created which also affects the background level. %Detailed shower simulations with FLUKA provide a first estimate of that background reduction.


\subsubsection{Effect of TCT5s on halo background in IR1}

%We studied also how the TCT load changes when the optics change from round to flat for the case that the TCT5s are included. The result is shown in Fig.~\ref{compOptics} for the separate TCTs in IR1/5. In the figure, one can see per IR, that the number of hits are very similar for round and flat beam except for TCTV.4 in IR1 where a flat B1 would create about a factor 4 more hits. When summing the hits of TCT4s and TCT5s, the number of TCT hits are very similar and so possibly is also the background level in IR1 and potentially even slightly better in IR5.

% ------------------------------------------------------------------------------------------
% 

\begin{figure}
\begin{center}
\includegraphics[width=0.43\textwidth]{figures/HL/tct5inrd/Ekin_BH_HL_tct5inrdB1_20MeV.pdf}
\includegraphics[width=0.43\textwidth]{figures/HL/tct5inrd/PhiEnDist_BH_HL_tct5inrdB1_20MeV.pdf}
\includegraphics[width=0.43\textwidth]{figures/HL/tct5inrd/RadNDist_BH_HL_tct5inrdB1_20MeV.pdf}
\includegraphics[width=0.43\textwidth]{figures/HL/tct5inrd/RadEnDist_BH_HL_tct5inrdB1_20MeV.pdf}
\end{center}
%% \begin{picture} (0.,0.)
%% \setlength{\unitlength}{1.0cm}
%% \small{
%%     \put ( 4.,7.35){(a)}
%%     \put ( 12.4,7.35){(b)}
%%     \put ( 4.,1.){(c)}
%%     \put ( 12.4,1.){(d)}
%% }
%% \end{picture}
\vspace{-0.6cm}
 \caption{Particle distribution at the interface plane.}
  \label{tct5inrdb1retr}
\end{figure}


\begin{figure}
\begin{center}
\includegraphics[width=0.495\textwidth]{figures/OrigYZMuons_BH_HL_tct5otrdB1_20MeV}
\includegraphics[width=0.495\textwidth]{figures/OrigYZMuonsE100_BH_HL_tct5otrdB1_20MeV}
\includegraphics[width=0.495\textwidth]{figures/OrigYZMuons_BH_HL_tct5inrdB1_20MeV}
\includegraphics[width=0.495\textwidth]{figures/OrigYZMuonsE100_BH_HL_tct5inrdB1_20MeV}
\end{center}
\begin{picture} (0.,0.)
\setlength{\unitlength}{1.0cm}
\small{
    \put ( 4.,7.35){(a)}
    \put ( 12.4,7.35){(b)}
    \put ( 4.,1.){(c)}
    \put ( 12.4,1.){(d)}
}
\end{picture}
\vspace{-0.6cm}
 \caption{Origin of muons for all energies (a,c) and for an energy above 100~GeV (b,d) in the y-z plane with TCT5 out (a,b) and TCT5 in (c,d).
  \label{OrigMuonE}}
\end{figure}

% ------------------------------------------------------------------------------------------
% comparisons
In a further simulation step, we evaluate the actual change in particle flux at the interface plane in IR1 using \fluka. Although the load on TCT4s is reduced by about a factor 4 for round B1 optics, shower particles created at TCT5s contribute to halo-induced background as well. We show in Fig.~\ref{tct5inrdb1retr} the case \twosigmaret~settings and TCT5s included. We compare to the case with the same collimator settings but TCT4s only in Fig.~\ref{compTCT5inoutEkin} the energy spectra for all particles and muons only reaching the interface location. Their transversal radial distribution is shown in Fig.~\ref{compRadN}. One can see in the ratio of the top plot of Fig.~\ref{Ekin} that all particles will be reduced except those with an energy reaching the beam energy when TCT5s are also installed. The integral ratio of the total number of particles indicates that close to 2 times less particles reach the interface plane when the TCT5s are in. The bottom plot of Fig.~\ref{Ekin} shows the number of muons with an energy of 100~GeV decreases, however the number of higher energy muons will rather increase by about 20~\% and even more. The lower bins of Fig.~\ref{compRadN} represent the space close to the beampipe. They will be up to a factor 3 less populated (almost entirely due to photons and electrons, not shown) if the TCT5s are in. The bottom plot of Fig.~\ref{compRadN} shows that muon distributions and their ratios feature a very similar shape.

\begin{figure}
\begin{center}
\includegraphics[width=0.43\textwidth]{figures/HL/compINOUTB1_retracted/perTCThit/ratioEkinAll}
\includegraphics[width=0.43\textwidth]{figures/HL/compINOUTB1_retracted/perTCThit/ratioEkinMuons}
\end{center}
\begin{picture} (0.,0.)
\setlength{\unitlength}{1.0cm}
\small{
    \put ( 4.,1.){(a)}
    \put ( 12.4,1.){(b)}
}
\end{picture}
\vspace{-0.6cm}
 \caption{Energy distribution for all particles (a) and muons (b) at the interface plane.
  \label{compTCT5inoutEkin}}
\end{figure}



\begin{figure}
\begin{center}
\includegraphics[width=0.43\textwidth]{figures/HL/compINOUTB1_retracted/perTCThit/ratioPhiNAll}
\includegraphics[width=0.43\textwidth]{figures/HL/compINOUTB1_retracted/perTCThit/ratioPhiNMuons}
\includegraphics[width=0.43\textwidth]{figures/HL/compINOUTB1_retracted/perTCThit/ratioPhiEnAll}
\includegraphics[width=0.43\textwidth]{figures/HL/compINOUTB1_retracted/perTCThit/ratioPhiEnMuons}
\end{center}
%% \begin{picture} (0.,0.)
%% \setlength{\unitlength}{1.0cm}
%% \small{
%%     \put ( 4.,1.){(a)}
%%     \put ( 12.4,1.){(b)}
%% }
%% \end{picture}
\vspace{-0.6cm}
 \caption{Azimuthal distribution of all particles and muons (top) and their energy (bottom).
  \label{compPhi}}
\end{figure}

\begin{figure}
\begin{center}
\includegraphics[width=0.43\textwidth]{figures/HL/compINOUTB1_retracted/perTCThit/ratioRadNAll}
\includegraphics[width=0.43\textwidth]{figures/HL/compINOUTB1_retracted/perTCThit/ratioRadNMuons}
\includegraphics[width=0.43\textwidth]{figures/HL/compINOUTB1_retracted/perTCThit/ratioRadEnAll}
\includegraphics[width=0.43\textwidth]{figures/HL/compINOUTB1_retracted/perTCThit/ratioRadEnMuons}
\end{center}
%% \begin{picture} (0.,0.)
%% \setlength{\unitlength}{1.0cm}
%% \small{
%%     \put ( 4.05,8.95){(a)}
%%     \put ( 12.45,8.95){(b)}
%%     \put ( 4.05,1.){(c)}
%%     \put ( 12.45,1.){(d)}
%% }
%% \end{picture}
\vspace{-0.6cm}
 \caption{Tranverse radial distribution of all particles and muons (top) and their energy (bottom).
  \label{compRad}}
\end{figure}

\subsubsection{Comparing layout for B2}
\begin{figure}
\begin{center}
\includegraphics[width=0.43\textwidth]{figures/HL/compINOUTB2_retracted/perTCThit/ratioEkinAll}
\includegraphics[width=0.43\textwidth]{figures/HL/compINOUTB2_retracted/perTCThit/ratioEkinMuons}
\end{center}
\begin{picture} (0.,0.)
\setlength{\unitlength}{1.0cm}
\small{
    \put ( 4.,1.){(a)}
    \put ( 12.4,1.){(b)}
}
\end{picture}
\vspace{-0.6cm}
 \caption{Energy distribution for all particles (a) and muons (b) at the interface plane.
  \label{Ekin}}
\end{figure}



\begin{figure}
\begin{center}
\includegraphics[width=0.43\textwidth]{figures/HL/compINOUTB2_retracted/perTCThit/ratioPhiNAll}
\includegraphics[width=0.43\textwidth]{figures/HL/compINOUTB2_retracted/perTCThit/ratioPhiNMuons}
\includegraphics[width=0.43\textwidth]{figures/HL/compINOUTB2_retracted/perTCThit/ratioPhiEnAll}
\includegraphics[width=0.43\textwidth]{figures/HL/compINOUTB2_retracted/perTCThit/ratioPhiEnMuons}
\end{center}
%% \begin{picture} (0.,0.)
%% \setlength{\unitlength}{1.0cm}
%% \small{
%%     \put ( 4.,1.){(a)}
%%     \put ( 12.4,1.){(b)}
%% }
%% \end{picture}
\vspace{-0.6cm}
 \caption{Azimuthal distribution of all particles and muons (top) and their energy (bottom).
  \label{compPhi}}
\end{figure}

\begin{figure}
\begin{center}
\includegraphics[width=0.43\textwidth]{figures/HL/compINOUTB2_retracted/perTCThit/ratioRadNAll}
\includegraphics[width=0.43\textwidth]{figures/HL/compINOUTB2_retracted/perTCThit/ratioRadNMuons}
\includegraphics[width=0.43\textwidth]{figures/HL/compINOUTB2_retracted/perTCThit/ratioRadEnAll}
\includegraphics[width=0.43\textwidth]{figures/HL/compINOUTB2_retracted/perTCThit/ratioRadEnMuons}
\end{center}
%% \begin{picture} (0.,0.)
%% \setlength{\unitlength}{1.0cm}
%% \small{
%%     \put ( 4.05,8.95){(a)}
%%     \put ( 12.45,8.95){(b)}
%%     \put ( 4.05,1.){(c)}
%%     \put ( 12.45,1.){(d)}
%% }
%% \end{picture}
\vspace{-0.6cm}
 \caption{Tranverse radial distribution of all particles and muons (top) and their energy (bottom).
  \label{compRad}}
\end{figure}



\subsubsection{Comparison of background with different collimator settings}


\begin{figure}
\begin{center}
\includegraphics[width=0.43\textwidth]{figures/HL/compNomRetrCollSett/perTCThit/ratioEkinAll.pdf}
\includegraphics[width=0.43\textwidth]{figures/HL/compNomRetrCollSett/perTCThit/ratioPhiEnAll.pdf}
\includegraphics[width=0.43\textwidth]{figures/HL/compNomRetrCollSett/perTCThit/ratioRadNAll.pdf}
\includegraphics[width=0.43\textwidth]{figures/HL/compNomRetrCollSett/perTCThit/ratioRadEnAll.pdf}
\end{center}
%% \begin{picture} (0.,0.)
%% \setlength{\unitlength}{1.0cm}
%% \small{
%%     \put ( 4.05,8.95){(a)}
%%     \put ( 12.45,8.95){(b)}
%%     \put ( 4.05,1.){(c)}
%%     \put ( 12.45,1.){(d)}
%% }
%% \end{picture}
\vspace{-0.6cm}
 \caption{Comparison background showers with nominal and \twosigmaret~collimator settings (see Tab.~\ref{HLcollSettings}).
  \label{compNomRetrSett}}
\end{figure}

\subsubsection{Comparison of B1 and B2 induced showers with TCT5s in}


\begin{figure}
\begin{center}
\includegraphics[width=0.43\textwidth]{figures/HL/compINB1B2/perTCThit/ratioEkinAll.pdf}
\includegraphics[width=0.43\textwidth]{figures/HL/compINB1B2/perTCThit/ratioPhiEnAll.pdf}
\includegraphics[width=0.43\textwidth]{figures/HL/compINB1B2/perTCThit/ratioRadNAll.pdf}
\includegraphics[width=0.43\textwidth]{figures/HL/compINB1B2/perTCThit/ratioRadEnAll.pdf}
\end{center}
%% \begin{picture} (0.,0.)
%% \setlength{\unitlength}{1.0cm}
%% \small{
%%     \put ( 4.05,8.95){(a)}
%%     \put ( 12.45,8.95){(b)}
%%     \put ( 4.05,1.){(c)}
%%     \put ( 12.45,1.){(d)}
%% }
%% \end{picture}
\vspace{-0.6cm}
 \caption{Comparison B1 and B2 induced background showers with \twosigmaret~collimator settings.
  \label{compINB1B2}}
\end{figure}
