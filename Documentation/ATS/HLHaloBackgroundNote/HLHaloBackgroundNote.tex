\documentclass{cernatsnote} % Specifies the document style.
%
%
% This texfile is a modified version from 
% http://frs.home.cern.ch/frs/Source/NL_MD_02.07.2011/NL-MD.tex
%
%
%
\usepackage{vmargin,times,graphicx,amsmath,amssymb,color} % ,draftcopy��use draftcopy for experiments
\usepackage{verbatim} % to allow for verbatim and comment
\usepackage{lineno} % to allow for linenumbers in the draft version
\usepackage{tabularx}
\usepackage{multirow}


\newcommand{\lumi}[1]{\ensuremath{{\mathcal{L}= 10^{#1}\,\mathrm{cm}^{-2}\,\mathrm{s}^{-1}}}}
\newcommand{\lumiapprox}[1]{\ensuremath{{\mathcal{L} \approx 10^{#1}\,\mathrm{cm}^{-2}\,\mathrm{s}^{-1}}}}
\newcommand{\lumif}[2]{\ensuremath{{\mathcal{L}= {#1} \times 10^{#2}\,\mathrm{cm}^{-2}\,\mathrm{s}^{-1}}}}
\newcommand{\ifb}{\mbox{fb$^{-1}$}}%  Inverse femtobarns.
\newcommand{\ipb}{\mbox{pb$^{-1}$}}%  Inverse picobarns.
\newcommand{\inb}{\mbox{nb$^{-1}$}}%  Inverse nanobarns.
\newcommand{\sph}{\mbox{$\mu$Sv/h}}%  mu Sv/h
\newcommand{\nseu}{$N_{\mathrm{SEU}}$}
\newcommand{\twosigmaret}{$2\sigma$-retracted}
\newcommand{\thnf}{\ensuremath{\Phi_{\mathrm{thn}}}}
\newcommand{\fluka}{\textsc{Fluka}}
\newcommand{\mypercent}{\%}

% now optionally modify manually to fill better the page (drop at end
% ?)
\setmarginsrb{22mm}{10mm}{20mm}{10mm}{12pt}{11mm}{0pt}{11mm}
%
\title{Background Simulation Studies for Run I, Run II and HL-LHC}
%
\author{R.~Kwee-Hinzmann, R.~Bruce, F.~Cerutti,~L.~S.~Esposito, S.~M.~Gibson, A.~Lechner, H.~Garcia-Morales}
%
\date{\today}
\email{\small{Regina.Kwee@rhul.ac.uk}}
%
\makeindex
%
\documentlabel{CERN-ACC-Note-2016-XXX PERF}\keywords{halo background, beam-gas, SixTrack, FLUKA, HL-LHC, TCT5, collimation, cleaning efficiency}
%
% End of preamble and beginning of text.
\begin{document}
%
% Produces the title block.
\maketitle

\renewcommand{\textfraction}{0.1}       % the minimum fraction of a text page that must be devoted to text
\renewcommand{\floatpagefraction}{0.8}  % the minimum fraction of a float page that must be occupied by floats
\renewcommand{\topfraction}{1.}         % the maximum fraction float on top, usually .7

%
%\linenumbers
% ----------------------------------------------------------------------------------------------
\begin{abstract}
%
This note summarises the simulation results on two background sources, beam-halo and local beam-gas, which induced particle showers to the experiments for several LHC and HL--LHC scenarios. %beam halo induced background at IR1 in the HL--LHC scenario using the final HL--LHC collimation layout and is essentially an extension of results presented in~\cite{thisyear}. Using the baseline optics of round beams (HLLHCv1.0), and the so-called 2$\sigma$-retracted collimator settings, SixTrack and Fluka simulations were performed to quantify the effect of the installation of additional tertiary collimators in cell 5 upstream the tertiary collimator pair that is already installed to protect the inner triplet magnets in IR1 and 5.
%In addition other flat beam optics is compared to round optics. For completion, tracking simulation results of other ultimate optics scenarios again with the 2$\sigma$-retracted collimator settings are included.
\end{abstract}
%
% ----------------------------------------------------------------------------------------------

\newpage
\tableofcontents
\newpage
\newpage

% uncomment to also show section header
%\addtocontents{toc}{\protect\setcounter{tocdepth}{1}}
%-------------------------------------------------------------------------------------------
\section{Introduction}


The LHC has finished its first run and is half way through its second run of operation. Proton-proton collisions collected by the two large experiments ATLAS and CMS have led to the long thought discovery of the Higgs particle~\cite{Aad20121,Chatrchyan201230}. To extend the discovery potential of the LHC, an upgrade is planned with even higher luminosities, HL-LHC~\cite{hl-lhc-prel-design}, and it is foreseen to take up major constructions in 2024.

In this report particle losses and background sources to experiments are discussed in all of these scenarios ranging from 4 to 6.5 TeV in LHC to 7~TeV in HL-LHC using different collimator settings and beam optics. Simulations have been performed for two of the most important sources of beam-induced background using new, improved techniques. Any interaction of beam particles with anything upstream the interaction point (IP), causing shower towards the experiment, can contribute to background.

Depending on how often signals from such interactions appear they are judged either critical or managable for operation. Often it is hard to precisely predict the rate as the it depends on machine conditions and settings, like the vacuum quality and settings of the collimators, which define the leakage to the experimental areas. Certainly, also very basic beam properties are crucial for background, e.g.~the beam life time directly correlates with the rates observable at the experiments and the lifetime is not easy to control or predict. There were also luminosity-dependent phenomena observed. In such a complex machine like the LHC with various filling schemes and different beam optics to meet various physics purposes at increasing beam energies, there is a good chance to encounter background phenomena which have not been well known from other machines, like in ATLAS where ``after-glow" or ``ghost-charge"~\cite{ATLAS_JINST_13} were observed. That is why the experiments, even if the background is judged to be not critical for operation in the current run, are motivated to study their origin and characteristics. Certain sources may become an issue in the future when machine parameters change. As experiments would like to analyse rare interaction processes, also background that does not appear at a high rate is of high interest. This enables them to study e.g.~how many high energy particles reach which part of the detector. This is vital for them to substract the background from signals that may mean a new particle. 

One can differentiate between several background sources depending on how they are created. Once typical characteristic distributions are known, it is a matter of normalisation which accounts for the specific running conditions to determine the rate. In this report, we present both, the study of shapes of background distributions at a virtual machine-detector interface plane and highlight rate estimates of two of the most important sources close by that can create background at the experiments. We discuss three sources here.

A well-known source is if beam protons hit residual gas molecules inelastically upstream and in the vicinity of the experiment producing a shower of secondaries and unwanted signals in the detector. The rate of such beam-gas events is tightly connected to the vacuum quality and thus has to be permanently monitored by both, the experiments and machine operators.

Another source of steady losses in the machine are halo protons diffusing out to high transverse amplitudes (they form a halo of the beam). Eventually they get lost before they reach one of the interaction points (IPs). In order to avoid losses in cold areas (where they can cause magnets to quench when depositing too much energy in the supraconductiong coils) an entire cleaning insertion in IR7 is dedicated to these unavoidable steady losses~\cite{LHCDesignRep,assmann05chamonix}. Halo particles escaping IR7 are even further reduced by tertiary collimators (TCTs) close to the experiments, each IR where the bunches cross, is equipped with such collimators. Anything that leaks from this system can produce a background shower for the experiment when interacting with the collimator material. 

A very similar system is installed in IR3 to clean off-momentum halo particles. These particles, when they deviate slightly from the nominal beam energy, will be bent slightly differently and run on a different orbit given by the dispersion function. The leakage out of IR3 impacts as well on the TCTs in the experimental IRs and contribute to background at the experiments. They had never been simulated as background before, we report here for the first time on their characteristics and how they compare to the other two background sources of local beam-gas and betatron halo. 

Shower distributions are evaluated at an artificial interface plane between the machine and detector, which is defined at 22.6~m from IP1. This is the main output of the studies presented in this report, and can be used as input for further background studies by the experiments within the detectors. We either highlight general particle distributions or focus on muons as experiments are usually interested in these particles. The IR1 layout of the beamline is identical to the one in IR5, but there are other differences, like the path length from cleaning insertions to IP1, the crossing plane is chosen horizontally in IR5 and vertically in IR1 (but the crossing angle in IR1 and IR5 are the same). Therefore, we study also collimator losses in IR5.

We also present details of several scenarios simulated with the final HL-LHC collimation layout. We investigate how much the shapes of particle distributions at the interface plane differ from the present LHC and what can be expected in terms of rates. As the stored beam energy increases from 360~MJ to 675~MJ, all scenarios of beam losses are more critical and one has to review dominant sources. This report highlights the evolution of background sources including future trends for the upgrade of the LHC.

%In Sect.~\ref{simSetup} we described the simulation techniques used in Sect.~\ref{run1run2} for the results for Run~1 and Run~2 as well as for scenarios in HL--LHC in Sect.~\ref{hllhcResults}. We discuss the evolution of background sources in Sect.~\ref{evolut} and give conclusions in Sect.~\ref{last}. 

\section{Simulation Setup}

\subsection{Simulation Codes}

\paragraph{Particle Tracking with SixTrack}

SixTrack is a particle tracking code that features multi-turn stability (symplecticity) for the full six dimensional phase-space. It is based on thin-lens optics (where elements have zero lengths and magnetic field strengths are split such that it is equal to the thick lens optics) and has its own Monte Carlo physics model available~\cite{K2Ref}. An extended version of SixTrack includes the LHC collimator model~\cite{SixTrackRef}, which has been used for our purposes.

\paragraph{Particle Showering with FLUKA}

To evaluate the contribution of certain background sources we use FLUKA~\cite{flukaRef1,flukaRef2}, a general purpose Monte Carlo generator that calculates particle interaction and transport within matter for a user-defined geometry. It uses modern physics models and experimental data as available by the Particle Data Group~\cite{pdgRef}.

We use a geometry built up to 550~m of the right side of IR1 for the present LHC machine (the layout is symmetric around the IP1), for HL-LHC scenarios the geometry reaches up the tertiary collimators of around 215~m. 


\subsection{Simulation Techniques}

\subsubsection{Beam-Halo simulations}
The simulation is performed in two main parts: First, we use SixTrack~\cite{SixTrackRef} to track halo proton distributions customised by the user through a magnetic field lattice, see also~\cite{chiarasThesis}. The beam halo is usually simulated in horizontal (h) and vertical (v) distributions. When a collimator is hit, a built-in, recently updated Monte-Carlo model~\cite{claudiasThesis} decides on the physics process. Protons continue in the lattice until they dissociate in an inelastic interaction with the collimator material or (in a post-processing step) are lost on the aperture. As a result, loss locations around the ring can be identified and protons lost on the TCTs serve in second step as an initial distribution in FLUKA.

\paragraph{Run I and Run II cases}
For Run I, we use the physics configuration of the LHC with $\beta^* = 60$ cm optics and collimator settings in which the IR1/IR5 TCTs were set to 9~$\sigma$~\cite{parametersRun1}. For Run II, the optics changed to $\beta^* = 80$~cm which was used in the machine throughout 2015 for proton-proton collisions and IR1/IR5 collimators were set to 13.7~$\sigma$. For both runs, a vertical crossing angle of $290~\mu$m was taken into account.

\begin{figure}[!htb]
\begin{center}
\includegraphics[width=0.9\textwidth]{figures/NominalLHC_IR1_layout.pdf}
\end{center}
\vspace{-0.6cm}
 \caption{Machine layout for the nominal LHC (as in Run I and II) of the left side of IR1 with IP1 at s = 0 for the incoming beam. Highlighted are the tertiary collimators at around -147~m.
  \label{nominalLHC_layout}}
\end{figure}


\paragraph{HL-LHC cases}
Several cases were simulated in order to characterise the cleaning efficiency for baseline settings of HL-LHC, deploying different collimator layouts (in IR1/5 TCT4s only and TCT4s + TCT5s) and alternative collimator settings, so called \twosigmaret~settings to quantify the effect of the TCT5s for incoming beams (B1 and B2). Inelastic interactions with beam protons are forced in FLUKA at initial conditions given by SixTrack on the TCT4s and (when included) TCT5s. These interactions generate a particle flux towards the experiment. All shower particles are recorded at the machine-detector interface plane at 22.6~m from the IP using in FLUKA a production and transportation cut-off at 20 MeV.

\begin{table}%[hbt]
   \centering
   \caption{HL half-gap collimator settings calculated for a normalised emittance of $\epsilon_{\mathrm{n}}$ of 3.5~$\mu$m. Full and updated settings can be found in~\cite{collSettRef}.}

   \begin{tabular}{l|c|c}
       \hline
       collimators &        nominal settings & $2\sigma$-retracted settings\\
                   &         [$\sigma$] &  [$\sigma$]\\
       \hline
       TCP3 & 12 (now 15) & 15 \\
       TCSG3 & 15.6 (now 18)& 18 \\
       TCP7 & 6 & 5.7 \\
       TCSG7 & 7 & 7.7 \\
       TCT IR1/5 & 8.3 & 10.5 \\
       \hline
   \end{tabular}
   \label{collSettings}
\end{table}

\subsubsection{Beam-Gas simulations}

In order to simulate proton beam particles colliding with residual gas molecules, we use a FLUKA model of the IR1. In this note we studied local beam-gas only sampling interactions up to 550~m away from the IP. In contrast to Ref.~\cite{nimPaperRod} where interactions were sampled along the ideal orbit of a particle, we consider in addition the variation of the transverse beam size which in particular is large just before the triplet when the beam is squeezed for collisions. 

Creating a set of initial conditions that are matched phase space coordinates in the transverse plane, their trajectories are used to sample the beam-gas interactions from. This new setup was considered for Run I 4 TeV and Run II 6.5 TeV local beam-gas simulations in IR1. In FLUKA, proton-nitrogen interactions are simulated in order to scale in a second step the contributions to the pressure profile usually expressed in $N_2$-equivalent.

\begin{figure}[!htb]
\begin{center}
\includegraphics[width=0.9\textwidth]{figures/IP1_gauss.pdf}
\includegraphics[width=0.9\textwidth]{figures/twiss_gauss.pdf}
\end{center}
\begin{picture} (0.,0.)
\setlength{\unitlength}{1.0cm}
\small{
    \put ( 4.,7.35){(a)}
    \put ( 12.4,7.35){(b)}
    \put ( 4.,1.){(c)}
    \put ( 12.4,1.){(d)}}
\end{picture}
\vspace{-0.6cm}
 \caption{Matched phase space coordinates at IP1 in x (a) and y (b) and at an example position (here TCTH).
  \label{ip1_gauss}}
\end{figure}



\begin{figure}[!htb]
\begin{center}
\includegraphics[width=0.44\textwidth]{figures/inputFluka6500GeV_yBGAS.pdf}
\includegraphics[width=0.44\textwidth]{figures/xBGAS10z1.pdf}
\end{center}
\begin{picture} (0.,0.)
\setlength{\unitlength}{1.0cm}
\small{
    \put ( 4.,1.){(a)}
    \put ( 12.4,1.){(b)}
}
\end{picture}
\vspace{-0.6cm}
 \caption{Positions as sampled in FLUKA with variations of the transverse beam size vertically (a) shown for full range until the arc and horizontally (b) shown for the first 85~m from the IP.
  \label{BGASflukaInp}}
\end{figure}



\section{Simulation Results for LHC's Run I and Run II\label{run1run2}}

\subsection{Run I and Run II simulation cases}
For Run~I and Run~II, we simulate the used $pp$ physics configurations of 2012 and 2015. At 4~TeV in 2012, the optics were for a $\beta^* = 60$ cm and TCT collimator settings in IR1/IR5 TCTs were set to 9~$\sigma$~\cite{parametersRun1}. For Run II, the energy was raised to 6.5~TeV and the optics changed to $\beta^* = 80$~cm which was used in the machine throughout 2015 for $pp$ collisions with IR1/IR5 TCTs were set to 13.7~$\sigma$. For both runs, a full vertical crossing angle of $290~\mu$m was taken into account in the simulations. More simulation and run parameters can be found in Tab.~\ref{paramsRun12}.

\begin{table}
   \centering
   \caption{Run I (2012)~\cite{bruce11evian} and Run II (2015)~\cite{bruce15_PRSTAB_betaStar} simulation parameters.}
   \begin{tabular}{l||c|c}
       \hline
       beam energy & 4 TeV & 6.5~TeV \\
       $\beta^*$ optics  & 60~cm &  80~cm \\
       bunch intensity & 1.4$\times 10^{11}$ protons &  1.12$\times 10^{11}$ protons\\
       number of bunches & 1380 & 2041\\
       bunch spacing & 50~ns & 25~ns\\
       half-crossing angle IP1~/~5 & 145~$\mu$rad & 145~$\mu$rad \\
       TCP.IR7~/~TCSG.IR7~/~TCT.IR1/5 & 4.3~/~6.3~/~9.0~$\sigma$ & 5.5~/~8.0~/~13.7~$\sigma$ \\
       TCP.IR3 & 12~$\sigma$ & 15~$\sigma$ \\
       \hline
   \end{tabular}
   \label{paramsRun12}
\end{table}

The two background sources as previously described were investigated for several run scenarios. During Run I, the LHC was operated with two beam energies, 3.5 TeV in 2010--11 and 4~TeV in 2012\footnote{A detailed overview of parameters in Run I can be found in~\cite{parametersRun1}.}. While simulations of machine-induced background at 3.5 TeV have been presented in~\cite{nimPaperRod}, we focus on new developments made since and use these improvements for 4 TeV and Run II simulation cases. %Beam-halo simulations consider now a crossing angle in the simulation as it has been present in the machine. For beam-gas simulations we consider now the transverse extension of the beam.

\subsection{Run I: 4 TeV betatron beam-halo}

Beam-halo induced showers from the tertiary collimators depend on the initial distribution of the positions at which the halo-proton interacts inelastically.
%For tungsten, the collimator material of the TCTs, one can expect that inelastic interactions close to the jaw surface produce more shower particles than those deeper inside when most of the inelastic processes is confined within the jaw thereby also the shower production.
The depth distribution for is shown for B1 and B2 in Fig.~\ref{inel4TeV} and one can see the hits are rather close to the jaw surface with most of the hits in the first few bins (note the logarithmic scale). Almost 7 million primary particles per beam were simulated in \fluka.

\begin{figure}[!htb]
\centering
\includegraphics[width=0.45\textwidth]{figures/inelposition_sum_impacts_real_NewScatt_4TeV_haloB1.pdf}
\includegraphics[width=0.45\textwidth]{figures/inelposition_sum_impacts_real_NewScatt_4TeV_haloB2.pdf}
 \caption{Positions of inelastic interactions of the two collimator jaws normalised to the total number of hits in the IR1 TCTs from h+v SixTrack simulations. The total number of lost protons on the TCPs is shown in Tab.~\ref{leakageFactorsIR7}.
  \label{inel4TeV}}
\end{figure}

As example we highlight some characteristic particle distributions of B1 halo induced showers at the interface plane, normalised per TCT hit, in Fig.~\ref{dist4TeVB1} and more can be found in the appendix~\ref{run1run2app}. The energy spectrum of the most important charged and neutral particle species is shown in Fig.~\ref{dist4TeVB1}. We note the peak at beam energy of the protons arising from single diffractive dissociations, emerging from such an interaction slightly off-energy. One can also see that in the 10~GeV to around 500 GeV regime muons are the most dominant particle species, at lower energies there are overwhelmingly many photons, followed by electrons. The azimuthal distribution of their energy as Fig.~\ref{dist4TeVB1} (right) is dominated by protons which is clear from the single diffractive part. We note, the energy of muons peak in the horizontal plane at $\phi \approx 0, \pm \pi$. 

\begin{figure}%[!htb]
\begin{center}
\includegraphics[width=0.49\textwidth]{figures/4TeV/haloB1_20MeV/Ekin_BH_4TeV_B1_20MeV.pdf}
\includegraphics[width=0.49\textwidth]{figures/4TeV/haloB1_20MeV/PhiEnDist_BH_4TeV_B1_20MeV.pdf}
\end{center}
\vspace{-0.6cm}
 \caption{Betatron halo induced background by beam 1 at the interface plane showing the energy spectrum (left) and its distribution in $\phi$ for different particle species.
  \label{dist4TeVB1}}
\end{figure}

Comparing B1 to B2 halo induced distributions, as shown in Fig.~\ref{fig:comp4TeVB1B2}, one can see that the shapes of the energy spectrum and the azimuthal energy distribution of all particles are quite similar. One can recognise that there can be more energy expected from B2 than from B1 per TCT hit due to the different input distributions.



\begin{figure}%[!htb]
\begin{center}
\includegraphics[width=0.49\textwidth]{figures/4TeV/compB1B2/perTCThit/ratioEkinAll.pdf}
\includegraphics[width=0.49\textwidth]{figures/4TeV/compB1B2/perTCThit/ratioPhiEnAll.pdf}
\end{center}
\vspace{-0.6cm}
 \caption{Comparison of B1 and B2 betatron halo induced shower distributions at the interface plane of all particles at 4~TeV, IR1. The numbers in the ratio plot is the ratio of both integrals of the top distributions. The error bars indicate statistical uncertainties.
  \label{fig:comp4TeVB1B2}}
\end{figure}


% --------------------------------------------------------------------------------------------
\subsection{Run I: 4 TeV Beam-Gas}

Characteristic distributions at the interface plane induced by beam-gas interactions are shown in Fig.~\ref{dist4TeVBGbs}. The energy spectra, Fig.~\ref{dist4TeVBGbs} (left), highlights a typcial shape for beam-gas interactions: in contrast to beam-halo showers it does not have the single diffractive peak at beam energy. Beam-gas induced protons show a much smoother increase up to beam-energy than in the halo case. It is still due to the same physics process, single diffractive dissociation, but further detailed studies are required to determine the exact reasons for the shape differences (e.g.~if the cause is in scattering partner material, tungsten or nitrogen, or how significant the contributions from other locations than the TCTs are).
One notices an immediate difference, when also looking at the right of that figure, on average a factor 10 or more particles and energy produced by a beam-gas interaction than by a TCT interaction. Nearly 6 million primary interactions were simulated in \fluka.%One can clearly see that comparing the energy of protons at small radii, essentially the first bin of Fig.~\ref{dist4TeVB1}~(d) and Fig.~\ref{dist4TeVBGbs}~(d).


\begin{figure}%[!htb]
\centering
\includegraphics[width=0.49\textwidth]{figures/4TeV/bs_20MeV/Ekin_BG_4TeV_20MeV_bs.pdf}
\includegraphics[width=0.49\textwidth]{figures/4TeV/bs_20MeV/PhiEnDist_BG_4TeV_20MeV_bs.pdf}
 \caption{Beam-gas induced background at the interface plane. The energy spectrum (left) of different particles and their energy in $\phi$ (right) is shown.
  \label{dist4TeVBGbs}}
\end{figure}
% --------------------------------------------------------------------------------------------
\subsection{Comparison to previous techniques}

We compare simulations performed for Run~I at 4 TeV and at 3.5~TeV (as has been presented in Ref.~\cite{nimPaperRod}). To ease the comparison of the new simulation techniques, we also show the ratio of the two compared distributions to overcome the difference due to the slightly different beam energy. In fact, at 3.5~TeV there were several optics and collimator settings deployed which enables us to conclude about the question if the TCT settings, which were 2.8~$\sigma$ different, will have any influence on beam-gas at the interface plane. 


\subsubsection{Transverse beam size}
To test the effect of the beam size, two cases were simulated in \fluka, with exactly the same setup for the 2012 Run I scenario, see Tab.~\ref{paramsRun12}, but using a different input file for positions at which beam-gas interactions are sampled. One file contains positions exactly on the reference orbit (``pointlike''), the other consideres transverse extensions of the real beam size. Direct comparisons are possible in this case and made in Fig.~\ref{bsRatioPhiAll}, highlighting the $\phi$ distribution of all particles and their energy. One can see there are small differences in the particle distribution (left) while the energy in $\phi$ (right) is smoother with the new technique. Especially the peaks at $\pm \frac{\pi}{2}$ in the energy plot show clearly there is a re-distribution to the ``shoulders''.

%As the energy in $\phi$ is almost shaped by protons, we show protons and muons as they are the ones reaching far into the detector region. Indeed, there is almost the same picture for the azimuthal proton distributions shown in Fig.\ref{bsRatioPhiMP}. The effect is more pronounced for high energy protons as well in the same figure. On muons also in the figure, the new technique does not seem to have a significant impact at all.

We investigate differences in the four s-sections from Fig.~\ref{twissfileBS} to see if there is a visible difference where one could expect the beam size to be rather large. This is compared in Fig.~\ref{bsZAll} for all particles where the multiplicity (top) and energy (bottom) is shown. One remarks a few places where one has clear differences like in the last s-section, 269 to 547~m, at $\phi \approx -1$, where the ratio goes down by $60~\%$. Numerically, it is a minor contribution with about 0.0033 particles per BG interaction compared e.g.~the first s-section from 22.6~m to 59~m where contributions range between 1.2 and 1.8 particles per beam-gas interaction. The statistical uncertainties are also largest in the last s-section meaning there are only a small contribution from that region. The corresponding plot in the bottom row reveals that at $\phi \approx -1$ the energy can be higher when the beam size is included by about 10~to 20~$\%$.
In the two middle s-sections, one can observe similar differences however the range is smaller with 10 to 15~$\%$ in the multiplicity and energy plot. The first section though contains the same dominating features seen in Fig.~\ref{bsRatioPhiAll}, which includes the integral over all s-sections. This ``smearing'' of the peaks at $\pm \frac{\pi}{2}$ is visible in both multiplicity and energy and makes up to 60~$\%$ difference.

Looking back at Fig.~\ref{bsRatioPhiAll} one can conclude that this new technique has a slight effect only up to $5~\%$ for the mulitplicity and up to 40~$\%$ in energy in the vertical plane. We discussed here only the differences in $\phi$, but we add other distributions also per particle types for further information in Fig.~\ref{bsRatioEkin}, Fig.~\ref{bsRatioRadN}, and Fig.~\ref{bsRatioRadEn}.

\begin{figure}%[!htb]
\begin{center}
  \includegraphics[width=0.49\textwidth]{figures/4TeV/beamsizeRatio/ratioPhiNAll.pdf}
  \includegraphics[width=0.49\textwidth]{figures/4TeV/beamsizeRatio/ratioPhiEnAll.pdf}
\end{center}
\vspace{-0.6cm}
 \caption{4 TeV beam-gas azimuthal distributions of all particles (right) and their energy (left).
  \label{bsRatioPhiAll}} 
\end{figure}


\begin{figure}%[!htb]
\begin{center}
  \includegraphics[width=0.24\textwidth]{figures/4TeV/beamsizeRatio/ratioPhiNAllZ1.pdf}
  \includegraphics[width=0.24\textwidth]{figures/4TeV/beamsizeRatio/ratioPhiNAllZ2.pdf}
  \includegraphics[width=0.24\textwidth]{figures/4TeV/beamsizeRatio/ratioPhiNAllZ3.pdf}
  \includegraphics[width=0.24\textwidth]{figures/4TeV/beamsizeRatio/ratioPhiNAllZ4.pdf}
  \includegraphics[width=0.24\textwidth]{figures/4TeV/beamsizeRatio/ratioPhiEnAllZ1.pdf}
  \includegraphics[width=0.24\textwidth]{figures/4TeV/beamsizeRatio/ratioPhiEnAllZ2.pdf}
  \includegraphics[width=0.24\textwidth]{figures/4TeV/beamsizeRatio/ratioPhiEnAllZ3.pdf}
  \includegraphics[width=0.24\textwidth]{figures/4TeV/beamsizeRatio/ratioPhiEnAllZ4.pdf}
\end{center}
\vspace{-0.6cm}
 \caption{Direct comparisons of 4~TeV beam-gas azimuthal distributions of all particles (top) and their energy (bottom) in different s-sections as in Fig.~\ref{twissfileBS}.
  \label{bsZAll}}
\end{figure}

 
\subsubsection{Effect of the crossing angle}

We can study the crossing angle effect on background in IR1 using the 3.5~TeV simulation data for $\beta^* = 1.0~$m optics and TCTs set to $11.8~\sigma$~\cite{nimPaperRod} thus betatron halo and beam-gas induced, without any crossing angle included in the simulations. We compare the 3.5~TeV data to 4~TeV halo data with a crossing angle of 290~$\mu$rad. For this comparison, we show the distributions at the interface plane per TCT hit. 

We find that the beam-gas data as shown in Fig.~\ref{xingCompBG} exhibit a clear signature of the improved simulation technique when including the crossing angle. There is a general accumulation in the upper part of the vertical plane; the incoming beam is directed downwards but as the sampling position is still at positive $y$, the peak of particles is around $+ \frac{\pi}{2}$. Muon distributions are not significantly affected as other effects are more dominating the azimuthal distribution (like magnetic fields effects). 

In contrast, the beam-halo data in Fig.~\ref{xingCompBHB1} do not have any clear signs of crossing angle effect when comparing 3.5~TeV\footnote{TCTs were at $15~\sigma$ for $\beta^*$ optics of 3.5~m.} to 4 TeV simulations in both beams. It is likely that secondary particles with all kind of forward directions are created at the TCTs losing the initial direction of the crossing angle of the beam protons.

\begin{figure}
\begin{center}
  \includegraphics[width=0.41\textwidth]{figures/4TeV/compBG_3p5_vs_4TeV/ratioPhiEnAll.pdf}
  \includegraphics[width=0.41\textwidth]{figures/4TeV/compBG_3p5_vs_4TeV/ratioPhiEnPhotons.pdf}
  \includegraphics[width=0.41\textwidth]{figures/4TeV/compBG_3p5_vs_4TeV/ratioPhiEnMuons.pdf}
  \includegraphics[width=0.41\textwidth]{figures/4TeV/compBG_3p5_vs_4TeV/ratioPhiEnMuE100.pdf}
\end{center}
\vspace{-0.6cm}
 \caption{Run I beam-gas data: A clear effect of crossing angle is visible (top) showing accumulations of particles at around $\frac{\pi}{2}$ in the vertical plane. The distribution of muons and in particular high energy muons are not significantly affected (bottom). This is the same if the data with beamsize is used, see Fig.~\ref{xingCompBG2}.
  \label{xingCompBG}}
\end{figure}


\begin{figure}
\centering
    \includegraphics[width=0.41\textwidth]{figures/compBHB1_3p5vs4TeV/ratioPhiEnAll.pdf}
    \includegraphics[width=0.41\textwidth]{figures/compBHB1_3p5vs4TeV/ratioPhiEnPhotons.pdf}
    \caption{B1 betatron halo data without significant sign of a crossing angle effect, in particular no enhancement around $\pi/2$. B2 is very similar, see Fig.~\ref{xingCompBHB2}.
      \label{xingCompBHB1}
      }
\end{figure}

\subsubsection{Effect of TCT settings on beam-gas}

Since the TCTs were set to 11.8~$\sigma$ to accommodate the $\beta^* = 1.0~$m optics at 3.5~TeV and 9~$\sigma$ at 4~TeV for $\beta^* = 60~$cm optics, the comparison of contributions from upstream of the TCTs (147~m and further away from the IP) can give indications about the influence of TCT settings on beam-gas, Fig.~\ref{compBGrun1} shows there is no significant influence, focussing on muons in the closest s-section to the TCTs and a transverse radius between 1~m and 2~m.

\begin{figure}
  \centering
  \includegraphics[width=0.41\textwidth]{figures/4TeV/compBG_3p5_vs_4TeV/perBGint_bs/ratioPhiEnMuonsZ3R1.pdf}
  \includegraphics[width=0.41\textwidth]{figures/4TeV/compBG_3p5_vs_4TeV/perBGint_bs/ratioPhiEnMuonsZ3R2.pdf}
  \caption{Influence of TCT settings in beam-gas simulations in the s-section closest to the TCTs: no clear difference visible in the azimuthal energy distributions of muons.
  \label{compBGrun1}}
\end{figure}

\subsection{Run I: Renormalisation of beam-gas events with pressure profile \label{BGreweighted4TeV}}

\begin{table}
   \centering
   \caption{LHC fill 2736 Run I (2012)~\cite{refAccStats}}
   \begin{tabular}{l||c}
       \hline
       beam energy  & 4~TeV \\
       Fill start time (local Geneva time) & 16/06/2012 18:22\\
       Stable beam start time (local Geneva time) & 16/06/2012 20:10\\
       Stable beam duration [hh:mm] & 17:29\\
       intensity ring 1& 2.029$\times 10^{14}$ protons\\
       intensity ring 2& 1.995$\times 10^{14}$ protons\\
       number of bunches & 1374 \\
       \hline
   \end{tabular}
   \label{tab:fillRunI}
\end{table}
In order to calculate rates of particles at the interface plane, we use the estimated pressure along the insertion to reweight and normalise the simulations with beam size.
%We use the simulations with beam size to reweight to a pressure map.
The information of the original inelastic proton interaction of each particle at the interface plane was kept in order to attribute weights to each simulated beam-gas event according to a typical pressure map. The simulations of the gas densities are based on gauge data recorded in LHC fill 2736, judged to be a representative fill of vacuum during $pp$ physics in 2012, and were performed using VASCO (VAcuum Stability COde)~\cite{vascoRef}. The pressures were simulated up to the end of the long straight section of IR1 (LSS1) for the most dominant molecules, H$_2$, CH$_4$, CO, CO$_2$. Partial pressure equivalents expressed in molecular densities are shown in the top of Fig.~\ref{pressure2012} and are decomposed into atomic components at the bottom of that figure to show the interaction probability as function of the s-location using the inelastic cross-sections for H, C and O as indicated in Tab.~\ref{tab:atomicXsections} for a 4 TeV beam. Beyond LSS1, where the dispersion suppression region starts, the last value is assumed as constant up to 547~m.

As described in the previous section, the weights are calculated using Eq.~\ref{eq3}. The total number of protons is given by the maximum beam intensity of LHC fill 2736 which was about $2.0 \times 10^{14}$, see Tab.~\ref{tab:fillRunI} for more fill details. The method is shown in Fig.~\ref{fig:method}: in green the total interaction probability rate, in black the number of muons per beam-gas interaction and in red the multiplication of both. One can see, the muon rates along $s$ are mainly shaped by the interaction probability which is based on the pressure map. Only beyond s~=~270~m the shape comes from the shower simulations as the pressure was assumed to be constant in that region. There are three locations that contribute most to background with highest rates of more than $10^4$~particles/s at around 270~m, where transition region of the last quadrupole of the matching section in the LSS, Q7\footnote{Q7 is operated at 1.9~K like the triplet quadrupoles Q1, Q2, Q3 and all quadrupoles in the dispersion suppression and arc~\cite{LHCDesignRep}}, is located. Comparable are rates of about $10^4$~particles/s are produced by pressure spikes in the triplet and at $s \approx 147~$m, the location of the TCTs.

We compare in detail how the distributions change when normalising to a pressure that corresponds to a flat to a non-flat profile. We highlight in Fig.~\ref{fig:OrigZ4TeV} the origin of creation of all particles as function of $s$. To compare better, the distribution corresponding to a flat profile are scaled up by seven orders of magnitude to match the scale of the rate. The shape for all particles is mainly driven by the pressure in LSS1, one can recognise the same three peaks as discussed earlier (triplet, TCTs and Q7), however for all particles they seem slightly broader. Single particle species are shown in the appendix for photons and protons in Fig.~\ref{fig:OrigZ4TeV2}. 

We observe how the shapes of the integrated distributions at the interface plane change focussing here on muons. Generally, the shape of the energy spectrum in Fig.~\ref{fig:cv81EkinPhiEn4TeV} (top left) before and after reweighting produce high similarities. In the medium energy range of about 20~MeV to 10~GeV there seems a small overestimation of particles from the flat profile if the up-scale of $10^{7}$ is considered. From 10~GeV to about 1~TeV, there is an underestimation of muons from the flat profile, before they become identical. From the azimuthal distribution of the energy (top right) in Fig.~\ref{fig:cv81EkinPhiEn4TeV} one can see the additional energy shows up on the upper and lower hemisphere. While the bottom left plot show almost identical shapes, the bottom right plot exhibits similarities up to $r \approx 150~$cm from where they become clearly different when the reweighted curves become slightly shallower. Similar conclusions can be derived for all particles shown in the appendix in Fig.~\ref{fig:cv81EkinPhiEn4TeV2}. Differences for protons and neutrons before and after reweightening is shown in Fig.~\ref{fig:cv81ProtNeut4TeV}.

\begin{figure}%[!htb]
\begin{center}
  \includegraphics[width=0.75\textwidth]{figures/4TeV/LSS1_B1_Fill2736_Finala_pressure.pdf}
  \includegraphics[width=0.75\textwidth]{figures/4TeV/reweighted/cv65_pint.pdf}
\end{center}
\vspace{-0.6cm}
 \caption{Gas densities for the 2012 4 TeV scenario in LSS1 shown for the most common molecules (top) and split into atomic components to indicate the interaction probability (bottom). The incoming beam is from left to right, i.e.~only the part at negative values up to a distance of -22.6~m is used. Pressure rises usually at transition regions of different temperatures (triplet quadrupoles Q1~--~Q3 and those of the matching section Q4~--~Q7) and at locations with materials of different thermal outgassing, e.g.~the TCT region at -147~m.
  \label{pressure2012}}
\end{figure}

\begin{figure}%[!htb]
\begin{center}
  \includegraphics[width=0.95\textwidth]{figures/4TeV/reweighted/muons2012.pdf}
\end{center}
\vspace{-0.6cm}
 \caption{Origin of muon rates production normalised to the pressure for the 4 TeV beam-gas scenario as in Fig.~\ref{pressure2012}. See text for description.
  \label{fig:method}}
\end{figure}


\begin{figure}
\begin{center}
  \includegraphics[width=0.75\textwidth]{figures/4TeV/reweighted/cv81_OrigZAll_BG_4TeV_20MeV_bs}
%  \includegraphics[width=0.75\textwidth]{figures/4TeV/reweighted/cv81_OrigZMuon_BG_4TeV_20MeV_bs}
%  \includegraphics[width=0.75\textwidth]{figures/4TeV/reweighted/cv81_OrigZProtons_BG_4TeV_20MeV_bs}
\end{center}
\vspace{-0.6cm}
 \caption{Origin of all particles production along $s$, in black before and in pink after re-normalising to the 2012 pressure profile. The rates are dominated by photons created on the very last meters before the interface plane (also the spike at $s=520~$m is due to local photon production, see middle plot in Fig.~\ref{fig:OrigZ4TeV2}).
  \label{fig:OrigZ4TeV}}
\end{figure}

\begin{figure}
\begin{center}
%% %%  \includegraphics[width=0.45\textwidth]{figures/4TeV/reweighted/cv81_EkinAll_BG_4TeV_20MeV_bs}
%%   \includegraphics[width=0.45\textwidth]{figures/4TeV/reweighted/cv81_EkinMuons_BG_4TeV_20MeV_bs}
%% %%  \includegraphics[width=0.45\textwidth]{figures/4TeV/reweighted/cv81_PhiEnAll_BG_4TeV_20MeV_bs}
%%   \includegraphics[width=0.45\textwidth]{figures/4TeV/reweighted/cv81_PhiEnMuons_BG_4TeV_20MeV_bs}
%%   %% \includegraphics[width=0.45\textwidth]{figures/4TeV/reweighted/cv81_RadNAll_BG_4TeV_20MeV_bs}
%%    \includegraphics[width=0.45\textwidth]{figures/4TeV/reweighted/cv81_RadNMuons_BG_4TeV_20MeV_bs}
%%   %% \includegraphics[width=0.45\textwidth]{figures/4TeV/reweighted/cv81_RadEnAll_BG_4TeV_20MeV_bs}
%%    \includegraphics[width=0.45\textwidth]{figures/4TeV/reweighted/cv81_RadEnMuons_BG_4TeV_20MeV_bs}
  \includegraphics[width=0.45\textwidth]{figures/4TeV/compBGflat/ratioEkinMuons.pdf}
  \includegraphics[width=0.45\textwidth]{figures/4TeV/compBGflat/ratioPhiEnMuons.pdf}
  \includegraphics[width=0.45\textwidth]{figures/4TeV/compBGflat/ratioRadNMuons.pdf}
  \includegraphics[width=0.45\textwidth]{figures/4TeV/compBGflat/ratioRadEnMuons.pdf}
\end{center}
\vspace{-0.6cm}
 \caption{Energy spectrum (top left), energy in $\phi$ (top right), transverse radius $r$ (bottom left) and energy in $r$ (bottom right) of muons before (black) and after (light color) reweightening to the pressure profile. The black curves are scaled up by $10^7$ to better compare the shapes. %See Fig.~\ref{fig:cv16EkinPhiEn4TeV} for more details.
  \label{fig:cv81EkinPhiEn4TeV}} 
\end{figure}

\clearpage
\newpage
% --------------------------------------------------------------------------------------------
\subsection{Run I: Off-momentum halo and shower simulations at 4 TeV}

Analogue to IR7 leakage to the experimental IRs of ATLAS and CMS, one can now investigate background contributions from IR3 off-momentum cleaning using the losses leaking to IR1 and IR5 TCTs, also expressed in TCP-to-TCT conversion factors. They are listed in Tab.~\ref{tab:IR3leakageFactors} for IR1 and IR5 for all the simulated cases.

The off-momentum leakage in IR1 is almost entirely caused by B2 and in IR5 it originates from B1. This can be expected from the ring geometry considering the relative positions of IR3 and IR7 for both beams respectively: Almost all halo protons in B1 moving from IR3 towards IR1 are cleaned by IR7 on the way, as well as B2 protons moving towards IR5.

We also note something interesting from Tab.~\ref{tab:IR3leakageFactors} when comparing the case ``+500~Hz'' (plus-case) with ``-500~Hz'' (minus-case): At 4~TeV, the negative fequency shift causes a leakage to IR1 of $0.012$, in the ``plus-case'' it is nearly precisely the half of it, $0.0063$. The same is true for IR5 B1: The losses are reduced from $0.045$ to $0.025$ from positive to negative frequency shift. On absolute terms these numbers appear high compared to leakage losses from IR7, e.g.~at the same Run~I scenario in Tab.~\ref{leakageFactorsIR7}. However, the primary losses in IR3 are usually much lower than in IR7.

We depict in Fig.~\ref{inel4TeVOffmom} how deeply the inelastic interactions take place within the collimator jaws. These distributions show a clear difference to those leaking from IR7: First, we note the full contribution for IR1 comes from B2, there is only a negligable fraction of B1 protons hitting the TCTs. Vice versa for IR5, almost nothing ends up on the TCTs in IR5 from B2 and contributions from IR3 leakage is due to B1 only. Then we notice, the distributions go much further into the jaw compared to e.g.~Fig.~\ref{inel4TeV} for both beams in the ``minus-case'' but also for B1 in the ``plus-case''. B2 ``plus-case'' has the intercepted protons more on the surface, especially those in the TCTV go in to around 4~mm which is more similar to the betatron halo case at 4~TeV. 

We use the B2 TCT starting conditions in the shower simulations for IR1. General distributions for the main particle species are shown for a negative shift in Fig.~\ref{offmom4TeV}. They look rather similar to betatron halo induced showers. More characteristic distributions of the ``plus-case'' and a detailed comparison of the two cases are shown in the appendix in Fig.~\ref{offmom4TeV2}, Fig.~\ref{compPM_ekin} and Fig.~\ref{compPM_phien}.

As Fig.~\ref{compPM} shows, one can expect about 20~$\%$ more particles carrying about 30$~\%$ more energy when particles experience a negative than a positive shift per TCT interaction. This can be understood if one compares the first bins of the right side of Fig.~\ref{inel4TeVOffmom}. The distributions in the TCTV looks rather similar, however one can see many more interactions take place at the surface of the TCTH.4R1 in the ``plus-case'' than in the ``minus-case''. More comparisons are in the appendix in Fig.~\ref{compPM_ekin} and Fig.~\ref{compPM_phien}.


\begin{figure}
\begin{center}
\includegraphics[width=0.49\textwidth]{figures/inelposition_sum_impacts_real_4TeV_plus500Hz_TCT_B1.pdf}
\includegraphics[width=0.49\textwidth]{figures/inelposition_sum_impacts_real_4TeV_plus500Hz_TCT_B2.pdf}
\includegraphics[width=0.49\textwidth]{figures/inelposition_sum_impacts_real_minus500Hz_4TeVB1.pdf}
\includegraphics[width=0.49\textwidth]{figures/inelposition_sum_impacts_real_minus500Hz_4TeVB2.pdf}
\end{center}
\vspace{-0.6cm}
 \caption{4 TeV off-momentum halo: Positions of inelastic interactions as given by SixTrack within the IR1 and IR5 collimator jaws for a postive (top) and negative (bottom) frequency shift for B1 (left) and B2 (right).
  \label{inel4TeVOffmom}}
\end{figure}


\begin{figure}
\begin{center}
  \includegraphics[width=0.49\textwidth]{figures/4TeV/offmom/20MeV/Ekin_offmin500Hz_4TeV_B2_20MeV.pdf}
  \includegraphics[width=0.49\textwidth]{figures/4TeV/offmom/20MeV/PhiEnDist_offmin500Hz_4TeV_B2_20MeV.pdf}
\end{center}
\vspace{-0.6cm}
 \caption{B2 off-momentum halo induced particle distributions in IR1 at the interface plane for the 4~TeV scenario in Run~I.
  \label{offmom4TeV}}
\end{figure}

\begin{figure}
  \centering
  \includegraphics[width=0.49\textwidth]{figures/4TeV/offmom/comppm500Hz/ratioEkinAll.pdf}
  \includegraphics[width=0.49\textwidth]{figures/4TeV/offmom/comppm500Hz/ratioPhiEnAll.pdf}
  \caption{B2 off-momentum halo in IR1 at 4~TeV: Direct comparison the positive and negative frequency shifts induced distributions showing the energy spectrum of all particles (left) and azimuthal distribution of energy (right).
    \label{compPM}}
\end{figure}

% --------------------------------------------------------------------------------------------
\subsection{Run II: 6.5 TeV Betatron Halo}

We move to the Run~II 2015 scenario with $\beta^* = 80$~cm, and details are in Tab.~\ref{paramsRun12}, using the same method as described for Run~I at 4~TeV. The input distributions are shown for B1 and B2 in Fig.~\ref{inel6.5}. It can be seen that the B1 depth distribution in the TCTV is much shallower than at 4~TeV in Fig.~\ref{inel4TeV}. This is very likely due to the different optics and collimator settings with the TCTs being at almost 5~$\sigma$ further out. Characteristic distributions are shown in Fig.~\ref{dist6500GeVB2}, and a comparison to B1 distributions are in Fig.~\ref{compBHB1B2run2}. One can see that in contrast to 4 TeV data, B1 produces more shower particles and energy than B2 per inelastic interaction within the TCTs. 


\begin{figure}[!htb]
\begin{center}
\includegraphics[width=0.49\textwidth]{figures/inelposition_sum_HALOB1.pdf}
\includegraphics[width=0.49\textwidth]{figures/inelposition_sum_HALOB2.pdf}
\end{center}
 \caption{Positions of inelastic interactions in IR1 TCTs from betratron cleaning simulations with SixTrack for B1 (left) and B2 (right) at 6.5 TeV.
  \label{inel6.5}}
\end{figure}


\begin{figure}%[!htb]
\centering
\includegraphics[width=0.49\textwidth]{figures/BH_run2/b2/Ekin_BH_6500GeV_haloB2_20MeV.pdf}
\includegraphics[width=0.49\textwidth]{figures/BH_run2/b2/PhiEnDist_BH_6500GeV_haloB2_20MeV.pdf}
 \caption{B2 betatron halo induced background at the interface plane at 6.5~TeV. The distributions exhibit similar features as at 4 TeV (Fig.~\ref{dist4TeVB1}).
  \label{dist6500GeVB2}}
\end{figure}


\begin{figure}%[!htb]
\centering
  \includegraphics[width=0.49\textwidth]{figures/BH_run2/perTCThit/ratioEkinAll.pdf}
  \includegraphics[width=0.49\textwidth]{figures/BH_run2/perTCThit/ratioPhiEnAll.pdf}
 \caption{Run~II 2015 case: Comparison of B1/B2 betatron halo induced distributions per TCT hit. Muons are shown in the appendix in Fig.~\ref{compBHB1B2run22}.
  \label{compBHB1B2run2}}
\end{figure}

\subsubsection{Run II: Comparison of 2015 and 2016 IR7 leakages}

We include here a discussion on IR7 betatron leakages with a scenario as used in 2016 at 6.5~TeV, where the normal physics optics had a $\beta^*$ of 40~cm and the TCTs in IR1 and IR5 were set to 9~$\sigma$~\cite{bruceEvian2015}, see also Tab.~\ref{2016leakageFactorsIR7}. No \fluka~shower simulations were performed for this scenario. When comparing IR7 leakages to e.g~to IR1 of B1 and B2 of other scenarios in Run~I, Run~II (2015) and HL-LHC as listed in Tab.~\ref{tab:leakageFactorsIR7}, we note that the Run~II 2016 leakages have increased to $10^{-4}$, the same order as in HL-LHC. In particular, B1 gives higher leakages to IR1 with $4.2 \times 10^{-4}$ than B2 and $1.2 \times 10^{-4}$, which is the only case in all other cases studied.



\subsection{Run II: 6.5 TeV Beam-Gas}

This case has been simulated with the same method as for the Run~I scenario, accounting for the beam size. The beam optics and collimator settings are shown in Tab.~\ref{paramsRun12}. Characteristic distributions, the energy spectrum and the azimuthal distribution per particle type, are shown in Fig.~\ref{bg6500}. One can see, the obtained distributions at 6.5~TeV are qualitatively very similar to the ones at 4~TeV. More distributions are in the appendix in Fig.~\ref{bg65002}. 

\begin{figure}%[!htb]
\begin{center}
  \includegraphics[width=0.49\textwidth]{figures/6500GeV/20MeV/Ekin_BG_6500GeV_flat_20MeV_bs.pdf}
  \includegraphics[width=0.49\textwidth]{figures/6500GeV/20MeV/PhiEnDist_BG_6500GeV_flat_20MeV_bs.pdf}
%  \includegraphics[width=0.49\textwidth]{figures/6500GeV/20MeV/RadNDist_BG_6500GeV_flat_20MeV_bs.pdf}
%  \includegraphics[width=0.49\textwidth]{figures/6500GeV/20MeV/RadEnDist_BG_6500GeV_flat_20MeV_bs.pdf}
\end{center}
\vspace{-0.6cm}
 \caption{Characteristic beam-gas induced distributions at 6.5~TeV per BG interaction using the more realistic model of the beam size.
  \label{bg6500}}
\end{figure}

\subsection{Run II: Renormalisation of beam gas events with pressure profile}

Similar to Run I in 2012, a representative fill (number 4536) was chosen and used to simulate the pressure map. Fill details are summed up in Tab.~\ref{tab:fillRunII}. We use this fill to normalise the simulated particle distributions at the interface plane to absolute rates. 

\begin{table}
   \centering
   \caption{LHC fill 4536 Run II (2015)~\cite{refAccStats}}
   \begin{tabular}{l||c}
       \hline
       beam energy  & 6.5~TeV \\
       Fill start time (local Geneva time) & 26/10/2015 02:59\\
       Stable beam start time (local Geneva time) & 26/10/2015 10:26\\
       Stable beam duration [hh:mm] & 01:04\\
       intensity ring 1& 2.2899$\times 10^{14}$ protons\\
       intensity ring 2& 2.2137$\times 10^{14}$ protons\\
       number of bunches & 2041 \\
       \hline
   \end{tabular}
   \label{tab:fillRunII}
\end{table}

\paragraph{\textit{Pressure Map Simulations}}
In previous years (in particular for 3.5~TeV and 4~TeV runs data), as mentioned earlier in Sec.~\ref{BGreweighted4TeV} the code to simulate the pressure map, \textsc{VASCO}~\cite{vascoRef}, was used. For Run II in 2015, pressures were simulated with an improved version of the code. The code has been re-implemented in \textsc{Python} and modifications were made to automatise pressure simulations~\cite{christinasStudent}. The describtion of photon desorption, as determined by SynRad~\cite{synradRef}, and electron cloud effects as determined in PyECLOUD~\cite{giovanniPhd} were included and therefore the photon and electron flux simulations should be more accurate. Original algorithms were kept however a few assumptions were made to be able to simulate pressures around the entire ring, in particular where no vacuum instrumentation is placed (e.g.~in the arc). The variety of materials and pumping speeds was simplified. The pressure for fill 4536 shown in Fig.~\ref{pressure2015} and is probably underestimated due to the simplifications made and represents thus a scenario on the optimistic side. 

For comparison we added the total interaction rate based on the pressure map at 4~TeV. One can see that the beam-gas interaction probability per proton is 1--2 orders of magnitude lower in 2015 than in 2012. This is expected as during the first long shutdown of the LHC from 2013 into 2015, besides consolidation activities~\cite{KatyForazIpac14}, many hardware improvements for the vacuum were made. These were for vacuum in the long straight sections of the experimental insertions (between left Q4 to right Q4), where NEG (Non-Evapourable-Gatter) cartridges with additional 400l/s for H$_2$ pumping speed were installed at the same places as the pressure gauges. We also note that the peak at the TCT region that was present in 4~TeV's fill has vanished. This is very likely due to the simplification of materials made for the 2015 pressure map simulation.


\paragraph{\textit{Reweighted Results using 2015 pressure}}

The pressure simulation data based on Fill 4536 is shown in top of Fig.~\ref{pressure2015} for each gas type, while the bottom figure shows the interaction probability per second as in Eq.~\ref{eq2} and what is used to reweight the distributions at the interface plane as decribed in Sec.~\ref{BGdescript}. As beam intensity we considered what was present in ring 1, see Tab.~\ref{tab:fillRunII}.

We use muons to illustrate how the distributions change before and after reweighting. The muon production as function of $s$ and general distributions of muons at the interface plane are highlighted in Fig.~\ref{fig:OrigZ6p5} and Fig.~\ref{fig:EkinPhiEn6p5}, respectively. We note, the profile of the muon rates in Fig.~\ref{fig:OrigZ6p5} is very close to the pressure map that was used in particular in the arc. After reweightening, unlike at 4~TeV, the shapes of the distributions change significantly when comparing the up-scaled curve by $10^7$. The energy spectrum, visible in the top left of Fig.~\ref{fig:EkinPhiEn6p5}, one can see only up to 200~MeV the curves are similar. The azimuthal energy distribution on the right side of that figure becomes after reweightening remarkably periodic with peaks in the horizontal plane at $\pm \pi$ and $0$ and in the vertical plane at $\pm \pi/2$. The shape of the transverse radii of muons is only similar at the lower part (between 50~cm and 150~cm), in the left bottom figure, and for higher radii, from $r =$~180~cm on for transverse radii of energy distributions.

\begin{figure}
\begin{center}
  \includegraphics[width=0.95\textwidth]{figures/6500GeV/reweighted/Density_Fill4536_2041b_26158_B1_withECLOUD_rho.pdf}
  \includegraphics[width=0.95\textwidth]{figures/6500GeV/reweighted/compallpint.pdf}
\end{center}
\vspace{-0.6cm}
 \caption{Simulated gas densities in IR1 in 2015 shown for the most common molecules (top). The beam direction (top) is from left to right, and we use the left side of the incoming beam only to derive the total interaction probability for the 6.5~TeV fill (bottom), where the sign is inverted. The data for the 4~TeV fill, 2736, is also shown (but without constant extension out to the arc).
  \label{pressure2015}}
\end{figure}

\begin{figure}
\begin{center}
%  \includegraphics[width=0.75\textwidth]{figures/6500GeV/reweighted/cv81_OrigZAll_BG_6500GeV_flat_20MeV_bs.pdf}
  \includegraphics[width=0.75\textwidth]{figures/6500GeV/reweighted/cv81_OrigZMuon_BG_6500GeV_flat_20MeV_bs.pdf}
%  \includegraphics[width=0.75\textwidth]{figures/6500GeV/reweighted/cv81_OrigZProtons_BG_6500GeV_flat_20MeV_bs.pdf}
\end{center}
\vspace{-0.6cm}
 \caption{Origin of muons produced along $s$, in black before and in gold after re-normalising to the 2015 pressure profile. All particles, protons and photons are shown in the appendix in Fig.~\ref{fig:OrigZ6p52}. 
  \label{fig:OrigZ6p5}}
\end{figure}

\begin{figure}
\begin{center}

%  \includegraphics[width=0.45\textwidth]{figures/6500GeV/reweighted/cv81_EkinMuons_BG_6500GeV_flat_20MeV_bs}
%  \includegraphics[width=0.45\textwidth]{figures/6500GeV/reweighted/cv81_PhiEnMuons_BG_6500GeV_flat_20MeV_bs}
%   \includegraphics[width=0.45\textwidth]{figures/6500GeV/reweighted/cv81_RadNMuons_BG_6500GeV_flat_20MeV_bs}
  %   \includegraphics[width=0.45\textwidth]{figures/6500GeV/reweighted/cv81_RadEnMuons_BG_6500GeV_flat_20MeV_bs}
  \includegraphics[width=0.45\textwidth]{figures/6500GeV/compBGflat/ratioEkinMuons.pdf}
  \includegraphics[width=0.45\textwidth]{figures/6500GeV/compBGflat/ratioPhiEnMuons.pdf}
  \includegraphics[width=0.45\textwidth]{figures/6500GeV/compBGflat/ratioRadNMuons.pdf}
  \includegraphics[width=0.45\textwidth]{figures/6500GeV/compBGflat/ratioRadEnMuons.pdf}
\end{center}
\vspace{-0.6cm}
 \caption{Muon distributions before (black) and after (yellow) reweightening with the pressure map. The same set of plots for all particles are in the appendix, in Fig.~\ref{fig:EkinPhiEn6p52}.
  \label{fig:EkinPhiEn6p5}}
\end{figure}

\section{Beam-Halo Simulation Results for HL--LHC\label{hllhcResults}}

%Several cases were simulated in order to characterise the cleaning efficiency for baseline settings of HL-LHC, deploying different collimator layouts (in IR1/5 TCT4s only and TCT4s + TCT5s) and alternative collimator settings, so called \twosigmaret~settings to quantify the effect of the TCT5s for incoming beams (B1 and B2). Inelastic interactions with beam protons are forced in FLUKA at initial conditions given by SixTrack on the TCT4s and (when included) TCT5s. These interactions generate a particle flux towards the experiment. All shower particles are recorded at the machine-detector interface plane at 22.6~m from the IP using in FLUKA a production and transportation cut-off at 20 MeV.

This section describes the several case studies for HL, focussing on simulation results made since~\cite{kweeIpac14} for beam-halo induced background. An overview is given in Tab.~\ref{hlscenario} with two options of collimator layout (TCT5s in/out), two sets of collimator openings (nominal/\twosigmaret) and two beam optics (round/flat). Beam-gas simulations in an HL scenario were described in Ref.~\cite{kweeIpac14} remain sofar the most recent available simulations of that kind. Once, a new pressure map will be available for a 7~TeV beam in the HL scenario and major updates of the geometry are included, we recommend an update of the simulations. 

\subsection{Comparison of background with different collimator settings}

The aim was to study beam losses for different collimator layouts and settings to evaluate quantitatively on halo background in IR1. Settings of the collimation system were revised as described in detail in Ref.~\cite{collSettRef} and an updated version of collimator settings, the so-called \twosigmaret~settings, were inferred. As one can see from Tab.~\ref{hlscenario}, the IR7 secondary collimators are retracted w.r.t.~the primary collimators by 2$\sigma$, comparing to the nominal openings of the primary collimators they are even tighter. Having such a tight setting up in the hierarchy allows to relax the settings of the tertiary collimators in the IRs. It is anticipated that the new settings, \twosigmaret~settings will already be favorable for background.

We analysed what improvement can be expected using the final collimator layout (TCT5s included) and baseline \textit{achromatic squeeze} (ATS) optics~\cite{ATSref} (round beam). Comparing the produced particle distributions at the interface plane, we find that per TCT hit slightly more background is produced with the \twosigmaret~settings. The distributions in Fig.~\ref{fig:compNomRetrSett} show there number of particles (left) and the energy (right) will slightly increase with more open TCTs at IR1, about 50~\% more particles which produce 40~\% more energy will reach the interface plane. The reason for such an increase is if the TCTs are more open, more halo protons interact on the jaw surface and showers are less comprised in the jaws. Showers can develop better outside the jaws and hence produce more secondaries at the interface plane. In absolute terms, one has to consider the efficiency of the collimation system on top of it for which we compute the IR7 leakage towards the experimental IRs. This is listed in Tab.~\ref{leakageFactorsIR7}. The leakage to IR1 is in the case of nominal settings, $4.7 \times 10^{-4}$, triple as big as in the \twosigmaret~settings, being then only $1.2 \times 10^{-4}$. Already deploying the new settings will eventually lead to less leakage and thereby less halo-induced background in IR1, only 40~\% of the showers remain. Similar gain for the reduction of leakage is expected in IR5, with leakage factors of about 2.8~times smaller.



\begin{table}%[hbt]
   \centering
   \caption{Overview of beam-halo simulation cases performed using SixTrack and optics version HLLHC-v1.0.}\vskip2mm
   \begin{tabular}{|l|l|l|l|}
       \hline
       collimator settings & TCT5s & beam halo & optics \\
       \hline\hline
       nominal  & out & h+v B1 & round \\
       nominal  & in & h+v B1 & round \\\hline
       \twosigmaret & out & h+v B1 & round \\ 
       \twosigmaret & in  & h+v B1 & round \\ 
       \twosigmaret & out & h+v B2 & round \\
       \twosigmaret & in  & h+v B2 & round \\ \hline
       \twosigmaret & out  & h+v B1 & flat \\
       \twosigmaret & in  & h+v B1 & flat \\ 
%       \twosigmaret & in  & h+v B1 & sround \\ 
%       \twosigmaret & in  & h+v B1 & sflat \\ 

       \hline

   \end{tabular}
   \label{hlscenario}
\end{table}

%% \begin{figure}%[!htb]
%%   \centering
%%   \includegraphics[width=0.495\textwidth]{figures/inelposition_sum_impacts_real_HL_TCT5IN_nomColl_haloB1}
%%   %\includegraphics[width=0.495\textwidth]{figures/inelposition_sum_tct5inrd}
%%   \caption{Positions from the jaw surface into the jaw material for nominal collimator settings for beam 1.
%%   \label{fig:inel_nomColl}}
%% \end{figure}

\begin{figure}
\begin{center}
\includegraphics[width=0.43\textwidth]{figures/HL/compNomRetrCollSett/perTCThit/ratioEkinAll.pdf}
\includegraphics[width=0.43\textwidth]{figures/HL/compNomRetrCollSett/perTCThit/ratioPhiEnAll.pdf}
\end{center}
\vspace{-0.6cm}
 \caption{Comparison of halo-induced background showers with nominal and \twosigmaret~collimator settings .
  \label{fig:compNomRetrSett}}
\end{figure}


% ------------------------------------------------------------------------------------------

\subsection{Effect of TCT5s on halo background in IR1}

Each IR has a tertiary collimator pair, a horizontal and vertical TCT, in cell 4 on the side of the incoming beam with a protective role. Firstly they provide local protection against regular losses but also in case of abnormal losses as asynchronous beam dump. As in HL-LHC the triplet will be exchanged with one that has much larger apertures for higher $\beta^*$ reach, possible bottlenecks emerge at other locations. Triplet apertures become similar with those of Q4 and Q5 in cell 5, requiring some additional protection. This could be provided by an additional pair of collimators, TCT5s, with a vertical and horizontal TCT located just in front of Q5 and having exactly the same design as the TCT4s.

We evaluate these different cases in terms of halo background and are interested in how the load changes of the TCT4s when the TCT5s are included. The baseline collimator layout contains the TCT4s and TCT5s. A TCTV and TCTH pair is positioned at 131~m and 211~m, this may change in future a by a few meters due to lack of sufficient space at positions of TCT4s.

\begin{figure}%[!htb]
\begin{center}
\includegraphics[width=0.495\textwidth]{figures/inelposition_sum_tct5otrd.pdf}
\includegraphics[width=0.495\textwidth]{figures/inelposition_sum_tcotrdb2.pdf}
\includegraphics[width=0.495\textwidth]{figures/inelposition_sum_tct5inrd.pdf}
\includegraphics[width=0.495\textwidth]{figures/inelposition_sum_tcinrdb2.pdf}
\end{center}
\vspace{-0.6cm}
 \caption{Location of inelastic interactions within the collimator jaws as simulated with SixTrack for the cases ``TCT4s only'' (top) and ``TCT5s in'' (bottom) for B1 (left) and B2 (right) using \twosigmaret~settings. The numbers on the top right corner is the number of absorbed protons in the respective collimator from about 64 million simulated primaries (in all cases).
  \label{inelHLtctsInOut}}
\end{figure}

We compare the starting conditions in Fig.~\ref{inelHLtctsInOut} for the cases ``TCT4s only'', when no TCT5s were considered, and ``TCT5s in'', TCT4s and TCT5s were both included for B1 and B2. One notices that for both beams in ``TCT4s only'' inelastic interactions start at much deeper positions. The bulk load is taken by the TCTHs that comes first on the incoming beam. Globally, there are slightly more losses from B1, about 8~\%, for the case ``TCT5s in'' which can seen when comparing the leakage factors $1.2 \times 10^{-4}$ and $1.1 \times 10^{-4}$. For B2, there are negligably more losses in the case TCT5s are deployed as well and the leakage factor is in both cases around $2.4 \times 10^{-4}$ in IR1.\\

Assuming that TCT4s produce more halo backgound than the TCT5s merely due to their nearer location to the interface plane, we look at how the load on the TCT4s change for the two cases. This is listed in Tab.~\ref{tab:compLosses} for B1 and B2, IR1 and IR5. The ratio indicates by how much the intercepted losses would decrease and one can see, in all cases and improvement can be expected in the case TCT5s are included, for B1 more than for B2.


We show characteristic particle distributions and spectra at the interface plane for the case ``TCT5s~in'' for B2 in Fig.~\ref{tct5inrdb2retr} and the origin of muons along $z$ for different energies in Fig.~\ref{OrigMuon} for B1. \\

We compare in more detail the two cases with the distributions for all particles (left) and muons (right) at the interface plane in Fig.~\ref{fig:compInOutB1} for B1 and Fig.~\ref{fig:compInOutB2} for B2. The number written in the ratio plot is the ratio of the integral of the two distributions and indicates by how much the respective quantity is changed. The distributions are normalised by the number of simulated interaction in the TCTs (per TCT hit) and bin width. For both beams the energy spectrum, the azimuthal distribution $\phi$ and the energy in $\phi$ are shown.

We notice from the ratio plots that the shapes are quite different for B1. This is not the case for B2, there they look rather similar. This is entirely due to the initial distribution of inelastic interactions (convoluted of course with the geometry but this is the same for both beams).

One can expect about a factor 2 less particles, and about 50~\% less muons from B1. For B2, the particles fluxes go down by 20~\%, muons by 40~\%. Given the leakage from B2 is about twice as much as from B1, e.g.~for the case ``TCT5s~in'' it is $2.36~\times 10^{-4}$ compared to $1.1 \times 10^{-4}$ (see Tab.~\ref{leakageFactorsIR7}), even this small gain in B2 per TCT hit becomes twice as relevant in absolute terms. 

We can conclude there will be a positive effect for IR1 reducing halo-induced background if the additional collimator pair is installed. Also IR5 will benefit from backround reduction, as Tab.~\ref{tab:compLosses} shows even much more than IR1. However the IR7 leakage is generally smaller to IR5 than to IR1, factor 4 to 5 less for the two cases for B1 and significantly factor 35 and 46 for B2, again for the two cases respectively.


\begin{table}%[hbt]
   \centering
   %\vspace{-0.7cm}
   \caption{B1 and B2 losses on TCT4s  normalised on the sum of IR7 primaries and averaged over h and v simulations for the two cases (TCT5s \textit{in} and TCT4s \textit{only}) and load reduction factor on TCT4s (ratio).}
   \begin{tabular}{l|c|c|c}
       \hline
       IR & TCT5s \textit{in} &  TCT4s \textit{only} & ratio \\
       \hline\hline
       IR1 B1 & $2.3 \cdot 10^{-5}$ & $1.4 \cdot 10^{-4}$ & 6\\
       IR1 B2 & $7.0 \cdot 10^{-5}$ & $2.4 \cdot 10^{-4}$ & 3.4 \\ 
       IR5 B1 & $9.0 \cdot 10^{-7}$ & $2.7 \cdot 10^{-5}$ & 30\\
       IR5 B2 & $4.9 \cdot 10^{-7}$ & $5.2 \cdot 10^{-6}$ & 10\\

       \hline
   \end{tabular}
   \label{tab:compLosses}
\end{table}




\begin{figure}
\begin{center}
\includegraphics[width=0.495\textwidth]{figures/HL/tct5inrd/Ekin_BH_HL_tct5inrdB2_20MeV.pdf}
\includegraphics[width=0.495\textwidth]{figures/HL/tct5inrd/PhiEnDist_BH_HL_tct5inrdB2_20MeV.pdf}
\includegraphics[width=0.495\textwidth]{figures/HL/tct5inrd/RadNDist_BH_HL_tct5inrdB2_20MeV.pdf}
\includegraphics[width=0.495\textwidth]{figures/HL/tct5inrd/RadEnDist_BH_HL_tct5inrdB2_20MeV.pdf}
\end{center}
\vspace{-0.6cm}
 \caption{B2 induced particle distributions at the interface plane.}
  \label{tct5inrdb2retr}
\end{figure}


\begin{figure}
\begin{center}
\includegraphics[width=0.495\textwidth]{figures/OrigYZMuons_BH_HL_tct5otrdB1_20MeV}
\includegraphics[width=0.495\textwidth]{figures/OrigYZMuonsE100_BH_HL_tct5otrdB1_20MeV}
\includegraphics[width=0.495\textwidth]{figures/OrigYZMuons_BH_HL_tct5inrdB1_20MeV}
\includegraphics[width=0.495\textwidth]{figures/OrigYZMuonsE100_BH_HL_tct5inrdB1_20MeV}
\end{center}
\vspace{-0.6cm}
 \caption{Origin of muons for all energies (left) and with an energy above 100~GeV (right) in the y-z plane for the two cases (left, right) for B1.
  \label{OrigMuon}}
\end{figure}


\begin{figure}
\centering
\includegraphics[width=0.4\textwidth]{figures/HL/compINOUTB1_retracted/perTCThit/ratioEkinAll}
\includegraphics[width=0.4\textwidth]{figures/HL/compINOUTB1_retracted/perTCThit/ratioEkinMuons}
\includegraphics[width=0.4\textwidth]{figures/HL/compINOUTB1_retracted/perTCThit/ratioPhiNAll}
\includegraphics[width=0.4\textwidth]{figures/HL/compINOUTB1_retracted/perTCThit/ratioPhiNMuons}
\includegraphics[width=0.4\textwidth]{figures/HL/compINOUTB1_retracted/perTCThit/ratioPhiEnAll}
\includegraphics[width=0.4\textwidth]{figures/HL/compINOUTB1_retracted/perTCThit/ratioPhiEnMuons}
 \caption{Distributions of all particles (left) and muons (right) and their energy in the two cases, TCT4s only and TCT5s in, for B1.
  \label{fig:compInOutB1}}
\end{figure}




\begin{figure}
\begin{center}
\includegraphics[width=0.4\textwidth]{figures/HL/compINOUTB2_retracted/perTCThit/ratioEkinAll}
\includegraphics[width=0.4\textwidth]{figures/HL/compINOUTB2_retracted/perTCThit/ratioEkinMuons}
\includegraphics[width=0.4\textwidth]{figures/HL/compINOUTB2_retracted/perTCThit/ratioPhiNAll}
\includegraphics[width=0.4\textwidth]{figures/HL/compINOUTB2_retracted/perTCThit/ratioPhiNMuons}
\includegraphics[width=0.4\textwidth]{figures/HL/compINOUTB2_retracted/perTCThit/ratioPhiEnAll}
\includegraphics[width=0.4\textwidth]{figures/HL/compINOUTB2_retracted/perTCThit/ratioPhiEnMuons}
%\includegraphics[width=0.4\textwidth]{figures/HL/compINOUTB2_retracted/perTCThit/ratioRadEnAll}
%\includegraphics[width=0.4\textwidth]{figures/HL/compINOUTB2_retracted/perTCThit/ratioRadEnMuons}
\end{center}
\vspace{-0.6cm}
 \caption{Azimuthal distribution of all particles and muons (top) and their energy (bottom).
  \label{fig:compInOutB2}}
\end{figure}


% ------------------------------------------------------------------------------------------
%% \subsection{Comparison of B1 and B2 induced showers with TCT5s in}

%% The previous sections showed that the shape of B1 and B2 induced showers can differ significantly also when normalised per TCT hit. This section is dedicated to a detailed comparison of B1 and B2 for the case ``TCT5s~in''.

%% \begin{figure}
%% \centering
%% \includegraphics[width=0.43\textwidth]{figures/HL/compINB1B2_retracted/perTCThit/ratioEkinAll.pdf}
%% \includegraphics[width=0.43\textwidth]{figures/HL/compINB1B2_retracted/perTCThit/ratioPhiNAll.pdf}
%%  \caption{Comparison B1 and B2 induced background showers with \twosigmaret~collimator settings.
%%   \label{fig:compINB1B2_1}}
%% \end{figure}


%% \begin{figure}
%% \centering
%% \includegraphics[width=0.43\textwidth]{figures/HL/compINB1B2_retracted/perTCThit/ratioEkinMuons.pdf}
%% \includegraphics[width=0.43\textwidth]{figures/HL/compINB1B2_retracted/perTCThit/ratioPhiNMuons.pdf}
%%  \caption{Comparison B1 and B2 induced background showers with \twosigmaret~collimator settings.
%%   \label{fig:compINB1B2_3}}
%% \end{figure}

\subsection{Leakage to IR1 and IR5 TCTs with flat beam optics}

To maintain flexibility in low-$\beta^*$ reach, e.g~in the case of crab cavity failures in the experimental IRs, flat beam optics were developed~\cite{}. While in round optics, $\beta^*$ is 15~cm in IP1/5 in the horizontal and vertical plane, it is for flat beams 7.5~cm in the horizontal plane and 30~cm in the vertical plane at IP1 (vice versa for IP5)~\cite{opticsWebRef}. 

We analyse the intercepted losses in IR1 and IR5 for the two cases, TCT5s~in and TCT4s~only, using flat beam optics. The IR7 leakage to the TCT4s can be reduced only marginally with additional TCT5s, from $1.5 \times 10^{-4}$ to $1.4 \times 10^{-4}$ in IR1. In IR5, losses on TCT4s can be reduced from $1.3 \times 10^{-4}$ to $2.0 \times 10^{-5}$ which is roughly a factor 7. 

We can conclude that for flat beam optics a similar background level can be expected for B1, increasing slightly according to the integrated losses on the TCTs, $1.4 \times 10^{-4}$ for flat and $1.2 \times 10^{-4}$ for round beam optics.




\begin{figure}%[!htb]
\begin{center}
  \includegraphics[width=0.495\textwidth]{figures/HL/inelposition_sum_impacts_real_HL_TCT5IN_relaxColl_HaloB1_flatthin_IR1.pdf}
  \includegraphics[width=0.495\textwidth]{figures/HL/inelposition_sum_impacts_real_HL_TCT5IN_relaxColl_HaloB1_flatthin_IR5.pdf}
\end{center}
 \caption{Positions of intercepted protons inside the jaws for a given TCT for flat beam optics in IR1 (left) and IR5 (right). The initial number of simulated halo protons was for each simulation case (hB1, vB1) 64 million using \twosigmaret~settings and having TCT5s included.
   \label{fig:inelflat}}
\end{figure}

\newpage
\section{Evolution of Background in LHC and Comparison to HL--LHC\label{evolut}}

We focus on the evolution of beam-gas and halo showers in IR1 for the Run~I, II and HL scenario highlighting muons as they are usually the most important particles for measuring background from the machine side. Other selected particle distributions can be found in the App.~\ref{evolutApp}.

\subsection{Comparison of background sources in 2012 and 2015}


We analyse properties of muons created in beam-gas and halo interactions. A shape comparison is made in Fig.~\ref{fig:compAllBKG_muons} for different background sources simulated for the Run~I at 4~TeV showing the energy spectrum of muons and the azimuthal distribution of their energy. One notices in for both plots, that sources from the TCT impacts as halo and offmomentum induced are very similar distributions at the interface plane. The beam-gas shape of the left figure has a few distinct features, it starts ealier at low energy and has one a dint at 10 to 100~GeV, but beam-gas overtakes the tctimpacts for the energies just as close as the beam energy. The other figure exhibits another feature: Muons from beam-gas give highest contributions at $0$ and $\pm$, but lowest in between, which means their energy is mostly horizontally distributed.\\

\begin{figure}%[htp]
\begin{center}
  \includegraphics[width=0.42\textwidth]{figures/4TeV/compAllBKG/EkinMuons.pdf}
  \includegraphics[width=0.42\textwidth]{figures/4TeV/compAllBKG/PhiEnMuons.pdf}
\end{center}
\vspace{-0.6cm}
 \caption{Comparison of all background sources at 4 TeV normalised per interaction showing energy spectrum and energy in $\phi$ for muons.
  \label{fig:compAllBKG_muons}}
\end{figure}
To visualise the spatial distribution of muons, we show them for beam-gas (left) and halo on the TCTs (right) in Fig.~\ref{fig:XYNMuons} for the same scenario at 4~TeV. One can recognise a geometrical effect of the beginning of the ATLAS cavern, a rectangular shape with a higher particle flux inside. At $s \approx 28~m$, a concrete shielding starts blocking particles leaving air only up to $\pm 100~$cm in $x$ and $y$. The last quadrupole of the triplet (MQXA) before the IP extends up to around $r \approx 46~$ (including tank and vacuum vessel) which is also visible in that figure. While muons in beam-gas collisions are created inside the vacuum tube and most of them stay inside, muons from halo showers are rather shielded by the vessel and higher rates can be found outside.


\begin{figure} %[htp]
  \centering
  \includegraphics[width=0.495\textwidth]{figures/XYNMuons_BG_4TeV_20MeV_bs.pdf}
  %\includegraphics[width=0.495\textwidth]{figures/XYNMuons_BG_6500GeV_flat_20GeV_bs.pdf}  
  \includegraphics[width=0.495\textwidth]{figures/XYNMuons_BH_4TeV_B1_20MeV.pdf}
  %\includegraphics[width=0.495\textwidth]{figures/XYNMuons_BH_6500GeV_haloB1_20MeV.pdf}
  \caption{Spatial distribution of muons of all energies in an beam-gas (left) and beam-halo (right) scenario for a 4~TeV beam in Run I 2012. At 6.5~TeV the distributions look similar, see Fig.~\ref{fig:XYNMuons2}.
    \label{fig:XYNMuons}}
\end{figure}


We normalise the results to rates towards ATLAS, using Eq.~\ref{eqNormHalo}, the run parameteres as listed in Tab.~\ref{paramsRun12} and IR7 leakages from Tab.~\ref{leakageFactorsIR7} and present the rates for 4~TeV in Fig.~\ref{compAllBKG_run12}. In all cases, beam-gas contributions are significantly higher than from those from halos. At 4~TeV, halo contributions are below a percent-level for particles and around 2~\% for energy produced from what is produced in beam-gas. This increases by about an order of magnitude for Run~II, where halo-muons form 20~\% and 25~\% of beam-gas muons. However, in absolute terms beam-gas rates are lower than in 2012, as mentioned due to the improved vacuum stability. One can also read off that figure that at 4~TeV, B2 produced double the amount of B1, while in 2015 both contributions are almost equal.

What is new in 6.5~TeV is that the muon rates produced in the arc follow now a profile with spikes (at transition regions of the bending magnets) while rates at 3.5 and 4~TeV in the arc are dominated by the assumption of contstant pressure (for 3.5~TeV pressures and analysis see Ref.~\cite{nimPaperRod}). Another interesting thing to note are the rates within the triplet (up to $s =~50~$m). The one at 6.5~TeV is above the one at 4~TeV, although anywhere else it is on the lowest level.

The rates are shown in Fig.~\ref{fig:compBGreweighted1} to compare directly the 4 and 6.5~TeV case of reweighted beam-gas. One clearly sees the shape has changed, the ratio plot on the bottom of the left figure around shows that from 20 to 150~GeV there were fewer muons, up to a factor 5 less at 6.5~TeV then at 4~TeV. In the right figure one can see that the energy difference carried by muons was reduced thereby almost factor 4. The shape is clearly different: while most of muons are horizontally distributed at 4~TeV, they show at 6.5~TeV now peaks in the vertical plane as well. This regular structure must have it origins in the magnetic fields and as one can see from earlier figures (Fig.~\ref{fig:XYNMuons2}, \ref{fig:EkinPhiEn6p52}) these muons are in the centre of the beampipe. 
% --------------------------------------------------------------------------------------------

\begin{figure}
\begin{center}
  \includegraphics[width=0.49\textwidth]{figures/4TeV/reweighted/cv78_EkinMuons.pdf}
  \includegraphics[width=0.49\textwidth]{figures/4TeV/reweighted/cv78_PhiEnMuons.pdf}
  \includegraphics[width=0.49\textwidth]{figures/6500GeV/reweighted/cv78_EkinMuons.pdf}
  \includegraphics[width=0.49\textwidth]{figures/6500GeV/reweighted/cv78_PhiEnMuons.pdf}
\end{center}
\vspace{-0.6cm}
 \caption{Rate comparison of halo and beam-gas at 4~TeV (top) and 6.5~TeV (bottom). The numbers in the legend indicate the fraction to the beam-gas data of the respective scenario. They are formed from the integral of the distributions.
  \label{compAllBKG_run12}}
\end{figure}

\begin{figure}%[!htb]
\centering

\includegraphics[width=0.45\textwidth]{figures/compBGreweighted/ratioEkinMuons.pdf}
\includegraphics[width=0.45\textwidth]{figures/compBGreweighted/ratioPhiEnMuons.pdf}
\caption{Reweighted beam-gas distributions in the 2012 Run I and 2015 Run II scenario for muons showing the energy spectrum (left) and the azimuthal distribution of the energy (right).
  \label{fig:compBGreweighted1}}
\end{figure}


\begin{figure}
\begin{center}
%  \includegraphics[width=0.8\textwidth]{figures/cv87_allenergies_OrigZAll.pdf}
  \includegraphics[width=0.8\textwidth]{figures/cv87_allenergies_OrigZMuon.pdf}
\end{center}
\vspace{-0.6cm}
 \caption{Evolution of muon rates. 
  \label{fig:OrigZMuon}} 
\end{figure}




\begin{figure}
\begin{center}

  \includegraphics[width=0.49\textwidth]{figures/compBHB1_4TeV_vs_6p5TeV/normalised/ratioEkinMuons.pdf}
  \includegraphics[width=0.49\textwidth]{figures/compBHB1_4TeV_vs_6p5TeV/normalised/ratioPhiEnMuons.pdf}

\end{center}
\vspace{-0.6cm}
 \caption{Comparison of halo induced background at 4 and 6.5~TeV in the azimuthal distributions of all particles at the interface plane (top) and high-energy muons and protons (bottom) and their energy.
  \label{compBHB1run1run2}}
\end{figure}


\begin{figure}
\centering
  \includegraphics[width=0.49\textwidth]{figures/compBHB2_4TeV_vs_6p5TeV/normalised/ratioEkinMuons.pdf}
  \includegraphics[width=0.49\textwidth]{figures/compBHB2_4TeV_vs_6p5TeV/normalised/ratioPhiEnMuons.pdf}
 \caption{Comparison of halo induced background at 4 and 6.5~TeV in the azimuthal distributions of all particles at the interface plane (top) and high-energy muons and protons (bottom) and their energy.
  \label{compBHB1run1run2}}
\end{figure}




\subsection{Run II vs HL--LHC}

\begin{figure}
\begin{center}
  \includegraphics[width=0.42\textwidth]{figures/HLRunII/cv78_EkinMuons.pdf}
  \includegraphics[width=0.42\textwidth]{figures/HLRunII/cv78_PhiEnMuons.pdf}
\end{center}
\vspace{-0.6cm}
 \caption{Comparison of beam-gas (BG) in Run II and beam-halo in HL using the baseline layout (TCT5s in, \twosigmaret~settings) and round beam optics. Characteristics of all particles is shown in Fig.~\ref{fig:compHLRun2All}.
  \label{fig:hlrun2}}
\end{figure}

\section{Conclusion and outlook~\label{last}}

We investigated in detail and systematically two sources of machine-induced background, beam-gas and betatron halo, two of the most relevant sources for the experiments. The study was carried out for IR1 and both beams in general. Although there is a layout symmetry for incoming and outgoing beams in IR1 and IR5, both beams traverse different parts of the machine to arrive there from IR7, as well as have different crossing planes and transverse rotation of the beam screen, which could result in different backround patterns. We studied in detail IR7 betatron leakages to IR1 and IR5 for the $pp$ physics cases of Run~I in 2012, Run~II in 2015 and and several HL-LHC cases. For betatron halo, we find that B2 gives systematically higher leakages than B1 in IR1 in all simulated cases, see Tab.~\ref{tab:leakageFactorsIR7}. Although IR5 leakages from IR7 can be significantly smaller than the leakage to IR1, this is not necessarily generally the case: it is true for 2015 Run~II and the HL-LHC reference case (TCT5s in, round beam optics, \twosigmaret~settings), but not for the Run~I case at 4~TeV when the leakage is on a similar level for B1 and slightly higher for B2 (IR1: $2.4 \times 10^{-5}$, IR5: $2.6 \times 10^{-5}$).

Tracking simulations with the main settings for proton physics in 2016 were recently performed and one can see from Tab.~\ref{2016leakageFactorsIR7} that the different optics can cause B2 to contribute only about a third of B1 despite the shorter path length of B2 from IR7 to IR1 (IR1, B1: $4.2 \times 10^{-4}$, IR1 B2:$1.2 \times 10^{-4}$). These are examples that show that a more detailed analysis e.g.~considering the phase advance of the collimators are needed to understand more precisely collimation leakages to the experimental areas. We also showed examples e.g.~when comparing the HL-LHC baseline case to the current collimation layout of only one pair of TCTs installed in the experimental IR, that the surface hits in the collimator jaws cause most of the shower development at the interface plane. If one could reduce such interactions, e.g.~by retracting the TCT4s more with respect to the TCT5s it may be well beneficial, but it remains to be validated by new simulations.

The shower simulations of local beam-gas in IR1 were performed with new improved techniques sampling events according to transverse extension of the local beam size and including the crossing angle. The inclusion of the beam size could be directly compared to simulations where the beam-gas events were sampled only along the central orbit, and we find minor changes in the distributions (see Fig.~\ref{bsRatioPhiAll} and \ref{bsZAll}) at the interface plane. Since it is a more realistic model we used the new method also in Run~II simulations. The crossing angle was found to have a major impact in angular distributions at the interface plane.

We also presented for the first time a simulation analysis of off-momentum halo leakage as third background source to the experiments, simulating particle shower development to IR1. A strong B1/B2 asymmetry is visible in the simulations concerning the amount of off-momentum halo leakage to IR1 and IR5. IR1 recieves almost entirely only contributions from B2 while background at IR5 of this source is exclusively produced by B1. These simulations show that off-momentum halo is also cleaned by the betatron insertion at IR7 which is situated only two octants before IR1 for B1 and also two octants before IR5 for B2. From experience with the machine it is known, that off-momentum losses are also caught in IR7 and fractions of the betratron losses are cleaned in IR3. On an absolute scale, betatron losses are much higher than off-momentum losses and the contribution from off-momemtum halo on the TCT to the total particle flux at the interface plane remains to be quantitatively estimated.
Generally, off-momentum induced distributions show similar shapes at the interface plane as the IR7 losses. We observe another interesting feature when analysing pure off-momentum halos in the scenarios of 2012 and 2015. At 4 TeV in 2012, the simulations with a positive frequency shift (corresponding to a negative dp/p) produces almost exactly twice as much as when we simulate with a negative frequency shift. The same behavior is observed for IR5, although for slightly smaller IR3 leakages. However in the 2015 scenario at 6.5~TeV, the leakages to IR1 and IR5 are very similar even for both frequency shifts, see Tab.~\ref{tab:IR3leakageFactors} for details. This is not well understood and can be an interesting study with a dedicated analysis, however as mentioned IR3 losses due to off-momentum particles are so far not estimated to be problematic for the experiments. 


While in Run I in 2011 at 3.5~TeV, in 2012 at 4~TeV as well in 2015 in Run~II at 6.5~TeV local beam-gas was the dominating source creating background to the experiments, betatron halo can, in HL-LHC, become of about equal importance to local beam-gas. This can be the case when optimistic scenarios for the vacuum quality are assumed as shown Fig.~\ref{fig:HLR2Muons} with the currently best estimate for vacuum in HL-LHC. However, it should be noted that all quantitative estimates for HL-LHC have a very high uncertainty.

It should be also noted that the comparisons presented in this note concern the distributions at the machine-detector interface plane, and that the actual background in the experiment depends also on the detector response. Therefore, in order quantify the background, a further simulation step of the detector is required. The simulations described here can be used as starting conditions for such a study.

%Implications for the machine.
Any measures which help to improve the vacuum quality have proven to be very useful in terms of background reduction from beam-gas like machine conditioning (scrubbing). Particularly effective were also all the hardware measures during LS1, with which pressures in the experimental IRs were partially lowered below sensitivity of some gauges.






%-------------------------------------------------------------------------------------------
\section*{Acknowledgments}
%

%-------------------------------------------------------------------------------------------
\newpage
\clearpage
\appendix
\newpage
\section{HL loss maps \label{lossmapszooms}}
% ---------------------------------------------------------------------------------------------------------------------
% fullring round/flat

We show here zooms of the full loss maps of round and flat beam optics for HL-LHC and \twosigmaret~settings per simulation case, for a horizontal and vertical halo distribution in SixTrack. First we validate the settings by looking at IR7 losses, comparing nominal and \twosigmaret~settings, then we analyse the zooms for IR1 and IR5 to evaluate the affect of additional collimators in the experimental IRs. The beam direction in all figures, IR7 zooms Fig.~\ref{IR7_zooms}, IR1 round beam 1 comparisons Fig.~\ref{IR1_roundB1}, IR5 round beam 1 comparisons in Fig.~\ref{IR5_roundB1} and IR1/IR5 for flat beam 1 in Fig.~\ref{IR1_flatB1}/\ref{IR5_flatB1}, is from left to right. 

\subsubsection{Losses at IR1/IR5 tertiary collimators in HL--LHC}

A zoom into loss locations in the experimental IRs of ATLAS and CMS are shown in Fig.~\ref{IR15_roundB1_nomSett} for nominal collimator settings. The similar set of zooms are shown and discussed for the \twosigmaret~settings in the Appendix~\ref{lossmapszooms}. The beam direction is from left to right. The two black bars at upstream of the IP are the losses on the pair of tertiary collimators (TCT4s and TCT5s), while the black bars downstream are losses on TCLs, debris collimators. These losses are normalised to total number of lost particles. One can see, IR5 losses are generally lower than in IR1, an expected feature known from Run I and II. A more direct comparison of the losses is made in Fig.~\ref{compTCT5INOUT}. The losses are normalised to the number of simulated primary per simulation case, then the sum per collimator is shown, i.e. as example the bin of a specific TCTH is set to $\big(\frac{\mathrm{hits\,on\,TCTH}}{\#\mathrm{primary}}\big)_{\mathrm{h}} + \big(\frac{\mathrm{hits\,on\,TCTH}}{\#\mathrm{primary}}\big)_{\mathrm{v}}$. The simulation cases shown are in Fig.~\ref{compTCT5INOUT}~(a) for round, and Fig.~\ref{compTCT5INOUT}~(b) for flat beams.

\begin{figure} [!htb]
\begin{center}

\includegraphics[width=0.48\textwidth]{figures/lossmaps/coll_loss_H5_HL_nomSett_hHalo_b1_IR1}
\includegraphics[width=0.48\textwidth]{figures/lossmaps/coll_loss_H5_HL_nomSett_vHalo_b1_IR1}
\includegraphics[width=0.48\textwidth]{figures/lossmaps/coll_loss_H5_HL_nomSett_hHalo_b1_IR5}
\includegraphics[width=0.48\textwidth]{figures/lossmaps/coll_loss_H5_HL_nomSett_vHalo_b1_IR5}
\end{center}
\vspace{-0.3cm}
 \caption{Zoom into IR1 (top) and IR5 (bottom, IP5 is at 133,300~m) when the TCT5s were inserted using round optics. Horizontal beam 1 is on the left, vertical beam 1 on the right.
  \label{IR15_roundB1_nomSett}}
\end{figure}



%% \begin{figure}
%% \begin{center}
%% \vskip-12mm
%% \includegraphics[width=0.92\textwidth]{figures/lossmaps/coll_loss_H5_HL_TCT5LOUT_relaxColl_hHaloB1_roundthin_fullring}
%% \includegraphics[width=0.92\textwidth]{figures/lossmaps/coll_loss_H5_HL_TCT5LOUT_relaxColl_vHaloB1_roundthin_fullring}
%% \end{center}
%% \vspace{-0.3cm}
%%  \caption{Loss maps for horizontal (top) and vertical (bottom) B1 halo using round optics with TCT4s only.
%%   \label{fullring_roundB1_TCT5LOUT }}
%% \end{figure}

%% \begin{figure}
%% \begin{center}
%% \vskip-12mm
%% \includegraphics[width=0.92\textwidth]{figures/lossmaps/coll_loss_H5_HL_TCT5IN_relaxColl_hHaloB1_roundthin_fullring}
%% \includegraphics[width=0.92\textwidth]{figures/lossmaps/coll_loss_H5_HL_TCT5IN_relaxColl_vHaloB1_roundthin_fullring}
%% \end{center}
%% \vspace{-0.3cm}
%%  \caption{Loss maps for horizontal (top) and vertical (bottom) B1 halo using round optics with TCT4s and TCT5s.
%%   \label{fullring_roundB1_TCT5IN}}
%% \end{figure}


%% \begin{figure}
%% \begin{center}
%% \vskip-12mm
%% \includegraphics[width=0.92\textwidth]{figures/lossmaps/coll_loss_H5_HL_TCT5LOUT_relaxColl_hHaloB1_flatthin_fullring}
%% \includegraphics[width=0.92\textwidth]{figures/lossmaps/coll_loss_H5_HL_TCT5LOUT_relaxColl_vHaloB1_flatthin_fullring}
%% \end{center}
%% \vspace{-0.3cm}
%%  \caption{Loss maps for horizontal (top) and vertical (bottom) B1 halo using flat optics with TCT4s only.
%%   \label{fullring_flatB1_TCT5LOUT}}
%% \end{figure}

%% \begin{figure}
%% \begin{center}
%% \vskip-12mm
%% \includegraphics[width=0.92\textwidth]{figures/lossmaps/coll_loss_H5_HL_TCT5IN_relaxColl_hHaloB1_flatthin_fullring}
%% \includegraphics[width=0.92\textwidth]{figures/lossmaps/coll_loss_H5_HL_TCT5IN_relaxColl_vHaloB1_flatthin_fullring}
%% \end{center}
%% \vspace{-0.3cm}
%%  \caption{Loss maps for horizontal (top) and vertical (bottom) B1 halo using flat optics with TCT4s and TCT5s.
%%   \label{fullring_flatB1_TCT5IN}}
%% \end{figure}

% ---------------------------------------------------------------------------------------------------------------------
% IR7 zooms

\begin{figure}[!htb]
\begin{center}
%\vskip-12mm
\includegraphics[width=0.48\textwidth]{figures/lossmaps/coll_loss_H5_HL_nomSett_hHalo_b1_IR7}
\includegraphics[width=0.48\textwidth]{figures/lossmaps/coll_loss_H5_HL_nomSett_vHalo_b1_IR7}
\includegraphics[width=0.48\textwidth]{figures/lossmaps/coll_loss_H5_HL_TCT5IN_relaxColl_hHaloB1_roundthin_IR7}
\includegraphics[width=0.48\textwidth]{figures/lossmaps/coll_loss_H5_HL_TCT5IN_relaxColl_vHaloB1_roundthin_IR7}
\includegraphics[width=0.48\textwidth]{figures/lossmaps/coll_loss_H5_HL_TCT5IN_relaxColl_hHaloB1_flatthin_IR7}
\includegraphics[width=0.48\textwidth]{figures/lossmaps/coll_loss_H5_HL_TCT5IN_relaxColl_vHaloB1_flatthin_IR7}
\end{center}
%% \begin{picture} (0.,0.)
%% \setlength{\unitlength}{1.0cm}
%% \small{
%%     \put ( 4.,7.35){(a)}
%%     \put ( 12.4,7.35){(b)}
%%     \put ( 4.,1.){(c)}
%%     \put ( 12.4,1.){(d)}
%% }
%% \end{picture}
\vspace{-0.3cm}
 \caption{Zoom into IR7 for round with nominal (top) and \twosigmaret~settings (middle), and flat optics and \twosigmaret~settings (bottom). Horizontal beam 1 is on the left, vertical beam 1 on the right.
  \label{IR7_zooms}}
\end{figure}

The view to IR7 in Fig.~\ref{IR7_zooms} shows the cleaning inefficiency, leaking protons from IR7 primaries and secondaries, is most critical in the two prominent cold region blocks. They usually serve as a benchmark region for collimator settings and optics as the maximum energy is deposited in these blocks. The maximum cold loss is at 10$^{-4}$ for \twosigmaret~round and flat beams, and a factor 2 lower with nominal settings.

\begin{figure}[!tb]
\begin{center}
%\vskip-12mm
\includegraphics[width=0.48\textwidth]{figures/lossmaps/coll_loss_H5_HL_TCT5LOUT_relaxColl_hHaloB1_roundthin_IR1}
\includegraphics[width=0.48\textwidth]{figures/lossmaps/coll_loss_H5_HL_TCT5LOUT_relaxColl_vHaloB1_roundthin_IR1}
\includegraphics[width=0.48\textwidth]{figures/lossmaps/coll_loss_H5_HL_TCT5IN_relaxColl_hHaloB1_roundthin_IR1}
\includegraphics[width=0.48\textwidth]{figures/lossmaps/coll_loss_H5_HL_TCT5IN_relaxColl_vHaloB1_roundthin_IR1}
\end{center}
%% \begin{picture} (0.,0.)
%% \setlength{\unitlength}{1.0cm}
%% \small{
%%     \put ( 4.,7.35){(a)}
%%     \put ( 12.4,7.35){(b)}
%%     \put ( 4.,1.){(c)}
%%     \put ( 12.4,1.){(d)}
%% }
%% \end{picture}
\vspace{-0.3cm}
 \caption{Zoom into IR1 (IP1 is at 0~m) for TCT5s out (top) and TCT5s in (bottom) B1 halo using round optics. Horizontal beam 1 is on the left, vertical beam 1 on the right.
  \label{IR1_roundB1}}
\end{figure}

The losses in IR1 with and without additional TCT5s are shown in Fig.~\ref{IR1_roundB1} for the more realistic \twosigmaret~setting. The nominal settings are in Fig.~\ref{IR5_roundB1_nomSett}
% -------------------------------------------------------------------------------------------------------------------

\begin{figure}
\begin{center}
\vskip-12mm
\includegraphics[width=0.48\textwidth]{figures/lossmaps/coll_loss_H5_HL_TCT5LOUT_relaxColl_hHaloB1_roundthin_IR5}
\includegraphics[width=0.48\textwidth]{figures/lossmaps/coll_loss_H5_HL_TCT5LOUT_relaxColl_vHaloB1_roundthin_IR5}
\includegraphics[width=0.48\textwidth]{figures/lossmaps/coll_loss_H5_HL_TCT5IN_relaxColl_hHaloB1_roundthin_IR5}
\includegraphics[width=0.48\textwidth]{figures/lossmaps/coll_loss_H5_HL_TCT5IN_relaxColl_vHaloB1_roundthin_IR5}
\end{center}
\vspace{-0.3cm}
 \caption{Zoom into IR5 (IP5 is at 133,300~m) for TCT5s out (top) and TCT5s in (bottom) B1 halo using round optics. Horizontal beam 1 is on the left, vertical beam 1 on the right.
  \label{IR5_roundB1}}
\end{figure}

\begin{figure}
\begin{center}
\vskip-12mm
\includegraphics[width=0.48\textwidth]{figures/lossmaps/coll_loss_H5_HL_TCT5LOUT_relaxColl_hHaloB1_flatthin_IR1}
\includegraphics[width=0.48\textwidth]{figures/lossmaps/coll_loss_H5_HL_TCT5LOUT_relaxColl_vHaloB1_flatthin_IR1}
\includegraphics[width=0.48\textwidth]{figures/lossmaps/coll_loss_H5_HL_TCT5IN_relaxColl_hHaloB1_flatthin_IR1}
\includegraphics[width=0.48\textwidth]{figures/lossmaps/coll_loss_H5_HL_TCT5IN_relaxColl_vHaloB1_flatthin_IR1}
\end{center}
%% \begin{picture} (0.,0.)
%% \setlength{\unitlength}{1.0cm}
%% \small{
%%     \put ( 4.,7.35){(a)}
%%     \put ( 12.4,7.35){(b)}
%%     \put ( 4.,1.){(c)}
%%     \put ( 12.4,1.){(d)}
%% }
%% \end{picture}
\vspace{-0.3cm}
 \caption{Zoom into IR1 (IP1 is at 0~m) for TCT5s out (top) and TCT5s in (bottom) B1 halo using flat optics. Horizontal beam 1 is on the left, vertical beam 1 on the right.
  \label{IR1_flatB1}}
\end{figure}


\begin{figure}
\begin{center}
\vskip-12mm
\includegraphics[width=0.48\textwidth]{figures/lossmaps/coll_loss_H5_HL_TCT5LOUT_relaxColl_hHaloB1_flatthin_IR5}
\includegraphics[width=0.48\textwidth]{figures/lossmaps/coll_loss_H5_HL_TCT5LOUT_relaxColl_vHaloB1_flatthin_IR5}
\includegraphics[width=0.48\textwidth]{figures/lossmaps/coll_loss_H5_HL_TCT5IN_relaxColl_hHaloB1_flatthin_IR5}
\includegraphics[width=0.48\textwidth]{figures/lossmaps/coll_loss_H5_HL_TCT5IN_relaxColl_vHaloB1_flatthin_IR5}
\end{center}
%% \begin{picture} (0.,0.)
%% \setlength{\unitlength}{1.0cm}
%% \small{
%%     \put ( 4.,7.35){(a)}
%%     \put ( 12.4,7.35){(b)}
%%     \put ( 4.,1.){(c)}
%%     \put ( 12.4,1.){(d)}
%% }
%% \end{picture}
\vspace{-0.3cm}
 \caption{Zoom into IR5 (IP5 is at 133,300~m) with TCT5s out (top) and TCT5s in (bottom) B1 halo using flat optics. Horizontal beam 1 is on the left, vertical beam 1 on the right.
  \label{IR5_flatB1}}
\end{figure}

% ---------------------------------------------------------------------------------------------------------------------
\newpage
\section{Appendix B}
\section{Full tables of SixTrack TCP-to-TCT conversion factors}
\subsection{IR7 to TCT conversion factors in halo cleaning simulations}
\begin{table}
   \centering
   \caption{Intercepted protons on IR7 TCPs (\textsc{TCP.D6, TCP.C6, TCP.B6}), IR1/5 and IR7 leakage to IR1/IR5 TCTs using Eq.~\ref{eq3}.}

   \begin{tabular}{c|cc|cc}

       & hB1 & vB1 & hB2 & vB2\\ \hline       
       & \multicolumn{4}{c}{4 TeV 2012} \\   %\cline{2-5}%\hline
       TCPs in IR7 & 50807535 & 53036514 & 49207325 & 46222723 \\
       TCTs in IR1 & 622 & 930 & 1179 & 967 \\
       ICTs in IR5 & 958 & 626 & 1893 & 135 \\ %
       IR7 to IR1  & \multicolumn{2}{c|}{1.5 $\times 10^{-5}$} & \multicolumn{2}{c}{2.4 $\times 10^{-5}$ } \\
       IR7 to IR5  & \multicolumn{2}{c|}{1.5 $\times 10^{-5}$} & \multicolumn{2}{c}{2.6 $\times 10^{-5}$ } \\
       \hline
       & \multicolumn{4}{c}{6.5 TeV 2015} \\      
       total losses & 62515929 & 62692523 & 50890652 & 63119778 \\
       %     peak loss in IR7 & 5.18402e+07 & 
       TCPs in IR7 & 53731448 & 52806720 & 43692659 & 52962459 \\
       TCTs in IR1 & 739 & 585 & 779 & 773 \\
       TCTs in IR5 & 346 & 408 & 302 & 106 \\% \cline{2-5}
       IR7 to IR1 &  \multicolumn{2}{c|}{1.2 $\times 10^{-5}$} &  \multicolumn{2}{c}{1.6 $\times 10^{-5}$ } \\
       IR7 to IR5 &  \multicolumn{2}{c|}{7.1 $\times 10^{-6}$} &  \multicolumn{2}{c}{4.5 $\times 10^{-6}$ } \\
       \hline       

       & \multicolumn{4}{c}{HL nominal settings, TCT5s in, round beam}  \\ %see lhc_mib/HL1.0/factors
       
       TCPs in IR7 & 52836357 & 50278617 & & \\
       TCTs in IR1 & 32557 &15813 & & \\
       TCTs in IR5 & 7500 & 3154   & & \\
       IR7 to IR1  &  \multicolumn{2}{c|}{ 4.7 $\times 10^{-4}$ }& &   \\ 
       IR7 to IR5  &  \multicolumn{2}{c|}{ 1.0 $\times 10^{-4}$} & &  \\ 
       \hline
       & \multicolumn{4}{c}{ HL \twosigmaret~settings, TCT5s in, round beam }  \\
       TCPs in IR7 & 54532193 & 52154816 & 40401333 & 53199970 \\
       TCTs in IR1 & 9712 & 3366 & 9948 &  12028\\
       TCTs in IR5 & 1506 & 1122 & 473  & 95 \\
       IR7 to IR1  &  \multicolumn{2}{c|}{ 1.2 $\times 10^{-4}$ } &  \multicolumn{2}{c} { 2.36 $\times 10^{-4}$} \\
       IR7 to IR5 & \multicolumn{2}{c|}{ 2.5 $\times 10^{-5}$} & \multicolumn{2}{c} {6.8 $\times 10^{-6}$ } \\
       \hline
       & \multicolumn{4}{c}{ HL \twosigmaret~settings, TCT4s only, round beam }  \\
       TCPs in IR7 & 54609869 & 52175081 & 40392116 & 53157089 \\
       TCTs in IR1 & 9024 & 3071 & 9936 & 11898 \\
       TCTs in IR5 & 1408 & 1024 & 368 & 70 \\
       IR7 to IR1  &  \multicolumn{2}{c|}{ 1.1 $\times 10^{-4}$ } &  \multicolumn{2}{c} { 2.35 $\times 10^{-4}$} \\
       IR7 to IR5  &  \multicolumn{2}{c|}{ 2.7 $\times 10^{-5}$ } &  \multicolumn{2}{c} { 5.2 $\times 10^{-6}$} \\
       \hline
       & \multicolumn{4}{c}{ HL \twosigmaret~settings, TCT5s in, flat beam }  \\
       TCPs in IR7 & 35831038 & 5052849 &  &  \\
       TCTs in IR1 & 7814 & 306 & & \\
       TCTs in IR5 & 660 & 75 & \\
       IR7 to IR1  & \multicolumn{2}{c|}{1.4 $\times 10^{-4}$ } &  \\
       IR7 to IR5 & \multicolumn{2}{c|}{1.7 $\times 10^{-5}$} & \\
       \hline
       & \multicolumn{4}{c}{ HL \twosigmaret~settings, TCT4s only, flat beam }  \\
       TCPs in IR7 & 14320971 & 50517818 &  &  \\
       TCTs in IR1 & 3035 & 3035 & & \\
       TCTs in IR5 & 338 & 824 & \\
       IR7 to IR1  & \multicolumn{2}{c|}{1.4 $\times 10^{-4}$ } &  \\
       IR7 to IR5 & \multicolumn{2}{c|}{2.0 $\times 10^{-5}$} & \\

       %% HL retracted settings, flat beam  & B1 \\       
       %% IR1 & 1.39 $\times 10^{-4}$ & \\ % B1 : .5*(7814.0/35831038.0 + 306.0/5052849.0)
       %% IR5 & 1.66 $\times 10^{-5}$ & \\ % B1 : .5*(660.0/35831038.0 + 75.0/5052849.0)
       \hline
   \end{tabular}
   \label{leakageFactorsIR7}
\end{table}


\begin{table}
   \centering
   \caption{Intercepted protons on IR7 TCPs (\textsc{TCP.D6, TCP.C6, TCP.B6}), IR1/5 and IR7 leakage to IR1/IR5 TCTs using Eq.~\ref{eq3}. In the last line the leakage was calculated wrt total losses in the simulations.}
   \begin{tabular}{c|cc|cc}
     \hline
       & \multicolumn{4}{c}{6.5 TeV 2016} \\      
       total losses & 19062803 & 18407758 & 19082325 & 18404556 \\       
       TCPs in IR7 & 16245854 & 15577729 & 16268025 & 15583550 \\
       TCTs in IR1 & 8346 & 5083 & 2138 & 1641 \\
       TCTs in IR5 & 2563 & 3709 & 2500 & 970 \\% \cline{2-5}
       IR7 to IR1 &  \multicolumn{2}{c|}{4.2 $\times 10^{-4}$} &  \multicolumn{2}{c}{1.2 $\times 10^{-4}$ } \\
       IR7 to IR5 &  \multicolumn{2}{c|}{2.0 $\times 10^{-4}$} &  \multicolumn{2}{c}{1.1 $\times 10^{-4}$ } \\
       IR7 to IR1 &  \multicolumn{2}{c|}{3.6 $\times 10^{-4}$} &  \multicolumn{2}{c}{1.0 $\times 10^{-4}$ } \\

       \hline
   \end{tabular}
   \label{2016leakageFactorsIR7}
\end{table}


\subsection{IR3 to TCT conversion factors in off-momentum simulations}
\begin{table}[!h]
   \centering
   \caption{Leakage from IR3 primary collimator to IR1/IR5 TCTs from off-momentum simulations. The numbers in brackets show the losses per TCT, TCTH + TCTV. }

   \begin{tabular}{l|c|c}
       \hline
       4 TeV, + 500~Hz  & B1 & B2\\
       TCP.IR3  & 6279727 & 995698  \\
       TCTs IR1 & 6 (6 + 0) & 11919 (695 + 11224) \\
       TCTs IR5 & 28304 (5387 + 22917) & 16 (14 + 2) \\
       IR3 to IR1 & 9.5 10$^{-7}$ & 0.012 \\
       IR3 to IR5 & 0.0045 & 1.6 10$^{-5}$ \\
       \hline
       4 TeV, -~500~Hz  & B1 & B2\\
       TCP.IR3  & 1853387 & 3501844 \\
       TCTs IR1  & 93 (37 + 56) & 22278 (4851+17427) \\
       TCTs IR5  &  4735 (1746 + 2989) & 23 (22 + 1) \\
       IR3 to IR1 & 5.0 10$^{-5}$ & 0.0063 \\
       IR3 to IR5 & 0.0025 & 6.2 10$^{-6}$ \\
       \hline

       6.5 TeV, + 500~Hz  & B1 & B2\\
       TCP.IR3  & 8055019 & 5648325  \\
       TCTs IR1 & 3 (3 + 0) & 11951 (5607 + 6344) \\
       TCTs IR5 & 17963 (8711 + 9252) & 9 (9 + 0)\\
       IR3 to IR1 & 3.7 10$^{-7}$ & 0.0021 \\
       IR3 to IR5 & 0.0022 & 1.5 10$^{-6}$ \\
       \hline
       6.5 TeV, - 500~Hz  & B1 & B2\\
       TCP.IR3  &  3389385 &   \\
       TCTs IR1  &  1 (1 + 0) &   \\
       TCTs IR5  &  7642 (4337 + 3305) &  \\
       IR3 to IR1 &  2.9 10$^{-7}$ &  \\
       IR3 to IR5 &  0.0022 &   \\
       \hline

   \end{tabular}
   \label{tab:IR3leakageFactors}
\end{table}

%% \begin{figure}
%% \begin{center}
%%   \includegraphics[width=0.49\textwidth]{figures/4TeV/bs_20MeV/RadNMuons_BG_4TeV_20MeV_bs.pdf}
%%   \includegraphics[width=0.49\textwidth]{figures/6500GeV/20MeV/RadNMuons_BG_6500GeV_flat_20MeV_bs.pdf}
%%   \includegraphics[width=0.49\textwidth]{figures/4TeV/haloB1_20MeV/RadNMuons_BH_4TeV_B1_20MeV.pdf}
%%   \includegraphics[width=0.49\textwidth]{figures/4TeV/haloB2_20MeV/RadNMuons_BH_4TeV_B2_20MeV.pdf}
%%   \includegraphics[width=0.49\textwidth]{figures/4TeV/offmom/20MeV/RadNMuons_offplus500Hz_4TeV_B2_20MeV.pdf}
%%   \includegraphics[width=0.49\textwidth]{figures/4TeV/offmom/20MeV/RadNMuons_offmin500Hz_4TeV_B2_20MeV.pdf}
%%   \includegraphics[width=0.49\textwidth]{figures/BH_run2/b1/RadNMuons_BH_6500GeV_haloB1_20MeV.pdf}
%%   \includegraphics[width=0.49\textwidth]{figures/BH_run2/b2/RadNMuons_BH_6500GeV_haloB2_20MeV.pdf}
%% \end{center}
%% \vspace{-0.6cm}
%%  \caption{Comparison 
%%   \label{compRadNMuonsRun12}}
%% \end{figure}

\begin{figure}
  \centering
  \includegraphics[width=0.495\textwidth]{figures/xzTCT5s.pdf}
  \includegraphics[width=0.495\textwidth]{figures/HL/HL_hybrid_relaxedSett_yz_TCT5_tct5inrd_b1.pdf}
  \caption{
    \label{}}
\end{figure}


% ----------------------


%% \begin{figure}
%%   \centering
%% %  \includegraphics[width=0.495\textwidth]{figures/xzTCT5s.pdf}
%% %  \includegraphics[scale=0.49]{figures/HL/xy_z2760_-1.pdf}
%% %  \includegraphics[width=0.49\textwidth]{figures/4TeV/bs_20MeV/XYNMuons_BG_4TeV_20MeV_bs.pdf}
%%   \caption{
%%     \label{}}
%% \end{figure}

\begin{figure}
  \begin{center}
%    \includegraphics[width=0.495\textwidth]{figures/BH_run2/b1/XYNCharZoom_BH_6500GeV_haloB1_20MeV.pdf}
%    \includegraphics[width=0.495\textwidth]{figures/BH_run2/b2/XYNCharZoom_BH_6500GeV_haloB2_20MeV.pdf}  
    %\includegraphics[width=0.49\textwidth]{figures/4TeV/bs_20MeV/XYNCharZoom_BG_4TeV_20MeV_bs.pdf}
    %\includegraphics[width=0.49\textwidth]{figures/4TeV/bs_20MeV/XYNNeutronsE10_BG_4TeV_20MeV_bs.pdf}
    %\includegraphics[width=0.495\textwidth]{figures/4TeV/haloB1_20MeV/XYNCharZoom_BH_4TeV_B1_20MeV.pdf}
    \includegraphics[width=0.495\textwidth]{figures/HL/tct5inrd/XYNCharZoom_BH_HL_tct5inrdB1_20MeV.pdf}
    \includegraphics[width=0.495\textwidth]{figures/HL/tct5otrd/XYNCharZoom_BH_HL_tct5otrdB1_20MeV.pdf} 
    \includegraphics[width=0.495\textwidth]{figures/HL/tct5otrd/XYNNeutronsE10_BH_HL_tct5otrdB1_20MeV.pdf}
    \includegraphics[width=0.495\textwidth]{figures/HL/tct5inrd/XYNNeutronsE10_BH_HL_tct5inrdB1_20MeV.pdf}
    \includegraphics[width=0.495\textwidth]{figures/HL/tct5otrd/XYNPhotonsE10_BH_HL_tct5otrdB1_20MeV.pdf}
    \includegraphics[width=0.495\textwidth]{figures/HL/tct5inrd/XYNPhotonsE10_BH_HL_tct5inrdB1_20MeV.pdf}
    \includegraphics[width=0.495\textwidth]{figures/HL/tct5otrd/XYNPhotonsE10_BH_HL_tct5otrdB2_20MeV.pdf}
    \includegraphics[width=0.495\textwidth]{figures/HL/tct5inrd/XYNPhotonsE10_BH_HL_tct5inrdB2_20MeV.pdf}

%    \includegraphics[width=0.495\textwidth]{figures/HL/tct5inrd/XYNNeutronsE10_BH_HL_tct5inrdB2_20MeV.pdf}
\end{center}
\vspace{-0.6cm}
 \caption{unit is 1/cm$^{2}$/TCT hit. Top: charged particles.
  \label{fig:XYN}}
\end{figure}

\begin{figure}
  \begin{center}
    \includegraphics[width=0.495\textwidth]{figures/4TeV/bs_20MeV/XYNCharZoom_BG_4TeV_20MeV_bs.pdf}
    \includegraphics[width=0.495\textwidth]{figures/4TeV/bs_20MeV/XYNNeutronsE10_BG_4TeV_20MeV_bs.pdf}
    \includegraphics[width=0.495\textwidth]{figures/4TeV/haloB1_20MeV/XYNNeutronsE10_BH_4TeV_B1_20MeV.pdf}
    \includegraphics[width=0.495\textwidth]{figures/4TeV/haloB2_20MeV/XYNNeutronsE10_BH_4TeV_B2_20MeV.pdf}
    \includegraphics[width=0.495\textwidth]{figures/4TeV/haloB1_20MeV/XYNPhotonsZoom_BH_4TeV_B1_20MeV.pdf}
    \includegraphics[width=0.495\textwidth]{figures/4TeV/haloB2_20MeV/XYNPhotonsZoom_BH_4TeV_B2_20MeV.pdf}
    \includegraphics[width=0.495\textwidth]{figures/4TeV/haloB1_20MeV/XYNCharZoom_BH_4TeV_B1_20MeV.pdf}
    \includegraphics[width=0.495\textwidth]{figures/4TeV/haloB2_20MeV/XYNCharZoom_BH_4TeV_B2_20MeV.pdf}
    
    %% \includegraphics[width=0.495\textwidth]{figures/HL/tct5otrd/XYNNeutronsE10_BH_HL_tct5otrdB2_20MeV.pdf}
    %% \includegraphics[width=0.495\textwidth]{figures/HL/tct5inrd/XYNNeutronsE10_BH_HL_tct5inrdB1_20MeV.pdf}
    %% \includegraphics[width=0.495\textwidth]{figures/HL/tct5inrd/XYNCharZoom_BH_HL_tct5inrdB1_20MeV.pdf}
    %% \includegraphics[width=0.495\textwidth]{figures/HL/tct5inrd/XYNElecPosi_BH_HL_tct5inrdB1_20MeV.pdf}
    %% \includegraphics[width=0.495\textwidth]{figures/HL/tct5inrd/XYNProtonsE10_BH_HL_tct5inrdB1_20MeV.pdf}
    %% \includegraphics[width=0.495\textwidth]{figures/HL/tct5inrd/XYNPhotonsE10_BH_HL_tct5inrdB1_20MeV.pdf}
\end{center}
\vspace{-0.6cm}
 \caption{unit is 1/cm$^{2}$/TCT hit.
  \label{fig:XYNPho}}
\end{figure}

\newpage

        \bibliographystyle{abbrv}
        \bibliography{bibliography}

%% \begin{thebibliography}{99}
%% \bibitem{lastYear} R.~Kwee-Hinzmann et al., \textit{First Background Studies at IR1}, Proceedings IPAC14, 2014.

%% \end{thebibliography}
%
\end{document}
%
