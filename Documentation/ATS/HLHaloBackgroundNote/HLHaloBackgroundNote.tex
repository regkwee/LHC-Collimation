\documentclass{cernatsnote} % Specifies the document style.
%
%
% This texfile is a modified version from 
% http://frs.home.cern.ch/frs/Source/NL_MD_02.07.2011/NL-MD.tex
%
%
%
\usepackage{vmargin,times,graphicx,amsmath,amssymb,color} % ,draftcopy��use draftcopy for experiments
\usepackage{verbatim} % to allow for verbatim and comment
\usepackage{lineno} % to allow for linenumbers in the draft version
\usepackage{tabularx}
\usepackage{multirow}


\newcommand{\lumi}[1]{\ensuremath{{\mathcal{L}= 10^{#1}\,\mathrm{cm}^{-2}\,\mathrm{s}^{-1}}}}
\newcommand{\lumiapprox}[1]{\ensuremath{{\mathcal{L} \approx 10^{#1}\,\mathrm{cm}^{-2}\,\mathrm{s}^{-1}}}}
\newcommand{\lumif}[2]{\ensuremath{{\mathcal{L}= {#1} \times 10^{#2}\,\mathrm{cm}^{-2}\,\mathrm{s}^{-1}}}}
\newcommand{\ifb}{\mbox{fb$^{-1}$}}%  Inverse femtobarns.
\newcommand{\ipb}{\mbox{pb$^{-1}$}}%  Inverse picobarns.
\newcommand{\inb}{\mbox{nb$^{-1}$}}%  Inverse nanobarns.
\newcommand{\sph}{\mbox{$\mu$Sv/h}}%  mu Sv/h
\newcommand{\nseu}{$N_{\mathrm{SEU}}$}
\newcommand{\hehf}{\ensuremath{\Phi_{\mathrm{HEH}}}}
\newcommand{\thnf}{\ensuremath{\Phi_{\mathrm{thn}}}}
\newcommand{\neqf}{\ensuremath{\Phi_{\mathrm{neq}}}}
\newcommand{\mypercent}{\%}

% now optionally modify manually to fill better the page (drop at end
% ?)
\setmarginsrb{22mm}{10mm}{20mm}{10mm}{12pt}{11mm}{0pt}{11mm}
%
\title{Beam Halo induced Background Simulation Studies at IR1 for final HL-LHC Collimation Layout}
%
\author{R.~Kwee-Hinzmann, R.~Bruce, L.~S.~Esposito}
%
\date{\today}
\email{\small{Regina.Kwee@rhul.ac.uk}}
%
\makeindex
%
\documentlabel{CERN-ATS-Note-2015-XXX PERF}\keywords{HL-LHC, collimation, SixTrack, IR1, FLUKA, incoming beam, halo, TCT5}
%
% End of preamble and beginning of text.
\begin{document}
%
% Produces the title block.
\maketitle

\renewcommand{\textfraction}{0.1}       % the minimum fraction of a text page that must be devoted to text
\renewcommand{\floatpagefraction}{0.8}  % the minimum fraction of a float page that must be occupied by floats
\renewcommand{\topfraction}{1.}         % the maximum fraction float on top, usually .7

%
%\linenumbers
% ----------------------------------------------------------------------------------------------
\begin{abstract}
%
This note is contains the simulation results for beam halo induced background at IR1 in the HL-LHC scenario using the final HL-LHC collimation layout. SixTrack and Fluka simulations were performed to quantify the effect of the installation of additional tertiary collimators in cell 5 upstream the tertiary collimator pair that is already installed to protect the inner triplet magnets in IR1 and 5.
\end{abstract}
%
% ----------------------------------------------------------------------------------------------
\tableofcontents
\newpage
\newpage

% uncomment to only show section header
\addtocontents{toc}{\protect\setcounter{tocdepth}{1}}
%-------------------------------------------------------------------------------------------
\section{Introduction}

\section{Beam Halo Background}

\section{Simulation Setup}

\section{Simulation Results}

\subsection{SixTrack Results}
\begin{figure}
\begin{center}
\includegraphics[width=0.495\textwidth]{figures/compTCT5LINOUT_roundthin_B1_IR1IR5}
\includegraphics[width=0.495\textwidth]{figures/compTCT5LINOUT_flatthin_B1_IR1IR5}
\end{center}
\begin{picture} (0.,0.)
\setlength{\unitlength}{1.0cm}
\small{
    \put ( 4.,1.){(a)}
    \put ( 12.4,1.){(b)}
}
\end{picture}
\vspace{-0.6cm}
 \caption{Summary of hits load of the single TCTs for the cases TCT5 are out and in for round (a) and flat (b) beam optics.
  \label{compTCT5INOUT}}
\end{figure}


\subsection{Shower simulation results}

% ------------------------------------------------------------------------------------------
% 

\begin{figure}
\begin{center}
\includegraphics[width=0.495\textwidth]{figures/OrigYZMuons_BH_HL_tct5otrdB1_20MeV}
\includegraphics[width=0.495\textwidth]{figures/OrigYZMuonsE100_BH_HL_tct5otrdB1_20MeV}
\includegraphics[width=0.495\textwidth]{figures/OrigYZMuons_BH_HL_tct5inrdB1_20MeV}
\includegraphics[width=0.495\textwidth]{figures/OrigYZMuonsE100_BH_HL_tct5inrdB1_20MeV}
\end{center}
\begin{picture} (0.,0.)
\setlength{\unitlength}{1.0cm}
\small{
    \put ( 4.,7.35){(a)}
    \put ( 12.4,7.35){(b)}
    \put ( 4.,1.){(c)}
    \put ( 12.4,1.){(d)}
}
\end{picture}
\vspace{-0.6cm}
 \caption{Origin of muons for all energies (a,c) and for an energy above 100~GeV (b,d) in the y-z plane with TCT5 out (a,b) and TCT5 in (c,d).
  \label{OrigMuonE}}
\end{figure}

% ------------------------------------------------------------------------------------------
% comparisons

\begin{figure}
\begin{center}
\includegraphics[width=0.495\textwidth]{figures/ratioEkinAll}
\includegraphics[width=0.495\textwidth]{figures/ratioEkinMuons}
\end{center}
\begin{picture} (0.,0.)
\setlength{\unitlength}{1.0cm}
\small{
    \put ( 4.,1.){(a)}
    \put ( 12.4,1.){(b)}
}
\end{picture}
\vspace{-0.6cm}
 \caption{Energy distribution for all particles (a) and muons (b) at the interface plane.
  \label{compEkin}}
\end{figure}

\begin{figure}
\begin{center}
\includegraphics[width=0.495\textwidth]{figures/ratioEkinAllRInBP}
\includegraphics[width=0.495\textwidth]{figures/ratioEkinAllROutBP}
\end{center}
\begin{picture} (0.,0.)
\setlength{\unitlength}{1.0cm}
\small{
    \put ( 4.,1.){(a)}
    \put ( 12.4,1.){(b)}
}
\end{picture}
\vspace{-0.6cm}
 \caption{Energy distribution of all particles inside (a) and outside (b) the beampipe.
  \label{compEkinBP}}
\end{figure}



\begin{figure}
\begin{center}
\includegraphics[width=0.495\textwidth]{figures/ratioPhiNAll}
\includegraphics[width=0.495\textwidth]{figures/ratioPhiNMuons}
\end{center}
\begin{picture} (0.,0.)
\setlength{\unitlength}{1.0cm}
\small{
    \put ( 4.,1.){(a)}
    \put ( 12.4,1.){(b)}
}
\end{picture}
\vspace{-0.6cm}
 \caption{Azimuthal distribution of all particles (a) and muons (b) at the interface plane.
  \label{compPhiN}}
\end{figure}

\begin{figure}
\begin{center}
\includegraphics[width=0.495\textwidth]{figures/ratioPhiEnAll}
\includegraphics[width=0.495\textwidth]{figures/ratioPhiEnMuons}
\end{center}
\begin{picture} (0.,0.)
\setlength{\unitlength}{1.0cm}
\small{
    \put ( 4.,1.){(a)}
    \put ( 12.4,1.){(b)}
}
\end{picture}
\vspace{-0.6cm}
 \caption{Azimuthal energy distribution of all particles (a) and muons (b) at the interface plane.
  \label{compPhiEn}}
\end{figure}

\begin{figure}
\begin{center}
\includegraphics[width=0.495\textwidth]{figures/ratioRadNAll}
\includegraphics[width=0.495\textwidth]{figures/ratioRadNMuons}
\includegraphics[width=0.495\textwidth]{figures/ratioRadEnAll}
\includegraphics[width=0.495\textwidth]{figures/ratioRadEnMuons}
\end{center}
\begin{picture} (0.,0.)
\setlength{\unitlength}{1.0cm}
\small{
    \put ( 4.05,8.95){(a)}
    \put ( 12.45,8.95){(b)}
    \put ( 4.05,1.){(c)}
    \put ( 12.45,1.){(d)}
}
\end{picture}
\vspace{-0.6cm}
 \caption{Tranverse radial distribution of all particles (a) and muons (b) at the interface plane.
  \label{compRadN}}
\end{figure}

\section{Conclusion and outlook}

%-------------------------------------------------------------------------------------------
\section*{Acknowledgments}
%
The authors would like to thank F.~Cerutti and A.~Lechner of the FLUKA team (EN-STI-EET) for helpful discussions. 
%-------------------------------------------------------------------------------------------
\newpage
\clearpage
\appendix

\section{Appendix A}

\subsection{Loss map compendium}

% ---------------------------------------------------------------------------------------------------------------------
% fullring round/flat

\begin{figure}
\begin{center}
\vskip-12mm
\includegraphics[width=0.92\textwidth]{figures/coll_loss_H5_HL_TCT5LOUT_relaxColl_hHaloB1_roundthin_fullring}
\includegraphics[width=0.92\textwidth]{figures/coll_loss_H5_HL_TCT5LOUT_relaxColl_vHaloB1_roundthin_fullring}
\end{center}
\vspace{-0.3cm}
 \caption{Loss maps for horizontal (top) and vertical (bottom) B1 halo using round optics with TCT4s only.
  \label{fullring_roundB1_TCT5LOUT }}
\end{figure}

\begin{figure}
\begin{center}
\vskip-12mm
\includegraphics[width=0.92\textwidth]{figures/coll_loss_H5_HL_TCT5IN_relaxColl_hHaloB1_roundthin_fullring}
\includegraphics[width=0.92\textwidth]{figures/coll_loss_H5_HL_TCT5IN_relaxColl_vHaloB1_roundthin_fullring}
\end{center}
\vspace{-0.3cm}
 \caption{Loss maps for horizontal (top) and vertical (bottom) B1 halo using round optics with TCT4s and TCT5s.
  \label{fullring_roundB1_TCT5IN}}
\end{figure}


\begin{figure}
\begin{center}
\vskip-12mm
\includegraphics[width=0.92\textwidth]{figures/coll_loss_H5_HL_TCT5LOUT_relaxColl_hHaloB1_flatthin_fullring}
\includegraphics[width=0.92\textwidth]{figures/coll_loss_H5_HL_TCT5LOUT_relaxColl_vHaloB1_flatthin_fullring}
\end{center}
\vspace{-0.3cm}
 \caption{Loss maps for horizontal (top) and vertical (bottom) B1 halo using flat optics with TCT4s only.
  \label{fullring_flatB1_TCT5LOUT}}
\end{figure}

\begin{figure}
\begin{center}
\vskip-12mm
\includegraphics[width=0.92\textwidth]{figures/coll_loss_H5_HL_TCT5IN_relaxColl_hHaloB1_flatthin_fullring}
\includegraphics[width=0.92\textwidth]{figures/coll_loss_H5_HL_TCT5IN_relaxColl_vHaloB1_flatthin_fullring}
\end{center}
\vspace{-0.3cm}
 \caption{Loss maps for horizontal (top) and vertical (bottom) B1 halo using flat optics with TCT4s and TCT5s.
  \label{fullring_flatB1_TCT5IN}}
\end{figure}

% ---------------------------------------------------------------------------------------------------------------------
% zooms

\begin{figure}
\begin{center}
\vskip-12mm
\includegraphics[width=0.48\textwidth]{figures/coll_loss_H5_HL_TCT5LOUT_relaxColl_hHaloB1_roundthin_IR1}
\includegraphics[width=0.48\textwidth]{figures/coll_loss_H5_HL_TCT5LOUT_relaxColl_vHaloB1_roundthin_IR1}
\includegraphics[width=0.48\textwidth]{figures/coll_loss_H5_HL_TCT5IN_relaxColl_hHaloB1_roundthin_IR1}
\includegraphics[width=0.48\textwidth]{figures/coll_loss_H5_HL_TCT5IN_relaxColl_vHaloB1_roundthin_IR1}
\end{center}
\begin{picture} (0.,0.)
\setlength{\unitlength}{1.0cm}
\small{
    \put ( 4.,7.35){(a)}
    \put ( 12.4,7.35){(b)}
    \put ( 4.,1.){(c)}
    \put ( 12.4,1.){(d)}
}
\end{picture}
\vspace{-0.3cm}
 \caption{Zoom into IR1 for horizontal (top) and vertical (bottom) B1 halo using round optics with TCT4s only and both.
  \label{IR1_roundB1_TCT5LOUT }}
\end{figure}

\begin{figure}
\begin{center}
\vskip-12mm
\includegraphics[width=0.48\textwidth]{figures/coll_loss_H5_HL_TCT5LOUT_relaxColl_hHaloB1_roundthin_IR5}
\includegraphics[width=0.48\textwidth]{figures/coll_loss_H5_HL_TCT5LOUT_relaxColl_vHaloB1_roundthin_IR5}
\includegraphics[width=0.48\textwidth]{figures/coll_loss_H5_HL_TCT5IN_relaxColl_hHaloB1_roundthin_IR5}
\includegraphics[width=0.48\textwidth]{figures/coll_loss_H5_HL_TCT5IN_relaxColl_vHaloB1_roundthin_IR5}
\end{center}
\vspace{-0.3cm}
 \caption{Zoom into IR5 for horizontal (top) and vertical (bottom) B1 halo using round optics with TCT4s only.
  \label{IR5_roundB1_TCT5LOUT }}
\end{figure}

\begin{figure}
\begin{center}
\vskip-12mm
\includegraphics[width=0.48\textwidth]{figures/coll_loss_H5_HL_TCT5LOUT_relaxColl_hHaloB1_flatthin_IR1}
\includegraphics[width=0.48\textwidth]{figures/coll_loss_H5_HL_TCT5LOUT_relaxColl_vHaloB1_flatthin_IR1}
\includegraphics[width=0.48\textwidth]{figures/coll_loss_H5_HL_TCT5IN_relaxColl_hHaloB1_flatthin_IR1}
\includegraphics[width=0.48\textwidth]{figures/coll_loss_H5_HL_TCT5IN_relaxColl_vHaloB1_flatthin_IR1}
\end{center}
\begin{picture} (0.,0.)
\setlength{\unitlength}{1.0cm}
\small{
    \put ( 4.,7.35){(a)}
    \put ( 12.4,7.35){(b)}
    \put ( 4.,1.){(c)}
    \put ( 12.4,1.){(d)}
}
\end{picture}
\vspace{-0.3cm}
 \caption{Zoom into IR1 for horizontal (top) and vertical (bottom) B1 halo using flat optics with TCT4s and TCT5s.
  \label{IR1_flatB1_TCT5IN}}
\end{figure}


\begin{figure}
\begin{center}
\vskip-12mm
\includegraphics[width=0.48\textwidth]{figures/coll_loss_H5_HL_TCT5LOUT_relaxColl_hHaloB1_flatthin_IR5}
\includegraphics[width=0.48\textwidth]{figures/coll_loss_H5_HL_TCT5LOUT_relaxColl_vHaloB1_flatthin_IR5}
\includegraphics[width=0.48\textwidth]{figures/coll_loss_H5_HL_TCT5IN_relaxColl_hHaloB1_flatthin_IR5}
\includegraphics[width=0.48\textwidth]{figures/coll_loss_H5_HL_TCT5IN_relaxColl_vHaloB1_flatthin_IR5}
\end{center}
\begin{picture} (0.,0.)
\setlength{\unitlength}{1.0cm}
\small{
    \put ( 4.,7.35){(a)}
    \put ( 12.4,7.35){(b)}
    \put ( 4.,1.){(c)}
    \put ( 12.4,1.){(d)}
}
\end{picture}
\vspace{-0.3cm}
 \caption{Zoom into IR5 for horizontal (top) and vertical (bottom) B1 halo using flat optics with TCT4s only.
  \label{IR5_flatB1_TCT5LOUT}}
\end{figure}

% ---------------------------------------------------------------------------------------------------------------------
% s-optics

\begin{figure}
\begin{center}
\vskip-12mm
\includegraphics[width=0.92\textwidth]{figures/coll_loss_H5_HL_TCT5IN_relaxColl_hHaloB1_sroundthin_fullring}
\includegraphics[width=0.92\textwidth]{figures/coll_loss_H5_HL_TCT5IN_relaxColl_hHaloB1_sflatthin_fullring}
\end{center}
\vspace{-0.3cm}
 \caption{Loss maps for horizontal B1 halo using s-round (top) and s-flat optics with TCT4s and TCT5s.
  \label{fullring_sB1_TCT5IN}}
\end{figure}

\newpage
% ---------------------------------------------------------------------------------------------------------------------
\section{Appendix B}
\newpage

\begin{thebibliography}{99}
\bibitem{lastYear} R.~Kwee-Hinzmann et al., \textit{First Background Studies at IR1}, Proceedings IPAC14, 2014.

\end{thebibliography}
%
\end{document}
%
