\documentclass{cernatsnote} % Specifies the document style.
%
%
% This texfile is a modified version from 
% http://frs.home.cern.ch/frs/Source/NL_MD_02.07.2011/NL-MD.tex
%
%
%
\usepackage{vmargin,times,graphicx,amsmath,amssymb,color} % ,draftcopy��use draftcopy for experiments
\usepackage{verbatim} % to allow for verbatim and comment
\usepackage{lineno} % to allow for linenumbers in the draft version
\usepackage{tabularx}
\usepackage{multirow}


\newcommand{\lumi}[1]{\ensuremath{{\mathcal{L}= 10^{#1}\,\mathrm{cm}^{-2}\,\mathrm{s}^{-1}}}}
\newcommand{\lumiapprox}[1]{\ensuremath{{\mathcal{L} \approx 10^{#1}\,\mathrm{cm}^{-2}\,\mathrm{s}^{-1}}}}
\newcommand{\lumif}[2]{\ensuremath{{\mathcal{L}= {#1} \times 10^{#2}\,\mathrm{cm}^{-2}\,\mathrm{s}^{-1}}}}
\newcommand{\ifb}{\mbox{fb$^{-1}$}}%  Inverse femtobarns.
\newcommand{\ipb}{\mbox{pb$^{-1}$}}%  Inverse picobarns.
\newcommand{\inb}{\mbox{nb$^{-1}$}}%  Inverse nanobarns.
\newcommand{\sph}{\mbox{$\mu$Sv/h}}%  mu Sv/h
\newcommand{\nseu}{$N_{\mathrm{SEU}}$}
\newcommand{\hehf}{\ensuremath{\Phi_{\mathrm{HEH}}}}
\newcommand{\thnf}{\ensuremath{\Phi_{\mathrm{thn}}}}
\newcommand{\neqf}{\ensuremath{\Phi_{\mathrm{neq}}}}
\newcommand{\mypercent}{\%}

% now optionally modify manually to fill better the page (drop at end
% ?)
\setmarginsrb{22mm}{10mm}{20mm}{10mm}{12pt}{11mm}{0pt}{11mm}
%
\title{Beam Halo induced Background Simulation Studies at IR1 for final HL-LHC Collimation Layout}
%
\author{R.~Kwee-Hinzmann, R.~Bruce, L.~S.~Esposito}
%
\date{\today}
\email{\small{Regina.Kwee@rhul.ac.uk}}
%
\makeindex
%
\documentlabel{CERN-ATS-Note-2015-XXX PERF}\keywords{HL-LHC, collimation, SixTrack, IR1, FLUKA, incoming beam, halo, TCT5}
%
% End of preamble and beginning of text.
\begin{document}
%
% Produces the title block.
\maketitle

\renewcommand{\textfraction}{0.1}       % the minimum fraction of a text page that must be devoted to text
\renewcommand{\floatpagefraction}{0.8}  % the minimum fraction of a float page that must be occupied by floats
\renewcommand{\topfraction}{1.}         % the maximum fraction float on top, usually .7

%
%\linenumbers
% ----------------------------------------------------------------------------------------------
\begin{abstract}
%
This note is contains the simulation results for beam halo induced background at IR1 in the HL-LHC scenario using the final HL-LHC collimation layout. SixTrack and Fluka simulations were performed to quantify the effect of the installation of additional tertiary collimators in cell 5 upstream the tertiary collimator pair that is already installed to protect the inner triplet magnets in IR1 and 5.
\end{abstract}
%
% ----------------------------------------------------------------------------------------------
\tableofcontents
\newpage
\newpage

% uncomment to only show section header
\addtocontents{toc}{\protect\setcounter{tocdepth}{1}}
%-------------------------------------------------------------------------------------------
\section{Introduction}

\section{Beam Halo Background}

\section{Simulation Setup}

some test ~\cite{SixTrackRef,flukaRef1,claudiasThesis,flukaRef2,opticsWebRef}
\section{Simulation Results}

\subsection{SixTrack Results}
\begin{figure}
\begin{center}
\includegraphics[width=0.495\textwidth]{figures/compTCT5LINOUT_roundthin_B1_IR1IR5}
\includegraphics[width=0.495\textwidth]{figures/compTCT5LINOUT_flatthin_B1_IR1IR5}
\end{center}
\begin{picture} (0.,0.)
\setlength{\unitlength}{1.0cm}
\small{
    \put ( 4.,1.){(a)}
    \put ( 12.4,1.){(b)}
}
\end{picture}
\vspace{-0.6cm}
 \caption{Summary of hits load of the single TCTs for the cases TCT5 are out and in for round (a) and flat (b) beam optics.
  \label{compTCT5INOUT}}
\end{figure}


\subsection{Shower simulation results}

% ------------------------------------------------------------------------------------------
% 

\begin{figure}
\begin{center}
\includegraphics[width=0.495\textwidth]{figures/OrigYZMuons_BH_HL_tct5otrdB1_20MeV}
\includegraphics[width=0.495\textwidth]{figures/OrigYZMuonsE100_BH_HL_tct5otrdB1_20MeV}
\includegraphics[width=0.495\textwidth]{figures/OrigYZMuons_BH_HL_tct5inrdB1_20MeV}
\includegraphics[width=0.495\textwidth]{figures/OrigYZMuonsE100_BH_HL_tct5inrdB1_20MeV}
\end{center}
\begin{picture} (0.,0.)
\setlength{\unitlength}{1.0cm}
\small{
    \put ( 4.,7.35){(a)}
    \put ( 12.4,7.35){(b)}
    \put ( 4.,1.){(c)}
    \put ( 12.4,1.){(d)}
}
\end{picture}
\vspace{-0.6cm}
 \caption{Origin of muons for all energies (a,c) and for an energy above 100~GeV (b,d) in the y-z plane with TCT5 out (a,b) and TCT5 in (c,d).
  \label{OrigMuonE}}
\end{figure}

% ------------------------------------------------------------------------------------------
% comparisons

\begin{figure}
\begin{center}
\includegraphics[width=0.495\textwidth]{figures/ratioEkinAll}
\includegraphics[width=0.495\textwidth]{figures/ratioEkinMuons}
\end{center}
\begin{picture} (0.,0.)
\setlength{\unitlength}{1.0cm}
\small{
    \put ( 4.,1.){(a)}
    \put ( 12.4,1.){(b)}
}
\end{picture}
\vspace{-0.6cm}
 \caption{Energy distribution for all particles (a) and muons (b) at the interface plane.
  \label{compEkin}}
\end{figure}

\begin{figure}
\begin{center}
\includegraphics[width=0.495\textwidth]{figures/ratioEkinAllRInBP}
\includegraphics[width=0.495\textwidth]{figures/ratioEkinAllROutBP}
\end{center}
\begin{picture} (0.,0.)
\setlength{\unitlength}{1.0cm}
\small{
    \put ( 4.,1.){(a)}
    \put ( 12.4,1.){(b)}
}
\end{picture}
\vspace{-0.6cm}
 \caption{Energy distribution of all particles inside (a) and outside (b) the beampipe.
  \label{compEkinBP}}
\end{figure}



\begin{figure}
\begin{center}
\includegraphics[width=0.495\textwidth]{figures/ratioPhiNAll}
\includegraphics[width=0.495\textwidth]{figures/ratioPhiNMuons}
\end{center}
\begin{picture} (0.,0.)
\setlength{\unitlength}{1.0cm}
\small{
    \put ( 4.,1.){(a)}
    \put ( 12.4,1.){(b)}
}
\end{picture}
\vspace{-0.6cm}
 \caption{Azimuthal distribution of all particles (a) and muons (b) at the interface plane.
  \label{compPhiN}}
\end{figure}

\begin{figure}
\begin{center}
\includegraphics[width=0.495\textwidth]{figures/ratioPhiEnAll}
\includegraphics[width=0.495\textwidth]{figures/ratioPhiEnMuons}
\end{center}
\begin{picture} (0.,0.)
\setlength{\unitlength}{1.0cm}
\small{
    \put ( 4.,1.){(a)}
    \put ( 12.4,1.){(b)}
}
\end{picture}
\vspace{-0.6cm}
 \caption{Azimuthal energy distribution of all particles (a) and muons (b) at the interface plane.
  \label{compPhiEn}}
\end{figure}

\begin{figure}
\begin{center}
\includegraphics[width=0.495\textwidth]{figures/ratioRadNAll}
\includegraphics[width=0.495\textwidth]{figures/ratioRadNMuons}
\includegraphics[width=0.495\textwidth]{figures/ratioRadEnAll}
\includegraphics[width=0.495\textwidth]{figures/ratioRadEnMuons}
\end{center}
\begin{picture} (0.,0.)
\setlength{\unitlength}{1.0cm}
\small{
    \put ( 4.05,8.95){(a)}
    \put ( 12.45,8.95){(b)}
    \put ( 4.05,1.){(c)}
    \put ( 12.45,1.){(d)}
}
\end{picture}
\vspace{-0.6cm}
 \caption{Tranverse radial distribution of all particles (a) and muons (b) at the interface plane.
  \label{compRadN}}
\end{figure}

\section{Conclusion and outlook}

%-------------------------------------------------------------------------------------------
\section*{Acknowledgments}
%
The authors would like to thank F.~Cerutti and A.~Lechner of the FLUKA team (EN-STI-EET) for helpful discussions. 
%-------------------------------------------------------------------------------------------
\newpage
\clearpage
\appendix

%\section{HL loss maps \label{lossmapszooms}}
% ---------------------------------------------------------------------------------------------------------------------
% fullring round/flat

We show here zooms of the full loss maps of round and flat beam optics for HL-LHC and \twosigmaret~settings per simulation case, for a horizontal and vertical halo distribution in SixTrack. First we validate the settings by looking at IR7 losses, comparing nominal and \twosigmaret~settings, then we analyse the zooms for IR1 and IR5 to evaluate the affect of additional collimators in the experimental IRs. The beam direction in all figures, IR7 zooms Fig.~\ref{IR7_zooms}, IR1 round beam 1 comparisons Fig.~\ref{IR1_roundB1}, IR5 round beam 1 comparisons in Fig.~\ref{IR5_roundB1} and IR1/IR5 for flat beam 1 in Fig.~\ref{IR1_flatB1}/\ref{IR5_flatB1}, is from left to right. 

\subsubsection{Losses at IR1/IR5 tertiary collimators in HL--LHC}

A zoom into loss locations in the experimental IRs of ATLAS and CMS are shown in Fig.~\ref{IR15_roundB1_nomSett} for nominal collimator settings. The similar set of zooms are shown and discussed for the \twosigmaret~settings in the Appendix~\ref{lossmapszooms}. The beam direction is from left to right. The two black bars at upstream of the IP are the losses on the pair of tertiary collimators (TCT4s and TCT5s), while the black bars downstream are losses on TCLs, debris collimators. These losses are normalised to total number of lost particles. One can see, IR5 losses are generally lower than in IR1, an expected feature known from Run I and II. A more direct comparison of the losses is made in Fig.~\ref{compTCT5INOUT}. The losses are normalised to the number of simulated primary per simulation case, then the sum per collimator is shown, i.e. as example the bin of a specific TCTH is set to $\big(\frac{\mathrm{hits\,on\,TCTH}}{\#\mathrm{primary}}\big)_{\mathrm{h}} + \big(\frac{\mathrm{hits\,on\,TCTH}}{\#\mathrm{primary}}\big)_{\mathrm{v}}$. The simulation cases shown are in Fig.~\ref{compTCT5INOUT}~(a) for round, and Fig.~\ref{compTCT5INOUT}~(b) for flat beams.

\begin{figure} [!htb]
\begin{center}

\includegraphics[width=0.48\textwidth]{figures/lossmaps/coll_loss_H5_HL_nomSett_hHalo_b1_IR1}
\includegraphics[width=0.48\textwidth]{figures/lossmaps/coll_loss_H5_HL_nomSett_vHalo_b1_IR1}
\includegraphics[width=0.48\textwidth]{figures/lossmaps/coll_loss_H5_HL_nomSett_hHalo_b1_IR5}
\includegraphics[width=0.48\textwidth]{figures/lossmaps/coll_loss_H5_HL_nomSett_vHalo_b1_IR5}
\end{center}
\vspace{-0.3cm}
 \caption{Zoom into IR1 (top) and IR5 (bottom, IP5 is at 133,300~m) when the TCT5s were inserted using round optics. Horizontal beam 1 is on the left, vertical beam 1 on the right.
  \label{IR15_roundB1_nomSett}}
\end{figure}



%% \begin{figure}
%% \begin{center}
%% \vskip-12mm
%% \includegraphics[width=0.92\textwidth]{figures/lossmaps/coll_loss_H5_HL_TCT5LOUT_relaxColl_hHaloB1_roundthin_fullring}
%% \includegraphics[width=0.92\textwidth]{figures/lossmaps/coll_loss_H5_HL_TCT5LOUT_relaxColl_vHaloB1_roundthin_fullring}
%% \end{center}
%% \vspace{-0.3cm}
%%  \caption{Loss maps for horizontal (top) and vertical (bottom) B1 halo using round optics with TCT4s only.
%%   \label{fullring_roundB1_TCT5LOUT }}
%% \end{figure}

%% \begin{figure}
%% \begin{center}
%% \vskip-12mm
%% \includegraphics[width=0.92\textwidth]{figures/lossmaps/coll_loss_H5_HL_TCT5IN_relaxColl_hHaloB1_roundthin_fullring}
%% \includegraphics[width=0.92\textwidth]{figures/lossmaps/coll_loss_H5_HL_TCT5IN_relaxColl_vHaloB1_roundthin_fullring}
%% \end{center}
%% \vspace{-0.3cm}
%%  \caption{Loss maps for horizontal (top) and vertical (bottom) B1 halo using round optics with TCT4s and TCT5s.
%%   \label{fullring_roundB1_TCT5IN}}
%% \end{figure}


%% \begin{figure}
%% \begin{center}
%% \vskip-12mm
%% \includegraphics[width=0.92\textwidth]{figures/lossmaps/coll_loss_H5_HL_TCT5LOUT_relaxColl_hHaloB1_flatthin_fullring}
%% \includegraphics[width=0.92\textwidth]{figures/lossmaps/coll_loss_H5_HL_TCT5LOUT_relaxColl_vHaloB1_flatthin_fullring}
%% \end{center}
%% \vspace{-0.3cm}
%%  \caption{Loss maps for horizontal (top) and vertical (bottom) B1 halo using flat optics with TCT4s only.
%%   \label{fullring_flatB1_TCT5LOUT}}
%% \end{figure}

%% \begin{figure}
%% \begin{center}
%% \vskip-12mm
%% \includegraphics[width=0.92\textwidth]{figures/lossmaps/coll_loss_H5_HL_TCT5IN_relaxColl_hHaloB1_flatthin_fullring}
%% \includegraphics[width=0.92\textwidth]{figures/lossmaps/coll_loss_H5_HL_TCT5IN_relaxColl_vHaloB1_flatthin_fullring}
%% \end{center}
%% \vspace{-0.3cm}
%%  \caption{Loss maps for horizontal (top) and vertical (bottom) B1 halo using flat optics with TCT4s and TCT5s.
%%   \label{fullring_flatB1_TCT5IN}}
%% \end{figure}

% ---------------------------------------------------------------------------------------------------------------------
% IR7 zooms

\begin{figure}[!htb]
\begin{center}
%\vskip-12mm
\includegraphics[width=0.48\textwidth]{figures/lossmaps/coll_loss_H5_HL_nomSett_hHalo_b1_IR7}
\includegraphics[width=0.48\textwidth]{figures/lossmaps/coll_loss_H5_HL_nomSett_vHalo_b1_IR7}
\includegraphics[width=0.48\textwidth]{figures/lossmaps/coll_loss_H5_HL_TCT5IN_relaxColl_hHaloB1_roundthin_IR7}
\includegraphics[width=0.48\textwidth]{figures/lossmaps/coll_loss_H5_HL_TCT5IN_relaxColl_vHaloB1_roundthin_IR7}
\includegraphics[width=0.48\textwidth]{figures/lossmaps/coll_loss_H5_HL_TCT5IN_relaxColl_hHaloB1_flatthin_IR7}
\includegraphics[width=0.48\textwidth]{figures/lossmaps/coll_loss_H5_HL_TCT5IN_relaxColl_vHaloB1_flatthin_IR7}
\end{center}
%% \begin{picture} (0.,0.)
%% \setlength{\unitlength}{1.0cm}
%% \small{
%%     \put ( 4.,7.35){(a)}
%%     \put ( 12.4,7.35){(b)}
%%     \put ( 4.,1.){(c)}
%%     \put ( 12.4,1.){(d)}
%% }
%% \end{picture}
\vspace{-0.3cm}
 \caption{Zoom into IR7 for round with nominal (top) and \twosigmaret~settings (middle), and flat optics and \twosigmaret~settings (bottom). Horizontal beam 1 is on the left, vertical beam 1 on the right.
  \label{IR7_zooms}}
\end{figure}

The view to IR7 in Fig.~\ref{IR7_zooms} shows the cleaning inefficiency, leaking protons from IR7 primaries and secondaries, is most critical in the two prominent cold region blocks. They usually serve as a benchmark region for collimator settings and optics as the maximum energy is deposited in these blocks. The maximum cold loss is at 10$^{-4}$ for \twosigmaret~round and flat beams, and a factor 2 lower with nominal settings.

\begin{figure}[!tb]
\begin{center}
%\vskip-12mm
\includegraphics[width=0.48\textwidth]{figures/lossmaps/coll_loss_H5_HL_TCT5LOUT_relaxColl_hHaloB1_roundthin_IR1}
\includegraphics[width=0.48\textwidth]{figures/lossmaps/coll_loss_H5_HL_TCT5LOUT_relaxColl_vHaloB1_roundthin_IR1}
\includegraphics[width=0.48\textwidth]{figures/lossmaps/coll_loss_H5_HL_TCT5IN_relaxColl_hHaloB1_roundthin_IR1}
\includegraphics[width=0.48\textwidth]{figures/lossmaps/coll_loss_H5_HL_TCT5IN_relaxColl_vHaloB1_roundthin_IR1}
\end{center}
%% \begin{picture} (0.,0.)
%% \setlength{\unitlength}{1.0cm}
%% \small{
%%     \put ( 4.,7.35){(a)}
%%     \put ( 12.4,7.35){(b)}
%%     \put ( 4.,1.){(c)}
%%     \put ( 12.4,1.){(d)}
%% }
%% \end{picture}
\vspace{-0.3cm}
 \caption{Zoom into IR1 (IP1 is at 0~m) for TCT5s out (top) and TCT5s in (bottom) B1 halo using round optics. Horizontal beam 1 is on the left, vertical beam 1 on the right.
  \label{IR1_roundB1}}
\end{figure}

The losses in IR1 with and without additional TCT5s are shown in Fig.~\ref{IR1_roundB1} for the more realistic \twosigmaret~setting. The nominal settings are in Fig.~\ref{IR5_roundB1_nomSett}
% -------------------------------------------------------------------------------------------------------------------

\begin{figure}
\begin{center}
\vskip-12mm
\includegraphics[width=0.48\textwidth]{figures/lossmaps/coll_loss_H5_HL_TCT5LOUT_relaxColl_hHaloB1_roundthin_IR5}
\includegraphics[width=0.48\textwidth]{figures/lossmaps/coll_loss_H5_HL_TCT5LOUT_relaxColl_vHaloB1_roundthin_IR5}
\includegraphics[width=0.48\textwidth]{figures/lossmaps/coll_loss_H5_HL_TCT5IN_relaxColl_hHaloB1_roundthin_IR5}
\includegraphics[width=0.48\textwidth]{figures/lossmaps/coll_loss_H5_HL_TCT5IN_relaxColl_vHaloB1_roundthin_IR5}
\end{center}
\vspace{-0.3cm}
 \caption{Zoom into IR5 (IP5 is at 133,300~m) for TCT5s out (top) and TCT5s in (bottom) B1 halo using round optics. Horizontal beam 1 is on the left, vertical beam 1 on the right.
  \label{IR5_roundB1}}
\end{figure}

\begin{figure}
\begin{center}
\vskip-12mm
\includegraphics[width=0.48\textwidth]{figures/lossmaps/coll_loss_H5_HL_TCT5LOUT_relaxColl_hHaloB1_flatthin_IR1}
\includegraphics[width=0.48\textwidth]{figures/lossmaps/coll_loss_H5_HL_TCT5LOUT_relaxColl_vHaloB1_flatthin_IR1}
\includegraphics[width=0.48\textwidth]{figures/lossmaps/coll_loss_H5_HL_TCT5IN_relaxColl_hHaloB1_flatthin_IR1}
\includegraphics[width=0.48\textwidth]{figures/lossmaps/coll_loss_H5_HL_TCT5IN_relaxColl_vHaloB1_flatthin_IR1}
\end{center}
%% \begin{picture} (0.,0.)
%% \setlength{\unitlength}{1.0cm}
%% \small{
%%     \put ( 4.,7.35){(a)}
%%     \put ( 12.4,7.35){(b)}
%%     \put ( 4.,1.){(c)}
%%     \put ( 12.4,1.){(d)}
%% }
%% \end{picture}
\vspace{-0.3cm}
 \caption{Zoom into IR1 (IP1 is at 0~m) for TCT5s out (top) and TCT5s in (bottom) B1 halo using flat optics. Horizontal beam 1 is on the left, vertical beam 1 on the right.
  \label{IR1_flatB1}}
\end{figure}


\begin{figure}
\begin{center}
\vskip-12mm
\includegraphics[width=0.48\textwidth]{figures/lossmaps/coll_loss_H5_HL_TCT5LOUT_relaxColl_hHaloB1_flatthin_IR5}
\includegraphics[width=0.48\textwidth]{figures/lossmaps/coll_loss_H5_HL_TCT5LOUT_relaxColl_vHaloB1_flatthin_IR5}
\includegraphics[width=0.48\textwidth]{figures/lossmaps/coll_loss_H5_HL_TCT5IN_relaxColl_hHaloB1_flatthin_IR5}
\includegraphics[width=0.48\textwidth]{figures/lossmaps/coll_loss_H5_HL_TCT5IN_relaxColl_vHaloB1_flatthin_IR5}
\end{center}
%% \begin{picture} (0.,0.)
%% \setlength{\unitlength}{1.0cm}
%% \small{
%%     \put ( 4.,7.35){(a)}
%%     \put ( 12.4,7.35){(b)}
%%     \put ( 4.,1.){(c)}
%%     \put ( 12.4,1.){(d)}
%% }
%% \end{picture}
\vspace{-0.3cm}
 \caption{Zoom into IR5 (IP5 is at 133,300~m) with TCT5s out (top) and TCT5s in (bottom) B1 halo using flat optics. Horizontal beam 1 is on the left, vertical beam 1 on the right.
  \label{IR5_flatB1}}
\end{figure}

\newpage
% ---------------------------------------------------------------------------------------------------------------------
\section{Appendix B}
\newpage

        \bibliographystyle{abbrv}
        \bibliography{bibliography}

%% \begin{thebibliography}{99}
%% \bibitem{lastYear} R.~Kwee-Hinzmann et al., \textit{First Background Studies at IR1}, Proceedings IPAC14, 2014.

%% \end{thebibliography}
%
\end{document}
%
