% This is cernatsreport.tex in text format, as of 17 September 2012
% adapted from abreport.tex in text format, as of 27 March 2003.

% This file gives you a basic template for writing an CERN ATS Sector report.

% This file MUST be studied in conjunction with:
%    LaTeX: A Document Preparation System  by Leslie Lamport
%    Addison-Wesley, Reading, Massachussetts,  .
%

%------------------------------------------------------------------------
% This is a comment since TeX ignores everything to the right of a '%'.
% This file is much longer than it need be only because there are lots
% of comments to explain it.

% Now comes the first LaTeX command in this file.

\documentclass{cernatsreport}    % Specifies the document style.

	% Now follows the "preamble" with some declarations.
	% No text should be produced until after the \begin{document}

\documentlabel{CERN-ATS-2012-XXX}       %Declares label for this document. Modify as required. 

%\typist{RA/pm}     % optional command to indicate who typed the report.
                    % This command was useful in the good old days.

\title{Coupling of Beam Polarization  \\ % how to break a long title
       to Geomagnetic Currents}         % Declares the document's title.

\author{Ren\'e Q. \AE\"o\c ca\ss\ Jr.\thanks{On leave of absence from
                                             }
       }            %Declares author's name and optional affiliation etc.

\date{3 August 2012}  % The date, in European format (the default LaTeX \today should not be used as it formats in American style. 

% Now we give the text of the abstract as another declaration; this is
% different from other LaTeX styles in which the abstract is given as
% the abstract environment.

\abstract{This is a sample input file illustrating the
          use of \LaTeX. \LaTeX\ is a widely-available
          interface which makes \TeX\ easier to use and
          ensures consistency of document formats.
          Comparing this file with the output it
          generates could show you how to produce a simple ATS 
          Sector Report of your own.
          To learn more, however, you {\it  must read the \LaTeX\ book or other documentation.}
%
% The following sort of thing can be conveniently included
% as part of the abstract.
% Don't leave any blank lines within the abstract declaration.
%
          \begin{center}\textsl{ Invited paper at the\\
                       Intercontinental Workshop on Irrelevant Effects,\\
                       Isola de Eigg, 25--26 December 2012}
          \end{center}
         }  %%% this is the end of the abstract declaration

% N.B. Nothing has been printed so far, we've only made declarations.

\begin{document}           % End of preamble and beginning of text.



\maketitle                 % Produces the title page.

% Some or all of the following may be useful in a long report
     \tableofcontents
     \listoffigures
     \listoftables
     \newpage


\section{Ordinary Text}        % Produces section heading.  Lower-level
                               % sections are begun with similar
                               % \subsection and \subsubsection commands.
                               % For even lower levels there are
                               % \paragraph and \subparagraph

The ends  of words and sentences are marked
  by   spaces. It  doesn't matter how many
spaces    you type; one is as good as 100.  The
end of   a line counts as a space.

One   or more   blank lines denote the  end
of  a paragraph.

Since any number of consecutive spaces are treated like a single
one, the formatting of the input file makes no difference to
      \TeX,         % The \TeX command generates the TeX logo.
but it makes a difference to you.
When you use
      \LaTeX,       % The \LaTeX command generates the LaTeX logo.
making your input file as easy to read as possible
will be a great help as you write your document and when you
change it.  This sample file shows how you can add comments to
your own input file.

Because printing is different from typewriting, there are a
number of things that you have to do differently when preparing
an input file than if you were just typing the document directly.
Quotation marks like
       ``this''
have to be handled specially, as do quotes within quotes:
       ``\,`this'             % \, separates the double and single quote.
        is what I just
        wrote, not  `that'\,''.

Dashes come in three sizes: an
       intra-word
dash, a medium dash for number ranges like
       1--2,
and a punctuation
       dash---like
this.

A sentence-ending space should be larger than the space between words
within a sentence.  You sometimes have to type special commands in
conjunction with punctuation characters to get this right, as in the
following sentence.
       Gnats, gnus, etc.\    % `\ ' makes an inter-word space.
       all begin with G\@.   % \@ marks end-of-sentence punctuation.
You should check the spaces after periods when reading your output to
make sure you haven't forgotten any special cases.
Generating an ellipsis
       \ldots\    % `\ ' needed because TeX ignores spaces after
                  % command names like \ldots made from \ + letters.
                  %
                  % Note how a `%' character causes TeX to ignore the
                  % end of the input line, so these blank lines do not
                  % start a new paragraph.
with the right spacing around the periods
requires a special  command.

\TeX\ interprets some common characters as commands, so you must type
special commands to generate them.  These characters include the
following:
       \$ \& \% \# \{ and \}.

In printing, text is emphasized by using an
      {\em italic}  % The \/ command produces the tiny extra space that
                      % should be added between a slanted and a following
                      % unslanted letter.
type style.

In printing, text is emphasized by using an
      \emph{italic}  %  LaTeX2e takes care of this in an easier way
type style.

\begin{em}
   A long segment of text can also be emphasized in this way. Text within
   such a segment is given additional emphasis
          with\/ \emph{Roman}
   type.  Italic type loses its ability to emphasize and become simply
   distracting when used excessively.
\end{em}

It is sometimes necessary to prevent \TeX\ from breaking a line where
it might otherwise do so.  This may be at a space, as between the
``Mr.'' and ``Jones'' in
       ``Mr.~Jones'',        % ~ produces an unbreakable interword space.
or within a word---especially when the word is a symbol like
       \mbox{\em itemnum\/}
that makes little sense when hyphenated across
       lines.

Footnotes\footnote{This is an example of a footnote.}
pose no problem.

\TeX\ is good at typesetting mathematical formulas like
       \( x-3y = 7 \)
or
       \( a_{1} > x^{2n} / y^{2n} > x' \). 
Remember that a letter like
       $x$        % $ ... $  and  \( ... \)  are equivalent
is a formula when it denotes a mathematical symbol, and should
be treated as one. 

\section{Displayed Text}

Text is displayed by indenting it from the left margin.
Quotations are commonly displayed.  There are short quotations
\begin{quote}
   This is a short quotation.  It consists of a
   single paragraph of text.  There is no paragraph
   indentation.
\end{quote}
and longer ones.
\begin{quotation}
   This is a longer quotation.  It consists of two paragraphs
        of text.  The beginning of each paragraph is indicated
   by an extra indentation.

   This is the second paragraph of the quotation.  It is just
   as dull as the first paragraph.
\end{quotation}
Another frequently-displayed structure is a list.
The following is an example of an {\em itemized} list.
\begin{itemize}
   \item  This is the first item of an itemized list.  Each item
          in the list is marked with a ``tick''.  The document
          style determines what kind of tick mark is used.

   \item  This is the second item of the list.  It contains another
          list nested inside it.  The inner list is an {\em enumerated}
          list.
          \begin{enumerate}
              \item This is the first item of an enumerated list that
                    is nested within the itemized list.

              \item This is the second item of the inner list.  \LaTeX\
                    allows you to nest lists deeper than you really should.
          \end{enumerate}
          This is the rest of the second item of the outer list.  It
          is no more interesting than any other part of the item.
   \item  This is the third item of the list.
\end{itemize}

\subsection{Displayed mathematics}

\LaTeX\ is good at typesetting mathematical
formulas. The ability to do this depends on much more than just having
special symbols available. The correct layout depends on the logical
structure of the expressions. If you write a moderate amount of
mathematics you will find enough information in the \LaTeX\
book~\cite{Lamport}.  

There is a wealth of mathematical symbols available in many sizes and
styles: not only
$\epsilon\phi\rho\theta\pi$,
but also their variants:
                 $\varepsilon\varphi\varrho\vartheta\varpi$
and special symbols galore:
${\cal F} \otimes \Re \aleph \heartsuit\bowtie\hbar$.
You can use modern digits   $1234567890$ or give your mathematics
that scholarly 19th century look with $\mit 1234567890$ (numbers
like these don't look as if they came out of a computer).

A displayed formula is one-line long; multiline formulas require special
formatting instructions.  Here is a simple one
\begin{equation}
   x' + y^{2} = z_{i}^{2}
\label{simpleone}  % assigns number of the equation to 'simpleone'
\end{equation}
% avoid blank lines here so that the sentence can continue without a paragraph indent
and here is something more entertaining 

%------------------------------------------------------------------------
% Here is some LaTeX input which will generate a formula.
% It is expanded over many lines for clarity and to allow room for
% lots of explanatory comments.  Remember that TeX does not
% attach significance to spaces in mathematics mode.
%
% First make some useful definitions which will probably be useful
% several times in the same or other papers ("structured programming").
%
% The following appear under the final sum in the example:
%
\newcommand{\kpositive}{k_1, k_2, \ldots k_m \ge 0 }
\newcommand{\mksum}{k_1 + 2k_2 + \cdots + mk_m = m}
%
% and we notice a pattern in the terms of the last product:
%
\newcommand{\term}[1]{\frac{\displaystyle S_#1^{k_#1}}%
                           {\displaystyle #1^{k_#1} k_#1!}%
                     }
%
% where the argument #1 will be either 1, 2 or m.
%
% The calls to \displaystyle force TeX to typeset at the normal size
% instead of shrinking numerator and denominator.
% Some additional calls to \displaystyle are necessary below since
% the \eqnarray environment seems to set its elements in \textstyle.
%
\begin{eqnarray}              % Now enter environment for equation arrays
%                                 = signs between pair of &s will line up
G(z) &=& e^{\ln G(z) }
       = \exp\left(   \sum_{k \ge 1} { S_k z^k \over k}   \right)
       =  \displaystyle \prod_{k \ge 1}  e^{S_k z^k /k}
                          \nonumber \\   % end of first line of alignment
    &=& \left(                                         %open big brackets
             1 + S_1 z +
                {S_1^2 z^2 \over 2! }
                  + \ldots                    %makes proper trailing dots
        \right)                                       %close big brackets
       \left( 1 + {S_2   z^2 \over 2}
                + {S_s^2 z^4 \over 2^2 \cdot 2!}
                + \ldots
                  \right) \ldots
                       \nonumber  \\    % end of second line of alignment
   &=&\displaystyle \sum_{m \ge 0}
      \left( \sum_{\kpositive \atop \mksum}   % use our definitions here
        \term{1} \term{2}  \cdots                  % dots at middle level
                 \term{m} \right)
                               z^m
\label{toughone}                % assign a label just in case we need it
        %N.B. no '\\' for last line of alignment (which will be numbered)
\end{eqnarray}                  % leave the equation array environment
%
%------------------------------------------------------------------------
%
The number of Eq.~\ref{toughone} is attached to the last line
but this is not essential.
Don't start a paragraph with a displayed equation like
Eq.~\ref{simpleone}, % this is how you refer to an equation by its number
nor make one a paragraph by itself.



\subsubsection{Hidden depths}
Occasionally, a document will become so complicated that
a 3rd level of sectioning is necessary.

\begin{thebibliography}{9}

\bibitem{Lamport} Leslie Lamport,
      \LaTeX: A Document Preparation System,
      Addison-Wesley, Reading, Massachussets, 2nd Edn. 1994.
 

\end{thebibliography}


\appendix
\section{How to make appendices}

Sometimes you will want to make appendices.  In    \LaTeX, an
appendix is just the same as a section, except that you should
call the \verb|\appendix| command     {\it just once\/}
to start the new sequence of (usually alphabetical) numbering.

\section{Graphics etc.}

This is the second appendix---note how it was done!

The creation of floating figures, tables and such is adequately
described in \cite{Lamport}.   


\end{document}