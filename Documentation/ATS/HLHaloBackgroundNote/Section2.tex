\section{Conclusion and outlook~\label{last}}

We investigated in detail and systematically two sources of background, beam-gas and beam-halo, the most relevant sources for the experiments. The study was carried out for IR1 and both beams in general. Although there is a layout symmetry for incoming and outgoing beams in IR1 and IR5, both beams have a different ``history'' which will result in different backround patterns. We studied in detail IR7 leakages to IR1 and IR5 for the $pp$ physics cases of Run~I in 2012, Run~II in 2016 and and several HL-LHC cases. We find that B2 gives systematically higher contributions than B1 in IR1 in all simulated cases (see Tab.~\ref{tab:leakageFactorsIR7}). Although IR5 leakages from IR7 can be significantly smaller than the leakage to IR1, this is not necessarily generally the case: it is true for 2015 Run~II and the HL reference case (TCT5s in, round beam optics, \twosigmaret~settings), but not for the Run~I case at 4~TeV when the leakage is on a similar level for B1 and slightly higher for B2 (IR1: $2.4 \times 10^{-5}$, IR5: $2.6 \times 10^{-5}$). Tracking simulations with the main settings for proton physics in 2016 were recently performed and one can see from in Tab.~\ref{2016leakageFactorsIR7} that the different optics can cause B2 to contribute only about a third of B1 despite the shorter path length freshly cleaned from IR7. These are examples that show that a more detailed analysis e.g.~considering the phase advance of the collimators are needed to understand more precisely leakages to the experimental areas.

The shower simulations in IR1 were performed with new improved techniques considering the beam size and the crossing angle. The inclusion of the beam size could be directly compared to simulations without beam size, and we find minor changes in the distributions (see Fig.~\ref{bsRatioPhiAll} and \ref{bsZAll}) at the interface plane. Since it is a more realistic model we used the new method also in Run~II simulations. The crossing angle was found to have a major impact in angular distributions at the interface plane.\\

We also presented for the first time shower simulations for IR1 of a new background source, off-momentum leakage from IR3. They show similar shapes as the IR7 losses at the interface plane. However, the asymmetry in the distributions of energy deviations resulting in a positive and negative frequency shift, is not well understood. This can be an interesting study with a dedicated analysis, however IR3 losses are sofar not problematic for the experiments.\\


While in Run I at 3.5~TeV and still up to 2012 at 4~TeV beam operation background induced from local beam-gas interactions has been the dominating source it can actually change in HL, where rates become similar depending on the actual vacuum quality. We presented a comparison of all background sources simulated in Run~II and HL-LHC with the currently best estimate for vacuum in the HL-LHC (see Fig.~\ref{fig:HLR2Muons}). 


%Implications for the machine.
Any measures which help to improve the beam life time have proven to be very useful in terms of reduction of beam-gas background like machine conditioning (scrubbing). At least an order of magnitude could the pressure be lowered for the run in 2015. This could for 2016 be exploited to push the LHC performance and reach lower $\beta*$ values. 
