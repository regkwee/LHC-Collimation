\section{Background sources}
Several sources in the machine contribute to background signals in the experiments (often also refered to as \textit{MIB} - machine induced background). The most relevant ones are induced by beam protons interacting with residual gas molecules, refered to as \textit{beam-gas}, and beam halo protons interacting with the tertiary collimators around $\pm$150~m away from the IP on each side of the incoming beam. Both sources are considered here and described in the following. Other sources are e.g.~cross-talk where scattered protons in an elastic or inelastic interaction process at an IP is intercepted in another TCT where it creates a shower towards the respective experiment. 

\subsection{Beam-Gas}

Beam-gas interactions can be divided into different classes depending on whether they collisions are close to the experiment (\textit{local beam-gas}) or far out (\textit{global beam-gas}) and whether it was an inelastic or elastic interaction.

\subsection{Beam-Halo}
Beam particles in the accelerator oscillate around an ideal orbit but diffuse out the beam core due to various beam dynamics effects (e.g.~particle-particle scattering within a bunch, interactions between colliding bunches, scattering with residual gas-molecules) forming the \textit{beam halo} and unavoidable losses. The task of the collimation system is to safely remove these halo particles in two dedicated insertion regions, IR7 for betatron and IR3 for momentum cleaning~\cite{LHCDesignRep,collRef}. Tertiary collimators (TCTs) in IR1 and IR5 complement the IR7/3 collimators and are installed to provide local protection of the inner triplet magnets. Halo protons leaking from IR7/3, should mostly be stopped by the TCTs. While interacting with the collimator material, they produce particle showers that contribute to the machine-induced halo background. This background source is considered in the following.

\subsection{(Off-momentum losses and others)}

