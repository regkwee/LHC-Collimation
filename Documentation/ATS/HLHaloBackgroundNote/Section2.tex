\section{Conclusion and outlook~\label{last}}

We investigated in detail and systematically two sources of machine-induced background, beam-gas and betatron halo, two of the most relevant sources for the experiments. The study was carried out for IR1 and both beams in general. Although there is a layout symmetry for incoming and outgoing beams in IR1 and IR5, both beams traverse different parts of the machine to arrive there from IR7, as well as have different crossing planes and transverse rotation of the beam screen, which could result in different backround patterns. We studied in detail IR7 betatron leakages to IR1 and IR5 for the $pp$ physics cases of Run~1 in 2012, Run~2 in 2015 and and several HL-LHC cases. For betatron halo, we find that B2 gives systematically higher leakages than B1 in IR1 in all simulated cases, see Tab.~\ref{tab:leakageFactorsIR7}. Although IR5 leakages from IR7 can be significantly smaller than the leakage to IR1, this is not necessarily generally the case: it is true for 2015 Run~2 and the HL-LHC reference case (TCT5s in, round beam optics, \twosigmaret~settings), but not for the Run~1 case at 4~TeV when the leakage is on a similar level for B1 and slightly higher for B2 (IR1: $2.4 \times 10^{-5}$, IR5: $2.6 \times 10^{-5}$).

Tracking simulations with the main settings for proton physics in 2016 were recently performed and one can see from Tab.~\ref{2016leakageFactorsIR7} that the different optics can cause B2 to contribute only about a third of B1 despite the shorter path length of B2 from IR7 to IR1 (IR1, B1: $4.2 \times 10^{-4}$, IR1 B2:$1.2 \times 10^{-4}$). These are examples that show that a more detailed analysis e.g.~considering the phase advance of the collimators are needed to understand more precisely collimation leakages to the experimental areas. We also showed examples e.g.~when comparing the HL-LHC baseline case to the current collimation layout of only one pair of TCTs installed in the experimental IR, that the surface hits in the collimator jaws cause most of the shower development at the interface plane. If one could reduce such interactions, e.g.~by retracting the TCT4s more with respect to the TCT5s it may be well beneficial, but it remains to be validated by new simulations.

The shower simulations of local beam-gas in IR1 were performed with new improved techniques sampling events according to transverse extension of the local beam size and including the crossing angle. The inclusion of the beam size could be directly compared to simulations where the beam-gas events were sampled only along the central orbit, and we find minor changes in the distributions (see Fig.~\ref{bsRatioPhiAll} and \ref{bsZAll}) at the interface plane. Since it is a more realistic model we used the new method also in Run~2 simulations. The crossing angle was found to have a major impact in angular distributions at the interface plane.

We also presented for the first time a simulation analysis of off-momentum halo leakage as third background source to the experiments, simulating particle shower development to IR1. A strong B1/B2 asymmetry is visible in the simulations concerning the amount of off-momentum halo leakage to IR1 and IR5. IR1 recieves almost entirely only contributions from B2 while background at IR5 of this source is exclusively produced by B1. These simulations show that off-momentum halo is also cleaned by the betatron insertion at IR7 which is situated only two octants before IR1 for B1 and also two octants before IR5 for B2. From experience with the machine it is known, that off-momentum losses are also caught in IR7 and fractions of the betratron losses scattered out of IR7 are cleaned in IR3. On an absolute scale, betatron losses are much higher than off-momentum losses and the contribution from off-momemtum halo on the TCT to the total particle flux at the interface plane remains to be quantitatively estimated.
Generally, off-momentum induced distributions show similar shapes at the interface plane as the IR7 losses. We observe another interesting feature when analysing pure off-momentum halos in the scenarios of 2012 and 2015. At 4 TeV in 2012, the simulations with a positive frequency shift (corresponding to a negative dp/p) produces almost exactly twice as much as when we simulate with a negative frequency shift. The same behavior is observed for IR5, although for slightly smaller IR3 leakages. However in the 2015 scenario at 6.5~TeV, the leakages to IR1 and IR5 are very similar even for both frequency shifts, see Tab.~\ref{tab:IR3leakageFactors} for details. This is not well understood and can be an interesting study with a dedicated analysis, however as mentioned IR3 losses due to off-momentum particles are so far not estimated to be problematic for the experiments. 


While in Run I in 2011 at 3.5~TeV, in 2012 at 4~TeV as well in 2015 in Run~2 at 6.5~TeV local beam-gas was the dominating source creating background to the experiments, betatron halo can, in HL-LHC, become of about equal importance to local beam-gas. This can be the case when optimistic scenarios for the vacuum quality are assumed as shown Fig.~\ref{fig:HLR2Muons} with the currently best estimate for vacuum in HL-LHC. However, it should be noted that all quantitative estimates for HL-LHC have a very high uncertainty.

It should be also noted that the comparisons presented in this note concern the distributions at the machine-detector interface plane, and that the actual background in the experiment depends also on the detector response. Therefore, in order quantify the background, a further simulation step of the detector is required. The simulations described here can be used as starting conditions for such a study.

%Implications for the machine.
Any measures which help to improve the vacuum quality have proven to be very useful in terms of background reduction from beam-gas like machine conditioning (scrubbing). Particularly effective were also all the hardware measures during LS1, with which pressures in the experimental IRs were partially lowered below sensitivity of some gauges.


