\section{Simulation Results for LHC's Run I and Run II\label{run1run2}}

The two background sources as previously described were investigated for several run scenarios. During Run I, the LHC was operated with two beam energies, 3.5 TeV in 2010--11 and 4~TeV in 2012\footnote{A detailed overview of parameters in Run I can be found in~\cite{parametersRun1}}. While the 3.5~TeV background data has been presented in~\cite{nimPaperRod}, we focus on new developments made since and use these improvements for 4 TeV and Run II simulation cases. Beam-halo simulations consider now a crossing angle in the simulation as it has been present in the machine. For beam-gas simulations we consider now the transverse extension of the beam.

\subsection{Run I: 4 TeV Beam-Halo}

Beam-halo induced showers on the tertiary collimators depend on the initial distribution of the positions at which the halo-proton interacts inelastically. For tungsten, the collimator material of the TCTs, one can expect that inelastic interactions close to the jaw surface produce more shower particles than those deeper inside when most of the inelastic processes is confined within the jaw thereby also the shower production. This depth distribution for is shown for B1 and B2 in Fig.~\ref{inel4TeV} and one can see the hits are rather close to the jaw surface with most of the hits in the first few bins (note the logarithmic scale).

\begin{figure}[!htb]
\centering
\includegraphics[width=0.45\textwidth]{figures/inelposition_sum_impacts_real_NewScatt_4TeV_haloB1.pdf}
\includegraphics[width=0.45\textwidth]{figures/inelposition_sum_impacts_real_NewScatt_4TeV_haloB2.pdf}
 \caption{Positions of inelastic interactions of the two collimator jaws normalised to total hits in the simulation. The entries were normalised to the total number of hits in the TCTs. The total number of lost protons on the TCPs is shown in Tab.~\ref{leakageFactorsIR7}.
  \label{inel4TeV}}
\end{figure}

General, characteristic distributions of B1 halo induced showers per TCT hit are shown in Fig.~\ref{dist4TeVB1} and more can be found in the Appendix~\ref{run1run2app}. The energy spectrum of all charged and neutral particle species is shown in Fig.~\ref{dist4TeVB1}. We note the peak at beam energy of the protons arising from single diffractive dissociations, emerging from such an interaction slightly off-energy. One can also see that in the 10 to around 500 GeV regime muons are the most dominant particle species, at lower energies there are overwhelmingly many photons, followed by electrons. The azimuthal distribution of their energy as Fig.~\ref{dist4TeVB1} (right) is dominated by protons which is clear from the single diffractive part. We note, the energy of muons peak in the horizontal plane at $\phi \approx 0, \pm \pi$. 

\begin{figure}%[!htb]
\begin{center}
\includegraphics[width=0.49\textwidth]{figures/4TeV/haloB1_20MeV/Ekin_BH_4TeV_B1_20MeV.pdf}
\includegraphics[width=0.49\textwidth]{figures/4TeV/haloB1_20MeV/PhiEnDist_BH_4TeV_B1_20MeV.pdf}
\end{center}
\vspace{-0.6cm}
 \caption{Halo induced background by Beam 1 at the interface plane showing the energy spectrum (left) and its distribution in $\phi$.
  \label{dist4TeVB1}}
\end{figure}

Comparing B1 to B2 halo induced distributions as in Fig.~\ref{comp4TeVB1B2}, one can see all of the shapes are quite similar. The energy spectrum of all particles and the energy of all particles. One can recognise that there can be more energy expected from B2 than from B1 per TCT hit. 



\begin{figure}%[!htb]
\begin{center}
\includegraphics[width=0.49\textwidth]{figures/4TeV/compB1B2/perTCThit/ratioEkinAll.pdf}
\includegraphics[width=0.49\textwidth]{figures/4TeV/compB1B2/perTCThit/ratioPhiEnAll.pdf}
\end{center}
\vspace{-0.6cm}
 \caption{Comparison of B1 and B2 halo shower distributions at the interface plane. The numbers in the ratio plot is the ratio of both integrals of the top distributions to indicate by how much nominator or denominator is larger. The error bars indicate statistical uncertainties.
  \label{comp4TeVB1B2}}
\end{figure}


% --------------------------------------------------------------------------------------------
\subsection{Run I: 4 TeV Beam-Gas}

Characteristic distributions induced by beam-gas interactions are shown in Fig.~\ref{dist4TeVBGbs}. The energy spectra, Fig.~\ref{dist4TeVBGbs} (left), highlights a typcial shape for beam-gas interactions: in contrast to beam-halo showers it does not have the single diffractive peak at beam energy. Beam-gas induced protons show a much smoother increase up to beam-energy than in the halo case. It is still due to the same physics process, single diffractive dissociation, but since there is a much less localised source of interactions on the beamline as in the halo case (where the source are the TCTs) the peak is not as pronounced as for the halos.
One immediate difference one notices, when also looking at the right of that figure, there is on average a factor 10 or more particles and energy produced by a beam-gas interaction than by a TCT interaction. %One can clearly see that comparing the energy of protons at small radii, essentially the first bin of Fig.~\ref{dist4TeVB1}~(d) and Fig.~\ref{dist4TeVBGbs}~(d).


\begin{figure}%[!htb]
\centering
\includegraphics[width=0.49\textwidth]{figures/4TeV/bs_20MeV/Ekin_BG_4TeV_20MeV_bs.pdf}
\includegraphics[width=0.49\textwidth]{figures/4TeV/bs_20MeV/PhiEnDist_BG_4TeV_20MeV_bs.pdf}
 \caption{Beam-gas induced background at the interface plane. The energy spectrum (left) of different particles and their energy in $\phi$ (right) is shown.
  \label{dist4TeVBGbs}}
\end{figure}
% --------------------------------------------------------------------------------------------
\subsection{Comparison to previous techniques}

We compare simulations performed for Run~I at 4 TeV and at 3.5~TeV (as has been presented in Ref.~\cite{nimPaperRod}). The direct comparison of the new simulation techniques is not possible as the beam energy and optics were not exactly the same. We therefore also show a ratio to overcome the difference due to the slightly different beam energy. In fact, at 3.5~TeV there were several optics and collimator settings deployed which enables us to conclude about the question if the TCT settings, which were almost 4~$\sigma$ different, will have any influence on beam-gas. 


\paragraph{Transverse beam size}
We compare the simulations of the two different techniques either a pointlike size or with transverse extensions. Comparisons are made in Fig.~\ref{bsRatioPhiAll} showing $\phi$ distributions of all particles and their energy. One can see there are small differences in the particle distribution (left) while the energy in $\phi$ (right) are smoother with the new technique. Especially the peaks at $\pm \frac{\pi}{2}$ in the energy plot show clearly there is a re-distribution to the ``shoulders''.

%As the energy in $\phi$ is almost shaped by protons, we show protons and muons as they are the ones reaching far into the detector region. Indeed, there is almost the same picture for the azimuthal proton distributions shown in Fig.\ref{bsRatioPhiMP}. The effect is more pronounced for high energy protons as well in the same figure. On muons also in the figure, the new technique does not seem to have a significant impact at all.

We investigate differences in the four s-sections from Fig.~\ref{twissfileBS} to see if there is a visible difference where one could expect the beam size to be rather large. This is compared in Fig.~\ref{bsZAll} for all particles where the multiplicity (top) and energy (bottom) is shown. One remarks a few places where one has clear differences like in the last s-section, 269 to 547~m, at $\phi \approx -1$, where the ratio is going down by $60~\%$. Numerically, this is not important with about 0.0033 particles per BG interaction. The statistical uncertainties are also largest meaning there are only a small contribution from that region. The corresponding plot in the bottom row reveals that at $\phi \approx -1$ that the energy can be higher when the beam size is included by about 10~to 20~$\%$.
In the two middle s-sections, one can observe similar differences however the range is smaller with 10 to 15~$\%$ in the multiplicity and energy plot. The first section though contains the effect described visible in the distribution when integrated over all s-sections. This ``smearing'' of the peaks at $\pm \frac{\pi}{2}$ is visible in both multiplicity and energy and makes up to 60~$\%$ difference.

Looking back at Fig.~\ref{bsRatioPhiAll} one can conclude that this new technique has a slight effect only up to $5~\%$ for the mulitplicity and up to 40~$\%$ in the vertical plane. We discussed here only the differences in $\phi$, but we add other distributions also per particle types for further information in Fig.~\ref{bsRatioEkin}, Fig.~\ref{bsRatioRadN}, and Fig.~\ref{bsRatioRadEn}.

\begin{figure}%[!htb]
\begin{center}
  \includegraphics[width=0.49\textwidth]{figures/4TeV/beamsizeRatio/ratioPhiNAll.pdf}
  \includegraphics[width=0.49\textwidth]{figures/4TeV/beamsizeRatio/ratioPhiEnAll.pdf}
\end{center}
\vspace{-0.6cm}
 \caption{4 TeV azimuthal distributions of all particles and their energy.
  \label{bsRatioPhiAll}}
\end{figure}


%% \begin{figure}%[!htb]
%% \begin{center}
%%   \includegraphics[width=0.452\textwidth]{figures/4TeV/bs_20MeV/PhiEnAllZ_BG_4TeV_20MeV_bs.pdf}
%% %  \includegraphics[width=0.452\textwidth]{figures/4TeV/bs_20MeV/PhiEnProtZ_BG_4TeV_20MeV_bs.pdf}
%% \end{center}
%% %\begin{picture} (0.,0.)
%% %% \setlength{\unitlength}{1.0cm}
%% %% \small{
%% %%     \put ( 4.,2.){all}
%% %%     \put ( 12.4,2.){protons}}
%% %% \end{picture}
%%  \vspace{-0.6cm}
%%  \caption{4 TeV azimuthal energy distribution of all particles in different s-sections (see also Fig.~\ref{twissfileBS}).
%%   \label{bsZ2}}
%% \end{figure}

\begin{figure}%[!htb]
\begin{center}
  \includegraphics[width=0.24\textwidth]{figures/4TeV/beamsizeRatio/ratioPhiNAllZ1.pdf}
  \includegraphics[width=0.24\textwidth]{figures/4TeV/beamsizeRatio/ratioPhiNAllZ2.pdf}
  \includegraphics[width=0.24\textwidth]{figures/4TeV/beamsizeRatio/ratioPhiNAllZ3.pdf}
  \includegraphics[width=0.24\textwidth]{figures/4TeV/beamsizeRatio/ratioPhiNAllZ4.pdf}
  \includegraphics[width=0.24\textwidth]{figures/4TeV/beamsizeRatio/ratioPhiEnAllZ1.pdf}
  \includegraphics[width=0.24\textwidth]{figures/4TeV/beamsizeRatio/ratioPhiEnAllZ2.pdf}
  \includegraphics[width=0.24\textwidth]{figures/4TeV/beamsizeRatio/ratioPhiEnAllZ3.pdf}
  \includegraphics[width=0.24\textwidth]{figures/4TeV/beamsizeRatio/ratioPhiEnAllZ4.pdf}
\end{center}
\vspace{-0.6cm}
 \caption{Direct comparisons of azimuthal distributions of all particles (top) and their energy (bottom) in different s-sections as in Fig.~\ref{twissfileBS}.
  \label{bsZAll}}
\end{figure}

 
\paragraph{Effect of the crossing angle}

We can study the crossing angle effect on background in IR1 using the 3.5~TeV simulation data of~\cite{nimPaperRod}, thus halo and beam-gas induced, without any crossing angle included in the simulations and the 4~TeV halo data with a crossing angle of 290~$\mu$rad. For this comparison, we show the distributions at the interface plane per TCT hit. 

We find that the beam-gas data\footnote{$\beta^* = 1.0~$m had TCTs set to $11.8~\sigma$~\cite{nimPaperRod}} in Fig.~\ref{xingCompBG} exhibit a clear signature of the improved simulation technique including the crossing angle. There is a general accumulation in the upper part of the vertical plane; the beam is directed downwards but as the sampling position is still in the upper hemisphere the peak of particles is around $+ \frac{\pi}{2}$. Muon distributions are not significantly affected as other effects are more dominating the azimuthal distribution (like magnetic fields effects). 

In contrast, the beam-halo data in Fig.~\ref{xingCompBH} do not have any clear signs of crossing angle effect when comparing 3.5~TeV\footnote{TCTs were at $15~\sigma$ for a $\beta^*$ optics of 3.5~m.} to 4 TeV simulations in both beams. It is likely that secondary particles with all kind of forward directions are created at the TCTs losing the initial direction of the crossing angle of the beam protons.

\begin{figure}
\begin{center}
  \includegraphics[width=0.41\textwidth]{figures/4TeV/compBG_3p5_vs_4TeV/ratioPhiEnAll.pdf}
  \includegraphics[width=0.41\textwidth]{figures/4TeV/compBG_3p5_vs_4TeV/ratioPhiEnPhotons.pdf}
  \includegraphics[width=0.41\textwidth]{figures/4TeV/compBG_3p5_vs_4TeV/ratioPhiEnMuons.pdf}
  \includegraphics[width=0.41\textwidth]{figures/4TeV/compBG_3p5_vs_4TeV/ratioPhiEnMuE100.pdf}
\end{center}
\vspace{-0.6cm}
 \caption{Run I beam-gas data: Clear effect of crossing angle visible showing accumulations of particles at around $\frac{\pi}{2}$ in the vertical plane. The distribution of muons and in particular high energy muons are not significantly affected. This is the same if the data with beamsize is used, see Fig.~\ref{xingCompBG2}.
  \label{xingCompBG}}
\end{figure}


\begin{figure}
\begin{center}
  \includegraphics[width=0.41\textwidth]{figures/4TeV/compB2_3p5vs4TeV/ratioPhiEnAll.pdf}
  \includegraphics[width=0.41\textwidth]{figures/4TeV/compB2_3p5vs4TeV/ratioPhiEnPhotons.pdf}
\end{center}
\vspace{-0.6cm}
\caption{B2 beam halo data without a clear sign of a crossing-angle effect visible, in particular no enhancement around $\pi/2$. B1 is very similar shown in Fig.~\ref{xingCompBHB2}.
%   for b2 Energy enhancement between $-\frac{\pi}{2}$ and $-\pi$, enhancement of photons in horizontal plane?? More muons around $-\frac{\pi}{2}$?
  \label{xingCompBH}}
\end{figure}

\paragraph{Effect of TCT settings on beam-gas}

Since the TCTs were set to 11.8~$\sigma$ to accommodate the $\beta^* = 1.0~$m optics at 3.5~TeV and 9~$\sigma$ at 4~TeV for $\beta^* = 60~$cm, the comparison of contributions from downstream of the TCTs can give indications about the influence of TCT settings on beam-gas, Fig.~\ref{compBGrun1} shows there is no significant influence, focussing on a the closest s-section further away and the transverse radius between 1 and 2~m.

\begin{figure}
  \centering
  \includegraphics[width=0.41\textwidth]{figures/4TeV/compBG_3p5_vs_4TeV/perBGint_bs/ratioPhiEnMuonsZ3R1.pdf}
  \includegraphics[width=0.41\textwidth]{figures/4TeV/compBG_3p5_vs_4TeV/perBGint_bs/ratioPhiEnMuonsZ3R2.pdf}
  \caption{Influence of contributions from downstream the TCTs: no clear difference visible.
  \label{compBGrun1}}
\end{figure}

\subsection{Run I: Renormalisation of beam gas events with pressure profile \label{BGreweighted4TeV}}

\begin{table}
   \centering
   \caption{LHC fill 2736 Run I (2012)~\cite{refAccStats}}
   \begin{tabular}{l||c}
       \hline
       beam energy  & 4~TeV \\
       Fill start time (local Geneva time) & 16/06/2012 18:22\\
       Stable beam start time (local Geneva time) & 16/06/2012 20:10\\
       Stable beam duration [hh:mm] & 17:29\\
       bunch intensity ring 1& 2.029$\times 10^{14}$ protons\\
       bunch intensity ring 2& 1.995$\times 10^{14}$ protons\\
       number of bunches & 1374 \\
       \hline
   \end{tabular}
   \label{tab:fillRunI}
\end{table}

We use the simulations with beam size to reweight to a pressure map.
The information of the original inelastic proton interaction of each particle at the interface plane was kept in order to attribute weights according to a typical pressure map. For 2012, the simulations of the gas densities are based on gauge data recorded in LHC fill 2736 and were performed using VASCO (VAcuum Stability COde)~\cite{vascoRef}. The pressures were simulated up to the end of the long straight section of IR1 (LSS1) for the most dominant molecules, H$_2$, CH$_4$, CO, CO$_2$. Partial pressure euqivalents expressed in molcular densities are shown in the top of Fig.~\ref{pressure2012} and are decomposed into atomic components at the bottom of that figure to show the interaction probability as function of the s-location using the inelastic cross-sections for H, C and O as indicated in Tab.~\ref{tab:atomicXsections} for a 4 TeV beam. Beyond LSS1, where the dispersion suppression region starts, the last value is assumed as constant up to 547~m.

As described in the previous section, the weights are calculated using Eq.~\ref{eq3}. The total number of protons is given by the maximum beam intensity of LHC fill 2736 which was about $2.0 \times 10^{14}$, see Tab.~\ref{tab:fillRunI} for more fill details. The method is shown in Fig.~\ref{fig:method}: in green the total interaction probability rate, in black the number of muons per beam-gas interaction and in red the multiplication of both. One can see the result is mainly shaped by the interaction probability which is based on the pressure map. Only beyond s~=~270~m the shape comes from the shower simulations. There are three locations that contribute most to background with highest rates of more than $10^4$~particles/s at around 270~m, where transition region of the last quadrupole of the matching section in the LSS, Q7\footnote{Q7 is operated at 1.9~K like the triplet quadrupoles Q1, Q2, Q3 and all quadrupoles in the dispersion suppression and arc~\cite{LHCDesignRep}}, is located. Comparable are rates of about $10^4$~particles/s are produced by pressure spikes in the triplet and at $s \approx 147~$m, the location of the TCTs.

We compare in detail how the distributions change when normalising from a pressure that corresponds to a flat to a non-flat profile. We highlight in Fig.~\ref{fig:OrigZ4TeV} the origin of creation of all particles as function of $s$. To compare better, the distributions corresponding to a flat profile are scaled up by seven orders of magnitude to match the scale of the rate. The shape changes for all particles are mainly driven by the pressure in LSS1, one can recognise the same three peaks as discussed earlier (triplet, TCTs and Q7), however for all particles they seem slightly broader. Single particle species are shown in Fig.~\ref{fig:OrigZ4TeV2} in the appendix for photons and protons. 

We can observe that the integrated distributions at the interface plane change differently focussing here on muons. The energy spectrum in Fig.~\ref{fig:cv81EkinPhiEn4TeV} shows an underestimation of particles from about 10~GeV to the end of the spectrum. In the medium energy range of about 1 to 5~GeV there seems a small overestimation of particles. From the azimuatl distribution of the energy (top right) in Fig.~\ref{fig:cv81EkinPhiEn4TeV} one can see the additional energy shows up on the upper and lower hemisphere. While the bottom left plot show very similar shapes, the bottom right plot is clearly different for $r \approx 150~$cm or larger when the reweighted curves become slightly shallower. Similar conclusions can be derived for all particles shown in the appendix in Fig.~\ref{fig:cv81EkinPhiEn4TeV2}. Differences for protons and neutrons before and after reweightening is shown in Fig.~\ref{fig:cv81ProtNeut4TeV}.

\begin{figure}%[!htb]
\begin{center}
  \includegraphics[width=0.75\textwidth]{figures/4TeV/LSS1_B1_Fill2736_Finala_pressure.pdf}
  \includegraphics[width=0.75\textwidth]{figures/4TeV/reweighted/cv65_pint.pdf}
\end{center}
\vspace{-0.6cm}
 \caption{Gas densities in LSS1 shown for the most common molecules (top) and split into atomic components to indicate the interaction probability (bottom). The incoming beam is from left to right, i.e.~only the part at negative values up to a distance of -22.6~m is used. Pressure rises usually at transition regions of different temperatures (triplet quadrupoles Q1~--~Q3 and those of the matching section Q4~--~Q7) and at locations with materials of different thermal outgassing, e.g.~the TCT region at -147~m.
  \label{pressure2012}}
\end{figure}

\begin{figure}%[!htb]
\begin{center}
  \includegraphics[width=0.95\textwidth]{figures/4TeV/reweighted/muons2012.pdf}
\end{center}
\vspace{-0.6cm}
 \caption{Origin of muon production normalised to the pressure as in Fig.~\ref{pressure2012}. See text for description.
  \label{fig:method}}
\end{figure}


\begin{figure}
\begin{center}
  \includegraphics[width=0.75\textwidth]{figures/4TeV/reweighted/cv81_OrigZAll_BG_4TeV_20MeV_bs}
%  \includegraphics[width=0.75\textwidth]{figures/4TeV/reweighted/cv81_OrigZMuon_BG_4TeV_20MeV_bs}
%  \includegraphics[width=0.75\textwidth]{figures/4TeV/reweighted/cv81_OrigZProtons_BG_4TeV_20MeV_bs}
\end{center}
\vspace{-0.6cm}
 \caption{Origin of all particles production along $s$, in black before and in pink after re-normalising to the 2012 pressure profile. The rates are dominated by photons created on the very last meters before the interface plane (also the spike at $s=520~$m is due to local photon production, see middle plot in Fig.~\ref{fig:OrigZ4TeV2}).
  \label{fig:OrigZ4TeV}}
\end{figure}

\begin{figure}
\begin{center}
%% %%  \includegraphics[width=0.45\textwidth]{figures/4TeV/reweighted/cv81_EkinAll_BG_4TeV_20MeV_bs}
%%   \includegraphics[width=0.45\textwidth]{figures/4TeV/reweighted/cv81_EkinMuons_BG_4TeV_20MeV_bs}
%% %%  \includegraphics[width=0.45\textwidth]{figures/4TeV/reweighted/cv81_PhiEnAll_BG_4TeV_20MeV_bs}
%%   \includegraphics[width=0.45\textwidth]{figures/4TeV/reweighted/cv81_PhiEnMuons_BG_4TeV_20MeV_bs}
%%   %% \includegraphics[width=0.45\textwidth]{figures/4TeV/reweighted/cv81_RadNAll_BG_4TeV_20MeV_bs}
%%    \includegraphics[width=0.45\textwidth]{figures/4TeV/reweighted/cv81_RadNMuons_BG_4TeV_20MeV_bs}
%%   %% \includegraphics[width=0.45\textwidth]{figures/4TeV/reweighted/cv81_RadEnAll_BG_4TeV_20MeV_bs}
%%    \includegraphics[width=0.45\textwidth]{figures/4TeV/reweighted/cv81_RadEnMuons_BG_4TeV_20MeV_bs}
  \includegraphics[width=0.45\textwidth]{figures/4TeV/compBGflat/ratioEkinMuons.pdf}
  \includegraphics[width=0.45\textwidth]{figures/4TeV/compBGflat/ratioPhiEnMuons.pdf}
  \includegraphics[width=0.45\textwidth]{figures/4TeV/compBGflat/ratioRadNMuons.pdf}
  \includegraphics[width=0.45\textwidth]{figures/4TeV/compBGflat/ratioRadEnMuons.pdf}
\end{center}
\vspace{-0.6cm}
 \caption{Energy spectrum (top left), energy in $\phi$ (top right), transverse radius $r$ (bottom left) and energy in $r$ (bottom right) of muons before (black) and after (light color) reweightening to the pressure profile. The black curves are scaled up to better compare the shapes. %See Fig.~\ref{fig:cv16EkinPhiEn4TeV} for more details.
  \label{fig:cv81EkinPhiEn4TeV}} 
\end{figure}

\clearpage
\newpage
% --------------------------------------------------------------------------------------------
\subsection{Run I: Off-momentum leakage and shower simulations at 4 TeV}

Analogue to IR7 leakage to the experimental IRs of ATLAS and CMS, one can now investigate background contributions from IR3 off-momentum cleaning using the losses leaking to IR1 and IR5 TCTs, also expressed in TCP-to-TCT conversion factors. They are listed in Tab.~\ref{tab:IR3leakageFactors} for IR1 and IR5 for all the simulated cases.

We also note something interesting from Tab.~\ref{tab:IR3leakageFactors} when comparing the case ``+~500~Hz'' (plus-case) with ``-~500~Hz'' (minus-case): At 4~TeV, the negative fequency shift causes only half of the leakage to IR1 with $0.012$ and $0.0063$. The same is true for IR5: The losses are reduced from $0.045$ to $0.025$. On absolute terms these numbers appear high compared to leakage losses from IR7, e.g.~at the same Run~I scenario in Tab.~\ref{leakageFactorsIR7}. It will be interesting to estimate from data realistic loss rates to normalise these contributions.

We depict in Fig.~\ref{inel4TeVOffmom} how deep the inelastic interactions take place within the jaws. These distributions show a clear difference to those leaking from IR7: First, we note the full contribution for IR1 comes from B2, there is only a negligable fraction of B1 protons hitting the TCTs. Vice versa for IR5, almost nothing ends up on the TCTs in IR5 from B2 and contributions from IR3 leakage is due to B1 only. Then we notice, the distributions go much further into the jaw compared to e.g.~\ref{inel4TeV} for both beams in the ``minus-case'' but also for B1 in the ``plus-case''. B2 ``plus-case'' has the intercepted protons more on the surface, most of them up to around 5~mm which is more similar to the halo. 

We use the B2 TCT seeds in the shower simulations for IR1. General distributions for the main particle species are shown for a negative shift in Fig.~\ref{offmom4TeV}. They look rather familar from halo induced showers. More characteristic distributions of the ``plus-case'' and a detailed comparison of the two cases are shown in the appendix in Fig.~\ref{offmom4TeV2}, Fig.~\ref{compPM_ekin} and Fig.~\ref{compPM_phien}.

We observe as Fig.~\ref{compPM} shows, one can expect about 20~$\%$ more particles carrying about 30$~\%$ more energy when particles experience a negative than a positive shift per TCT interaction.


\begin{figure}
\begin{center}
\includegraphics[width=0.49\textwidth]{figures/inelposition_sum_impacts_real_4TeV_plus500Hz_TCT_B1.pdf}
\includegraphics[width=0.49\textwidth]{figures/inelposition_sum_impacts_real_4TeV_plus500Hz_TCT_B2.pdf}
\includegraphics[width=0.49\textwidth]{figures/inelposition_sum_impacts_real_minus500Hz_4TeVB1.pdf}
\includegraphics[width=0.49\textwidth]{figures/inelposition_sum_impacts_real_minus500Hz_4TeVB2.pdf}
\end{center}
\vspace{-0.6cm}
 \caption{Positions of inelastic interactions as given by SixTrack within the collimator jaws.
  \label{inel4TeVOffmom}}
\end{figure}


\begin{figure}
\begin{center}
  \includegraphics[width=0.49\textwidth]{figures/4TeV/offmom/20MeV/Ekin_offmin500Hz_4TeV_B2_20MeV.pdf}
  \includegraphics[width=0.49\textwidth]{figures/4TeV/offmom/20MeV/PhiEnDist_offmin500Hz_4TeV_B2_20MeV.pdf}
\end{center}
\vspace{-0.6cm}
 \caption{Off-momentum induced particle distributions.
  \label{offmom4TeV}}
\end{figure}

\begin{figure}
  \centering
  \includegraphics[width=0.49\textwidth]{figures/4TeV/offmom/comppm500Hz/ratioEkinAll.pdf}
  \includegraphics[width=0.49\textwidth]{figures/4TeV/offmom/comppm500Hz/ratioPhiEnAll.pdf}
  \caption{Direct comparison of the energy spectrum of all particles (left) and azimuthal distribution of energy (right).
    \label{compPM}}
\end{figure}

% --------------------------------------------------------------------------------------------
\subsection{Run II: 6.5 TeV Beam-Halo}

We move to the Run~II 2015 scenario with $\beta^* = 80$~cm, and details are in Tab.~\ref{paramsRun12}, using the same method as described for Run~I at 4~TeV. The input distributions are shown for B1 and B2 in Fig.~\ref{inel6.5} and can already see that the B1 depth distribution in the TCTV is much shallower than at 4~TeV in Fig.~\ref{inel4TeV}. This is very likely due to the different optics and collimator settings with the TCTs being now at almost 5~$\sigma$ further out. Characteristic distributions are shown in Fig.~\ref{dist6500GeVB2}, and a comparison to B1 distributions are in Fig.~\ref{compBHB1B2run2}. One can see that in contrast to 4 TeV data, B1 produces more shower particles and energy than B2 per inelastic interaction within the TCTs. 


\begin{figure}[!htb]
\begin{center}
\includegraphics[width=0.49\textwidth]{figures/inelposition_sum_HALOB1.pdf}
\includegraphics[width=0.49\textwidth]{figures/inelposition_sum_HALOB2.pdf}
\end{center}
 \caption{Depth of the inelastic interaction of the halo hits with the collimator material for B1 and B2 6.5 TeV hits.
  \label{inel6.5}}
\end{figure}


\begin{figure}%[!htb]
\centering
\includegraphics[width=0.49\textwidth]{figures/BH_run2/b2/Ekin_BH_6500GeV_haloB2_20MeV.pdf}
\includegraphics[width=0.49\textwidth]{figures/BH_run2/b2/PhiEnDist_BH_6500GeV_haloB2_20MeV.pdf}

 \caption{B2 halo induced background at the interface plane. The distributions exhibit the similar features as at 4 TeV.
  \label{dist6500GeVB2}}
\end{figure}


\begin{figure}%[!htb]
\centering
  \includegraphics[width=0.49\textwidth]{figures/BH_run2/perTCThit/ratioEkinAll.pdf}
  \includegraphics[width=0.49\textwidth]{figures/BH_run2/perTCThit/ratioPhiEnAll.pdf}

 \caption{Run~II 2015 case: Comparison of B1/B2 halo induced distributions per TCT hit.
  \label{compBHB1B2run2}}
\end{figure}



\subsection{Run II: 6.5 TeV Beam-Gas}

This case has been simulated with the same method as for the Run I scenario, using the extended beam size. The beam optics and collimator settings were different in the 2015 Run II, as detailed in Tab.~\ref{paramsRun12}. Characteristic distributions, the energy spectrum and the azimuthal distribution per particle type, are shown in Fig.~\ref{bg6500}. Very familiar shapes are produced at 6.5~TeV as for the beam-gas simulations at 4~TeV. More distributions are in the appendix in Fig.~\ref{bg65002}. 

\begin{figure}%[!htb]
\begin{center}
  \includegraphics[width=0.49\textwidth]{figures/6500GeV/20MeV/Ekin_BG_6500GeV_flat_20MeV_bs.pdf}
  \includegraphics[width=0.49\textwidth]{figures/6500GeV/20MeV/PhiEnDist_BG_6500GeV_flat_20MeV_bs.pdf}
%  \includegraphics[width=0.49\textwidth]{figures/6500GeV/20MeV/RadNDist_BG_6500GeV_flat_20MeV_bs.pdf}
%  \includegraphics[width=0.49\textwidth]{figures/6500GeV/20MeV/RadEnDist_BG_6500GeV_flat_20MeV_bs.pdf}
\end{center}
\vspace{-0.6cm}
 \caption{Characteristic beam-gas induced distributions at 6.5~TeV per BG interaction using the more realistic model of the beam size.
  \label{bg6500}}
\end{figure}

\subsection{Run II: Renormalisation of beam gas events with pressure profile}

Similar to Run I in 2012, a representative fill (number 4536) was chosen and used by the LHC Vacuum group to simulate the pressure map. Fill details are summed up in Tab.~\ref{tab:fillRunII}. 

\begin{table}
   \centering
   \caption{LHC fill 4536 Run II (2015)~\cite{refAccStats}}
   \begin{tabular}{l||c}
       \hline
       beam energy  & 6.5~TeV \\
       Fill start time (local Geneva time) & 26/10/2015 02:59\\
       Stable beam start time (local Geneva time) & 26/10/2015 10:26\\
       Stable beam duration [hh:mm] & 01:04\\
       bunch intensity ring 1& 2.2899$\times 10^{14}$ protons\\
       bunch intensity ring 2& 2.2137$\times 10^{14}$ protons\\
       number of bunches & 2041 \\
       \hline
   \end{tabular}
   \label{tab:fillRunII}
\end{table}

\paragraph{\textit{Pressure Map Simulations}}
In previous years (in particular for 3.5~TeV and 4~TeV runs data), as mentioned earlier in Sec.~\ref{BGreweighted4TeV} the code to simulate the pressure map, \textsc{VASCO}~\cite{vascoRef}, was used. For Run II in 2015, pressures were simulated with an improved version of the code. The code has been re-implemented in \textsc{Python} and modifications were made to automise pressure simulations~\cite{christinasStudent}. The describtion of photon desorption, as determined by SynRad~\cite{synradRef}, and electron cloud effects as determined in PyECLOUD~\cite{giovanniPhd} was included and therefore the photon and electron flux simulations should be more accurate~\cite{christinaPriv}. Original algorithms were kept however a few assumptions were made to be able to simulate pressures around the entire ring, in particular where no vacuum instrumentation is placed (e.g.~in the arc). The variety of materials and pumping speeds were simplified~\footnote{Some details were written up but are not publicly available.}. The pressure for fill 4536 shown in Fig.~\ref{pressure2015} is supposed to be on the lower limit of pressures~\cite{christinaPriv}. 

For comparison we added the total interaction rate based on the pressure map at 4~TeV. One can see there are about 1 to 2 orders of magnitude difference. This is expected as during the first long shutdown of the LHC from 2013 into 2015. Besides consolidation activities~\cite{KatyForazIpac14}, there were hardware improvements for vacuum in the long straight sections of the experimental insertions (between left Q4 to right Q4), where NEG (Non-Evapourable-Gatter) cartridges with additional 400l/s for H$_2$ pumping speed were installed at the same places as the pressure gauges~\cite{christinaPriv}. We also note that the peak at the TCT region as has been present in 4~TeV's fill has vanished. This is very likely due to the simplification of materials made for the 2015 pressure map simulation.


\paragraph{\textit{Reweighted Results using 2015 pressure}}

The pressure simulation data based on Fill 4536 is shown in top of Fig.~\ref{pressure2015} for each gas type, while the bottom figure shows the interaction probability per second as in Eq.~\ref{eq2} and what is used to reweight the distributions at the interface plane as decribed in Sec.~\ref{BGdescript}. As beam intensity we considered what was present in ring 1, see Tab.~\ref{tab:fillRunII}.

We use muons distributions to illustrate how they change before and after reweightening. The muon production as function of $s$ and general distributions integrated in $s$  muons at the interface plane are highlighted in Fig.~\ref{fig:OrigZ6p5} and Fig.~\ref{fig:EkinPhiEn6p5} respectively. We note the profile of the muon rate in Fig.~\ref{fig:OrigZ6p5} is very close to the pressure map that was used. 

\begin{figure}
\begin{center}
  \includegraphics[width=0.95\textwidth]{figures/6500GeV/reweighted/Density_Fill4536_2041b_26158_B1_withECLOUD_rho.pdf}
  \includegraphics[width=0.95\textwidth]{figures/6500GeV/reweighted/compallpint.pdf}
\end{center}
\vspace{-0.6cm}
 \caption{2015 pressure map derived gas densities shown for the most common molecules (top) and split into atomic components. The beam direction (top) is from left to right, and we use the left side of the incoming beam only to derive the total interaction probability (bottom). The data for the 4~TeV fill is also shown (but without constant extension out to the arc).
  \label{pressure2015}}
\end{figure}

\begin{figure}
\begin{center}
%  \includegraphics[width=0.75\textwidth]{figures/6500GeV/reweighted/cv81_OrigZAll_BG_6500GeV_flat_20MeV_bs.pdf}
  \includegraphics[width=0.75\textwidth]{figures/6500GeV/reweighted/cv81_OrigZMuon_BG_6500GeV_flat_20MeV_bs.pdf}
%  \includegraphics[width=0.75\textwidth]{figures/6500GeV/reweighted/cv81_OrigZProtons_BG_6500GeV_flat_20MeV_bs.pdf}
\end{center}
\vspace{-0.6cm}
 \caption{Origin of muons produced along $s$, in black before and in gold after re-normalising to the 2015 pressure profile. All particles, protons and photons are shown in the appendix in Fig.~\ref{fig:OrigZ6p52}. 
  \label{fig:OrigZ6p5}}
\end{figure}

\begin{figure}
\begin{center}

%  \includegraphics[width=0.45\textwidth]{figures/6500GeV/reweighted/cv81_EkinMuons_BG_6500GeV_flat_20MeV_bs}
%  \includegraphics[width=0.45\textwidth]{figures/6500GeV/reweighted/cv81_PhiEnMuons_BG_6500GeV_flat_20MeV_bs}
%   \includegraphics[width=0.45\textwidth]{figures/6500GeV/reweighted/cv81_RadNMuons_BG_6500GeV_flat_20MeV_bs}
  %   \includegraphics[width=0.45\textwidth]{figures/6500GeV/reweighted/cv81_RadEnMuons_BG_6500GeV_flat_20MeV_bs}
  \includegraphics[width=0.45\textwidth]{figures/6500GeV/compBGflat/ratioEkinMuons.pdf}
  \includegraphics[width=0.45\textwidth]{figures/6500GeV/compBGflat/ratioPhiEnMuons.pdf}
  \includegraphics[width=0.45\textwidth]{figures/6500GeV/compBGflat/ratioRadNMuons.pdf}
  \includegraphics[width=0.45\textwidth]{figures/6500GeV/compBGflat/ratioRadEnMuons.pdf}
\end{center}
\vspace{-0.6cm}
 \caption{Muon distributions before (black) and after (yellow) reweightening with the pressure map. The same set of plots for all particles are in the appendix, in Fig.~\ref{fig:EkinPhiEn6p52}.
  \label{fig:EkinPhiEn6p5}}
\end{figure}
