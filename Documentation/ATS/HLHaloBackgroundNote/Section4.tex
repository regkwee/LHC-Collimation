\section{Simulation Results for LHC's Run I and Run II\label{run1run2}}

The two background sources as previously described were investigated for several run scenarios. During Run I, the LHC was operated with two beam energies, 3.5 TeV in 2010--11 and 4~TeV in 2012\footnote{A detailed overview of parameters in Run I can be found in~\cite{parametersRun1}}. While the 3.5~TeV background data has been presented in~\cite{nimPaperRod}, we focus on new developments made since and use these improvements for 4 TeV and Run II simulation cases. Beam-halo simulations consider now a crossing angle in the simulation as it has been present in the machine. For beam-gas simulations we consider now the transverse extension of the beam.

\subsection{Run I: 4 TeV Beam-Halo}

Beam-halo induced showers on the tertiary collimators depend on the initial distribution of the positions at which the halo-proton interacts inelastically. For tungsten, the collimator material of the TCTs, one can expect that inelastic interactions close to the jaw surface produce more shower particles than those deeper inside when most of the inelastic processes is confined within the jaw thereby also the shower production. This depth distribution for is shown for B1 and B2 in Fig.~\ref{inel4TeV} and one can see the hits are rather close to the jaw surface with most of the hits in the first few bins (note the logarithmic scale).

\begin{figure}%[!htb]
\centering
\includegraphics[width=0.45\textwidth]{figures/inelposition_sum_impacts_real_NewScatt_4TeV_haloB1.pdf}
\includegraphics[width=0.45\textwidth]{figures/inelposition_sum_impacts_real_NewScatt_4TeV_haloB2.pdf}
 \caption{Positions of inelastic interactions of the two collimator jaws normalised to total hits in the simulation.
  \label{inel4TeV}}
\end{figure}

General, characteristic distributions of B1 halo induced showers per TCT hit are shown in Fig.~\ref{dist4TeVB1} and more can be found in the Appendix~\ref{run1run2app}. The energy spectrum of all types of charged and neutral particle species is shown in Fig.~\ref{dist4TeVB1}~(a). We note the peak at beam energy of the protons arising from single diffractive dissociations, emerging from such an interaction slightly off-energy and with an off-momentum. One can also see that in the 10 to around 500 GeV regime muons are the most dominant particle species, at low energy there are overwhelmingly many photons, followed by electrons. The azimuthal distribution of their energy as Fig.~\ref{dist4TeVB1}~(b) is dominated by protons which is clear from the single diffractive part. The energy of muons peak in the horizontal plane. The transverse radial distribtion shows that only muons reach out and most of the energy is confined to r~=~150~cm.  

\begin{figure}%[!htb]
\begin{center}
\includegraphics[width=0.49\textwidth]{figures/4TeV/haloB1_20MeV/Ekin_BH_4TeV_B1_20MeV.pdf}
\includegraphics[width=0.49\textwidth]{figures/4TeV/haloB1_20MeV/PhiEnDist_BH_4TeV_B1_20MeV.pdf}
\end{center}
\vspace{-0.6cm}
 \caption{Halo induced background by Beam 1 at the interface plane showing the energy spectrum (left) and its distribution in $\phi$.
  \label{dist4TeVB1}}
\end{figure}

Comparing B1 to B2 halo induced distributions as in Fig.~\ref{comp4TeVB1B2}, one can see all of the shapes are quite similar. The energy spectrum of all particles (a), the enery of all particles (b) of muons (c) and protons (d) in $\phi$. One can recognise that there can be more energy expected from B2 than from B1 per TCT hit. 


%% \begin{figure}[!htb]
%% \begin{center}
%% \includegraphics[width=0.411\textwidth]{figures/4TeV/Ekin_BH_4TeV_B2_20MeV.pdf}
%% \includegraphics[width=0.411\textwidth]{figures/4TeV/PhiEnDist_BH_4TeV_B2_20MeV.pdf}
%% \includegraphics[width=0.411\textwidth]{figures/4TeV/RadNDist_BH_4TeV_B2_20MeV.pdf}
%% \includegraphics[width=0.411\textwidth]{figures/4TeV/RadEnDist_BH_4TeV_B2_20MeV.pdf}
%% \end{center}
%% \begin{picture} (0.,0.)
%% \setlength{\unitlength}{1.0cm}
%% \small{
%%     \put ( 4.,7.35){(a)}
%%     \put ( 12.4,7.35){(b)}
%%     \put ( 4.,1.){(c)}
%%     \put ( 12.4,1.){(d)}}
%% \end{picture}
%% \vspace{-0.6cm}
%%  \caption{Halo induced background by Beam 2 at the interface plane. (a) shows the energy distribution, (b) is the energy in $\phi$, (c) is the radial transverse distribution $r$ and (d) the energy in $r$.
%%   \label{dist4TeVB2}}
%% \end{figure}


\begin{figure}%[!htb]
\begin{center}
\includegraphics[width=0.49\textwidth]{figures/4TeV/compB1B2/perTCThit/ratioEkinAll.pdf}
\includegraphics[width=0.49\textwidth]{figures/4TeV/compB1B2/perTCThit/ratioPhiEnAll.pdf}
\end{center}
\vspace{-0.6cm}
 \caption{Comparison of B1 and B2 halo shower distributions at the interface plane. The numbers in the ratio plot is the ratio of both integrals of the top distributions to indicate by how much nominator or denominator is larger. The error bars indicate statistical uncertainties.
  \label{comp4TeVB1B2}}
\end{figure}


% --------------------------------------------------------------------------------------------
\subsection{Run I: 4 TeV Beam-Gas}

Characteristic distributions induced by beam-gas interactions are shown in Fig.~\ref{dist4TeVBGbs}. The energy spectra, Fig.~\ref{dist4TeVBGbs}~(a), highlights a typcial shape for beam-gas interaction: in contrast to beam-halo showers it does not have the single diffractive peak at beam energy. Beam-gas induced protons show a much smoother increase up to beam-energy than in the halo case. It is still due to the same physics process, single diffractive dissociation but since there is a much less localised source of interactions on the beamline as in the halo case (where the source are the TCTs) the peak is not as pronounced as for the halos.
One immediate difference one notices is there is on average a factor 10 or more particles and energy produced by a beam-gas interaction than by a TCT interaction. One can clearly see that comparing the energy of protons at small radii, essentially the first bin of Fig.~\ref{dist4TeVB1}~(d) and Fig.~\ref{dist4TeVBGbs}~(d).


\begin{figure}%[!htb]
\centering
\includegraphics[width=0.49\textwidth]{figures/4TeV/bs_20MeV/Ekin_BG_4TeV_20MeV_bs.pdf}
\includegraphics[width=0.49\textwidth]{figures/4TeV/bs_20MeV/PhiEnDist_BG_4TeV_20MeV_bs.pdf}
 \caption{Beam-gas induced background at the interface plane. The energy spectrum (left) of different particles and their energy in $\phi$ (right) is shown.
  \label{dist4TeVBGbs}}
\end{figure}
% --------------------------------------------------------------------------------------------
\subsection{Comparison to previous techniques}

\paragraph{Transverse beam size}
We compare the simulations of the two different techniques either a pointlike size or with transverse extensions. Comparisons are made in Fig.~\ref{bsRatioPhiAll} showing $\phi$ distributions of all particles and their energy. One can see there are small differences in the particle distribution (left) while the energy in $\phi$ (right) are smoother with the new technique. Especially the peaks at $\pm \frac{\pi}{2}$ in the energy plot show clearly there is a re-distribution to the ``shoulders''.

%As the energy in $\phi$ is almost shaped by protons, we show protons and muons as they are the ones reaching far into the detector region. Indeed, there is almost the same picture for the azimuthal proton distributions shown in Fig.\ref{bsRatioPhiMP}. The effect is more pronounced for high energy protons as well in the same figure. On muons also in the figure, the new technique does not seem to have a significant impact at all.

We investigate differences in the four s-sections from Fig.~\ref{twissfileBS} to see if there is a visible difference where one could expect the beam size to be rather large. This is compared in Fig.~\ref{bsZAll} for all particles where the multiplicity (top) and energy (bottom) is shown. One remarks a few places where one has clear differences like in the last s-section, 269 to 547~m, at $\phi \approx -1$, where the ratio is going down by $60~\%$. Numerically, this is not important with about 0.0033 particles per BG interaction. The statistical uncertainties are also largest meaning there are only a small contribution from that region. The corresponding plot in the bottom row reveals that at $\phi \approx -1$ that the energy can be higher when the beam size is included by about 10~to 20~$\%$.
In the two middle s-sections, one can observe similar differences however the range is smaller with 10 to 15~$\%$ in the multiplicity and energy plot. The first section though contains the effect described visible in the distribution when integrated over all s-sections. This ``smearing'' of the peaks at $\pm \frac{\pi}{2}$ is visible in both multiplicity and energy and makes up to 60~$\%$ difference.

Looking back at Fig.~\ref{bsRatioPhiAll} one can conclude that this new technique has a slight effect only up to $5~\%$ for the mulitplicity and up to 40~$\%$ in the vertical plane. We discussed here only the differences in $\phi$, but we add other distributions also per particle types for further information in Fig.~\ref{bsRatioEkin}, Fig.~\ref{bsRatioRadN}, and Fig.~\ref{bsRatioRadEn}.

\begin{figure}%[!htb]
\begin{center}
  \includegraphics[width=0.49\textwidth]{figures/4TeV/beamsizeRatio/ratioPhiNAll.pdf}
  \includegraphics[width=0.49\textwidth]{figures/4TeV/beamsizeRatio/ratioPhiEnAll.pdf}
\end{center}
\vspace{-0.6cm}
 \caption{4 TeV azimuthal distributions of all particles and their energy.
  \label{bsRatioPhiAll}}
\end{figure}


%% \begin{figure}%[!htb]
%% \begin{center}
%%   \includegraphics[width=0.452\textwidth]{figures/4TeV/bs_20MeV/PhiEnAllZ_BG_4TeV_20MeV_bs.pdf}
%% %  \includegraphics[width=0.452\textwidth]{figures/4TeV/bs_20MeV/PhiEnProtZ_BG_4TeV_20MeV_bs.pdf}
%% \end{center}
%% %\begin{picture} (0.,0.)
%% %% \setlength{\unitlength}{1.0cm}
%% %% \small{
%% %%     \put ( 4.,2.){all}
%% %%     \put ( 12.4,2.){protons}}
%% %% \end{picture}
%%  \vspace{-0.6cm}
%%  \caption{4 TeV azimuthal energy distribution of all particles in different s-sections (see also Fig.~\ref{twissfileBS}).
%%   \label{bsZ2}}
%% \end{figure}

\begin{figure}%[!htb]
\begin{center}
  \includegraphics[width=0.24\textwidth]{figures/4TeV/beamsizeRatio/ratioPhiNAllZ1.pdf}
  \includegraphics[width=0.24\textwidth]{figures/4TeV/beamsizeRatio/ratioPhiNAllZ2.pdf}
  \includegraphics[width=0.24\textwidth]{figures/4TeV/beamsizeRatio/ratioPhiNAllZ3.pdf}
  \includegraphics[width=0.24\textwidth]{figures/4TeV/beamsizeRatio/ratioPhiNAllZ4.pdf}
  \includegraphics[width=0.24\textwidth]{figures/4TeV/beamsizeRatio/ratioPhiEnAllZ1.pdf}
  \includegraphics[width=0.24\textwidth]{figures/4TeV/beamsizeRatio/ratioPhiEnAllZ2.pdf}
  \includegraphics[width=0.24\textwidth]{figures/4TeV/beamsizeRatio/ratioPhiEnAllZ3.pdf}
  \includegraphics[width=0.24\textwidth]{figures/4TeV/beamsizeRatio/ratioPhiEnAllZ4.pdf}
\end{center}
\vspace{-0.6cm}
 \caption{Direct comparisons of azimuthal distributions of all particles (top) and their energy (bottom) in different s-sections as in Fig.~\ref{twissfileBS}.
  \label{bsZAll}}
\end{figure}

 
\paragraph{Effect of the crossing angle}

We can study the crossing angle effect on background in IR1 using the 3.5~TeV simulation data of~\cite{nimPaperRod}, thus halo and beam-gas induced, without any crossing angle included in the simulations and the 4~TeV halo data with a crossing angle of 290~$\mu$rad. For this comparison, we show the distributions at the interface plane per TCT hit. 

We find that the beam-gas data in Fig.~\ref{xingCompBG} exhibit a clear signature of the improved simulation technique including the crossing angle. There is a general accumulation in the upper part of the vertical plane; the beam is directed downwards but as the sampling position is still in the upper hemisphere the peak of particles is around $+ \frac{\pi}{2}$. Muon distributions are not significantly affected as other effects are more dominating the azimuthal distribution (like magnetic fields effects). 

In contrast, the beam-halo data in Fig.~\ref{xingCompBH} do not exhibit clear differences when comparing 3.5~TeV to 4 TeV simulations in both beams. As only effect we find that the distributions without crossing angle are slightly more symmetric in $\phi$. The reason for that is that particles of all forward directions are created at the TCTs and secondaries loose the initial direction of the crossing angle of the beam protons.

\begin{figure}
\begin{center}
  \includegraphics[width=0.41\textwidth]{figures/4TeV/compBG_3p5_vs_4TeV/ratioPhiEnAll.pdf}
  \includegraphics[width=0.41\textwidth]{figures/4TeV/compBG_3p5_vs_4TeV/ratioPhiEnPhotons.pdf}
  \includegraphics[width=0.41\textwidth]{figures/4TeV/compBG_3p5_vs_4TeV/ratioPhiEnMuons.pdf}
  \includegraphics[width=0.41\textwidth]{figures/4TeV/compBG_3p5_vs_4TeV/ratioPhiEnMuE100.pdf}
\end{center}
\vspace{-0.6cm}
 \caption{beam-gas data: Clear effect of crossing angle visible showing accumulations of particles at around $\frac{\pi}{2}$ in the vertical plane. The distribution of muons and in particular high energy muons are not significantly affected.
  \label{xingCompBG}}
\end{figure}

%% \begin{figure}
%% \begin{center}
%%   \includegraphics[width=0.41\textwidth]{figures/compBHB1_3p5vs4TeV/ratioPhiEnAll.pdf}
%%   \includegraphics[width=0.41\textwidth]{figures/compBHB1_3p5vs4TeV/ratioPhiEnPhotons.pdf}
%%   \includegraphics[width=0.41\textwidth]{figures/compBHB1_3p5vs4TeV/ratioPhiEnMuons.pdf}
%%   \includegraphics[width=0.41\textwidth]{figures/compBHB1_3p5vs4TeV/ratioPhiEnMuE100.pdf}
%% \end{center}
%% \vspace{-0.6cm}
%%  \caption{beam-halo data for B1
%%   \label{xingCompBH1}}
%% \end{figure}

\begin{figure}
\begin{center}
  \includegraphics[width=0.41\textwidth]{figures/compBHB1_3p5vs4TeV/ratioPhiEnAll.pdf}
  \includegraphics[width=0.41\textwidth]{figures/compBHB1_3p5vs4TeV/ratioPhiEnPhotons.pdf}
  \includegraphics[width=0.41\textwidth]{figures/4TeV/compB2_3p5vs4TeV/ratioPhiEnAll.pdf}
  \includegraphics[width=0.41\textwidth]{figures/4TeV/compB2_3p5vs4TeV/ratioPhiEnPhotons.pdf}
%  \includegraphics[width=0.41\textwidth]{figures/4TeV/compB2_3p5vs4TeV/ratioPhiEnMuons.pdf}
%  \includegraphics[width=0.41\textwidth]{figures/4TeV/compB2_3p5vs4TeV/ratioPhiEnMuE100.pdf}
\end{center}
\vspace{-0.6cm}
\caption{B1 (top) and B2 (bottom) beam-halo data. No clear sign of a crossing-angle effect visible.
%   for b2 Energy enhancement between $-\frac{\pi}{2}$ and $-\pi$, enhancement of photons in horizontal plane?? More muons around $-\frac{\pi}{2}$?
  \label{xingCompBH}}
\end{figure}


\subsection{Run I: Renormalisation of beam gas events with pressure profile}

\begin{table}
   \centering
   \caption{LHC fill 2736 Run I (2012)~\cite{refAccStats}}
   \begin{tabular}{l||c}
       \hline
       beam energy  & 4~TeV \\
       Fill start time (local Geneva time) & 16/06/2012 18:22\\
       Stable beam start time (local Geneva time) & 16/06/2012 20:10\\
       Stable beam duration [hh:mm] & 17:29\\
       bunch intensity ring 1& 2.029$\times 10^{14}$ protons\\
       bunch intensity ring 2& 1.995$\times 10^{14}$ protons\\
       number of bunches & 1374 \\
       \hline
   \end{tabular}
   \label{tab:fillRunI}
\end{table}

The information of the original inelastic proton interaction of each particle at the interface plane was kept in order to attribute weights according to a typical pressure map. For 2012, the simulations of the gas densities are based on gauge data recorded in LHC fill 2736 and were performed using VASCO (VAcuum Stability COde)~\cite{vascoRef}. The pressures were simulated up to the long straight section of IR1 (LSS1) for the most dominant molecules, H$_2$, CH$_4$, CO, CO$_2$. Partial pressure euqivalents expressed in molcular densities are shown in the top of Fig.~\ref{pressure2012} and are decomposed into atomic components at the bottom of that figure to show the interaction probability as function of the location using the inelastic cross-sections for H, C and O as indicated in Tab.~\ref{tab:atomicXsections} for a 4 TeV beam.

As described in the previous section, the weights are calculated using Eq.~\ref{eq3}. The total number of protons is given by the maximum beam-intensity in LHC fill 2736 which was about $2.04 \times 10^{14}$ for B1 and $2.07 \times 10^{14}$ for B2. The method is shown in Fig.~\ref{fig:method}: in green the total interaction probability rate up to the arc (the pressure has been extended to the arc by taking closest value as constant), in black the number of muons per beam-gas interaction and in red the multiplication of both scaled to the maximum beam intensity of that fill. One can see the result is mainly shaped by the pressure map, apart from the part in the arc (beyond s~=~260~m) where the shape comes from the shower simulations. 

We compare in detail how the distributions change when normalising from a pressure that corresponds to a flat to a non-flat profile. As a function in $s$ we show where particles are created in Fig.~\ref{fig:OrigZ}. To be able to compare better, the distributions corresponding to a flat profile are scaled up by seven orders of magnitude to match the scale of the rate of what is shown on the y-axis, in Fig.~\ref{fig:cv81EkinPhiEn4TeV} it is the particle counts per energy bin, the energy in $\phi$, transverse radial distributions and the energy of that for all particles and for muons. Another figure, Fig.~\ref{fig:cv81ProtNeut4TeV}, shows the same observables but for protons and neutrons.

We can observe that the integrated distributions at the interface plane change only slightly. As one can see from the distributions in Fig.~\ref{fig:OrigZ4TeV}, the shapes changes are mainly driven by the pressure in the long straight section (up to 260~m). Note that in the shown $\phi-$distributions the y-scale comprises only two orders of magnitude while in the others five and up to 17 orders of magnitude are shown.

\begin{figure}%[!htb]
\begin{center}
  \includegraphics[width=0.75\textwidth]{figures/4TeV/LSS1_B1_Fill2736_Finala_pressure.pdf}
  \includegraphics[width=0.75\textwidth]{figures/4TeV/reweighted/cv65_pint.pdf}
\end{center}
\vspace{-0.6cm}
 \caption{Gas densities in LSS1 shown for the most common molecules (top) and split into atomic components to indicate the interaction probability (bottom).
  \label{pressure2012}}
\end{figure}

\begin{figure}%[!htb]
\begin{center}
  \includegraphics[width=0.95\textwidth]{figures/4TeV/reweighted/muons2012.pdf}
\end{center}
\vspace{-0.6cm}
 \caption{Origin of muon production normalised to the pressure shown in Fig.~\ref{pressure2012}. 
  \label{fig:method}}
\end{figure}


\begin{figure}
\begin{center}
  \includegraphics[width=0.75\textwidth]{figures/4TeV/reweighted/cv81_OrigZAll_BG_4TeV_20MeV_bs}
%  \includegraphics[width=0.75\textwidth]{figures/4TeV/reweighted/cv81_OrigZMuon_BG_4TeV_20MeV_bs}
%  \includegraphics[width=0.75\textwidth]{figures/4TeV/reweighted/cv81_OrigZProtons_BG_4TeV_20MeV_bs}
\end{center}
\vspace{-0.6cm}
 \caption{Origin of all particles production along s, in black before and in pink after re-normalising to the 2012 pressure profile. The spike in the top plot at s~=~520~m is due to local photon production (see middle plot in Fig.~\ref{fig:OrigZ4TeV2}).
  \label{fig:OrigZ4TeV}}
\end{figure}

\begin{figure}
\begin{center}
  \includegraphics[width=0.45\textwidth]{figures/4TeV/reweighted/cv81_EkinAll_BG_4TeV_20MeV_bs}
  \includegraphics[width=0.45\textwidth]{figures/4TeV/reweighted/cv81_EkinMuons_BG_4TeV_20MeV_bs}
  \includegraphics[width=0.45\textwidth]{figures/4TeV/reweighted/cv81_PhiEnAll_BG_4TeV_20MeV_bs}
  \includegraphics[width=0.45\textwidth]{figures/4TeV/reweighted/cv81_PhiEnMuons_BG_4TeV_20MeV_bs}
  %% \includegraphics[width=0.45\textwidth]{figures/4TeV/reweighted/cv81_RadNAll_BG_4TeV_20MeV_bs}
  %% \includegraphics[width=0.45\textwidth]{figures/4TeV/reweighted/cv81_RadNMuons_BG_4TeV_20MeV_bs}
  %% \includegraphics[width=0.45\textwidth]{figures/4TeV/reweighted/cv81_RadEnAll_BG_4TeV_20MeV_bs}
  %% \includegraphics[width=0.45\textwidth]{figures/4TeV/reweighted/cv81_RadEnMuons_BG_4TeV_20MeV_bs}
\end{center}
\vspace{-0.6cm}
 \caption{From top to bottom for all particles (left) and muons (right): energy spectrum, energy in $\phi$, transverse radius $r$, and energy in $r$ before reweightening (black) and reweighted to pressure profile. The black curve are scaled up to better compare the shapes. 
  \label{fig:cv81EkinPhiEn4TeV}} 
\end{figure}


\newpage
% --------------------------------------------------------------------------------------------
\subsection{Run I: Off-momentum leakage and shower simulations at 4 TeV}

Analogue to IR7 leakage to the experimental IRs of ATLAS and CMS, one can now investigate background contributions from IR3 off-momentum cleaning using the losses leaking to IR1 and IR5 TCTs, also expressed in TCP-to-TCT conversion factors. They are listed in Tab.~\ref{tab:IR3leakageFactors} for IR1 and IR5 for all the simulated cases.

We also note something interesting from Tab.~\ref{tab:IR3leakageFactors} when comparing the case ``+~500~Hz'' with ``-~500~Hz'': At 4~TeV, the negative fequency shift causes only half of the leakage to IR1 with $0.012$ and $0.0063$. The same is true for IR5: The losses are reduced from $0.045$ to $0.025$.

We show how deep in the jaws the inelastic interactions take place in Fig.~\ref{inel4TeVOffmom}. These distributions show a clear difference to those leaking from IR7: First, we note the full contribution for IR1 comes from B2, there is only a negligable fraction of B1 protons hitting the TCTs. Vice versa for IR5, almost nothing ends up on the TCTs in IR5 from B2 and contributions from IR3 leakage is due to B1 only. Then we notice, the distributions go much further into the jaw compared to e.g.~\ref{inel4TeV} for both beams in the ``minus-case'' but also for B1 in the ``plus-case''. B2 ``plus-case'' has the intercepted protons more on the surface, most of them up to around 5~mm which is more similar to the halo. 

We use the B2 TCT seeds in the shower simulations for IR1. General distributions for the main particle species are shown for a negative shift in Fig.~\ref{offmom4TeV}. They have quite some familiarities with halo induced showers. More characteristic distributions of the ``plus-case'' and a detailed comparison of the two cases are shown in the appendix in Fig.~\ref{offmom4TeV2}, Fig.~\ref{compPM_ekin} and Fig.~\ref{compPM_phien}.

We observe per TCT interaction from Fig.~\ref{comPM} one can be expect about 20~$\%$ more particles carrying about 30$~\%$ more energy when particles experience a negative than a positive shift.


\begin{figure}
\begin{center}
\includegraphics[width=0.49\textwidth]{figures/inelposition_sum_impacts_real_4TeV_plus500Hz_TCT_B1.pdf}
\includegraphics[width=0.49\textwidth]{figures/inelposition_sum_impacts_real_4TeV_plus500Hz_TCT_B2.pdf}
\includegraphics[width=0.49\textwidth]{figures/inelposition_sum_impacts_real_minus500Hz_4TeVB1.pdf}
\includegraphics[width=0.49\textwidth]{figures/inelposition_sum_impacts_real_minus500Hz_4TeVB2.pdf}
\end{center}
\vspace{-0.6cm}
 \caption{Positions of inelastic interactions as given by SixTrack within the collimator jaws.
  \label{inel4TeVOffmom}}
\end{figure}


\begin{figure}
\begin{center}
  \includegraphics[width=0.49\textwidth]{figures/4TeV/offmom/20MeV/Ekin_offmin500Hz_4TeV_B2_20MeV.pdf}
  \includegraphics[width=0.49\textwidth]{figures/4TeV/offmom/20MeV/PhiEnDist_offmin500Hz_4TeV_B2_20MeV.pdf}
\end{center}
\vspace{-0.6cm}
 \caption{Off-momentum induced particle distributions.
  \label{offmom4TeV}}
\end{figure}

\begin{figure}
  \centering
  \includegraphics[width=0.49\textwidth]{figures/4TeV/offmom/comppm500Hz/ratioEkinAll.pdf}
  \includegraphics[width=0.49\textwidth]{figures/4TeV/offmom/comppm500Hz/ratioPhiEnAll.pdf}
  \caption{Direct comparison of the energy spectrum of all particles (left) and azimuthal distribution of energy (right).
    \label{compPM}}
\end{figure}

% --------------------------------------------------------------------------------------------
\subsection{Run II: 6.5 TeV Beam-Halo}

The same method is used as in the 4 TeV simulations changing to the Run II 2015 scenario, with $\beta^* = 80$~cm, see also Tab.~\ref{paramsRun12}. The input distributions are shown for B1 and B2 in Fig.~\ref{inel6.5}. One can already see that the B1 depth distribution in the TCTV is much shallower than at 4~TeV in Fig.~\ref{inel4TeV} most of the hits are in the very first bin. This is very likely due to the different collimator settings and optics. Characteristic distributions are shown in Fig.~\ref{dist6500GeVB2}, and a comparison to B1 distributions are in Fig.~\ref{compBHB1B2run2}. One can see that in contrast to 4 TeV data, B1 produces more shower particles and energy than B2 per interaction in the TCTs. 


\begin{figure}[!htb]
\begin{center}
\includegraphics[width=0.4\textwidth]{figures/inelposition_sum_HALOB1.pdf}
\includegraphics[width=0.4\textwidth]{figures/inelposition_sum_HALOB2.pdf}
\end{center}
 \caption{Depth of the inelastic interaction of the halo hits with the collimator material for B1 and B2 6.5 TeV hits.
  \label{inel6.5}}
\end{figure}


\begin{figure}%[!htb]
\centering
\includegraphics[width=0.49\textwidth]{figures/BH_run2/b2/Ekin_BH_6500GeV_haloB2_20MeV.pdf}
\includegraphics[width=0.49\textwidth]{figures/BH_run2/b2/PhiEnDist_BH_6500GeV_haloB2_20MeV.pdf}

 \caption{B2 halo induced background at the interface plane. The distributions exhibit the similar features as at 4 TeV.
  \label{dist6500GeVB2}}
\end{figure}


\begin{figure}%[!htb]
\centering
  \includegraphics[width=0.411\textwidth]{figures/BH_run2/perTCThit/ratioEkinAll.pdf}
  \includegraphics[width=0.411\textwidth]{figures/BH_run2/perTCThit/ratioPhiEnAll.pdf}

 \caption{Comparison of B1/B2 halo induced distributions per TCT hit.
  \label{compBHB1B2run2}}
\end{figure}



\subsection{Run II: 6.5 TeV Beam-Gas}

This case has been simulated with the same method as for the Run I scenario, using the extension of the beam size. The beam optics and collimator settings were different in the 2015 Run II, as detailed in Tab.~\ref{paramsRun12}. Characteristic distributions are shown in Fig.~\ref{bg6500} and some more are in the appendix in Fig.~\ref{bg65002}.

\begin{figure}%[!htb]
\begin{center}
  \includegraphics[width=0.49\textwidth]{figures/6500GeV/20MeV/Ekin_BG_6500GeV_flat_20MeV_bs.pdf}
  \includegraphics[width=0.49\textwidth]{figures/6500GeV/20MeV/PhiEnDist_BG_6500GeV_flat_20MeV_bs.pdf}
%  \includegraphics[width=0.49\textwidth]{figures/6500GeV/20MeV/RadNDist_BG_6500GeV_flat_20MeV_bs.pdf}
%  \includegraphics[width=0.49\textwidth]{figures/6500GeV/20MeV/RadEnDist_BG_6500GeV_flat_20MeV_bs.pdf}
\end{center}
\vspace{-0.6cm}
 \caption{Characteristic beam-gas induced distributions at 6.5~TeV per BG interaction using the more realistic model of the beam size.
  \label{bg6500}}
\end{figure}

\subsubsection{Run II: Renormalisation of beam gas events with pressure profile}

Similar to Run I in 2012, the four LHC experiments agreed on a LHC fill number where typical conditions for physics data taking were present. This fill, number 4536, was used by the LHC Vacuum group to simulate the pressure map. Fill details are summed up in Tab.~\ref{tab:fillRunII}. 

\begin{table}
   \centering
   \caption{LHC fill 4536 Run II (2015)~\cite{refAccStats}}
   \begin{tabular}{l||c}
       \hline
       beam energy  & 6.5~TeV \\
       Fill start time (local Geneva time) & 26/10/2015 02:59\\
       Stable beam start time (local Geneva time) & 26/10/2015 10:26\\
       Stable beam duration [hh:mm] & 01:04\\
       bunch intensity ring 1& 2.2899$\times 10^{14}$ protons\\
       bunch intensity ring 2& 2.2137$\times 10^{14}$ protons\\
       number of bunches & 2041 \\
       \hline
   \end{tabular}
   \label{tab:fillRunII}
\end{table}

\paragraph{\textit{Pressure Map Simulations}}
In previous years (in particular for 3.5~TeV and 4~TeV runs data), the code to simulate the pressure map, VASCO, was used. For Run II in 2015, pressures were simulated with an improved version of the code. The code has been re-implemented in \textsc{Python} and modifications were made to automise pressure simulations\cite{christinasStudent} including the describtion of photon desorption (determined by SynRad) and electron cloud effects (as determined in PyECLOUD~\cite{giovanniPhd}). Original algorithms were kept however a few assumptions had to be made to be able to simulate pressures around the entire ring. The variety of materials and pumping speeds were simplified. However, the pressure shown in Fig.~\ref{pressure2015} is meant to be rather on the lower limit of gas densities~\cite{christinaPriv}. 

\paragraph{\textit{Reweighted Results using 2015 pressure}}

The pressure simulation data based on Fill 4536 is shown in top of Fig.~\ref{pressure2015} for each gas type, while the bottom figure shows the interaction probability per second as in Eq.~\ref{eq2} and what is used to reweight the distributions at the interface plane.

\begin{figure}
\begin{center}
  \includegraphics[width=0.95\textwidth]{figures/6500GeV/reweighted/Density_Fill4536_2041b_26158_B1_withECLOUD_rho.pdf}
  \includegraphics[width=0.95\textwidth]{figures/6500GeV/reweighted/pint_compallpint.pdf}
\end{center}
\vspace{-0.6cm}
 \caption{Gas densities up to the arc for the most common molecules (top) and split into atomic components to indicate the interaction probability (bottom).
  \label{pressure2015}}
\end{figure}

\begin{figure}
\begin{center}
%  \includegraphics[width=0.75\textwidth]{figures/6500GeV/reweighted/cv81_OrigZAll_BG_6500GeV_flat_20MeV_bs.pdf}
  \includegraphics[width=0.75\textwidth]{figures/6500GeV/reweighted/cv81_OrigZMuon_BG_6500GeV_flat_20MeV_bs.pdf}
%  \includegraphics[width=0.75\textwidth]{figures/6500GeV/reweighted/cv81_OrigZProtons_BG_6500GeV_flat_20MeV_bs.pdf}
\end{center}
\vspace{-0.6cm}
 \caption{Origin of all particles (top), muons (middle) and protons (bottom) production along s, in black before and in pink after re-normalising to the 2015 pressure profile. 
  \label{fig:OrigZ6p5}}
\end{figure}

\begin{figure}
\begin{center}
  \includegraphics[width=0.45\textwidth]{figures/6500GeV/reweighted/cv81_EkinAll_BG_6500GeV_flat_20MeV_bs}
  \includegraphics[width=0.45\textwidth]{figures/6500GeV/reweighted/cv81_EkinMuons_BG_6500GeV_flat_20MeV_bs}
  \includegraphics[width=0.45\textwidth]{figures/6500GeV/reweighted/cv81_PhiEnAll_BG_6500GeV_flat_20MeV_bs}
  \includegraphics[width=0.45\textwidth]{figures/6500GeV/reweighted/cv81_PhiEnMuons_BG_6500GeV_flat_20MeV_bs}
  %% \includegraphics[width=0.45\textwidth]{figures/6500GeV/reweighted/cv81_RadNAll_BG_6500GeV_flat_20MeV_bs}
  %% \includegraphics[width=0.45\textwidth]{figures/6500GeV/reweighted/cv81_RadNMuons_BG_6500GeV_flat_20MeV_bs}
  %% \includegraphics[width=0.45\textwidth]{figures/6500GeV/reweighted/cv81_RadEnAll_BG_6500GeV_flat_20MeV_bs}
  %% \includegraphics[width=0.45\textwidth]{figures/6500GeV/reweighted/cv81_RadEnMuons_BG_6500GeV_flat_20MeV_bs}
\end{center}
\vspace{-0.6cm}
 \caption{The distributions are normalised to 1 to better compare the shapes for all particles (left) and muons (right).
  \label{fig:EkinPhiEn6p5}}
\end{figure}
