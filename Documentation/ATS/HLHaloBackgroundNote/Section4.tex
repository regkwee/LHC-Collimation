\section{Simulation Results for LHC's Run I and Run II\label{run1run2}}

The two background sources as previously described were investigated for several run scenarios. During Run I, the LHC was operated with two beam energies, 3.5 TeV in 2010--11 and 4~TeV in 2012\footnote{A detailed overview of parameters in Run I can be found in~\cite{parametersRun1}}. While the 3.5~TeV background data has been presented in~\cite{nimPaperRod}, we focus on new developments made since that paper and use these improvements for 4 TeV and Run II simulation cases. Beam-halo simulations consider now a crossing angle in the simulation as it has been present in the machine. Another improvement concerns beam-gas simulations for which the transverse beam size is taken into account to have the simulations closer to reality. First general characteristics of both background sources at an virtual plane are presented.

\subsection{Run I: 4 TeV Beam-Halo}

Beam-halo induced showers on the tertiary collimators depend on the initial distribution of the positions at which the halo-proton interacts inelastically. For tungsten, the collimator material of the TCTs, one can expect that inelastic interactions close to the jaw surface produce more shower particles than those deeper inside when most of the inelastic processes is confined within the jaw thereby also the shower production. This depth distribution for is shown for B1 and B2 in Fig.~\ref{inel4TeV} and one can see the hits are rather close to the jaw surface with most of the hits in the first few bins (note the logarithmic scale).

\begin{figure}[!htb]
\begin{center}
\includegraphics[width=0.45\textwidth]{figures/inelposition_sum_impacts_real_NewScatt_4TeV_haloB1.pdf}
\includegraphics[width=0.45\textwidth]{figures/inelposition_sum_impacts_real_NewScatt_4TeV_haloB2.pdf}
\end{center}
\begin{picture} (0.,0.)
\setlength{\unitlength}{1.0cm}
\small{
    \put ( 4.,1.){(a)}
    \put ( 12.4,1.){(b)}
}
\end{picture}
\vspace{-0.6cm}
 \caption{Positions of inelastic interactions of the two collimator jaws normalised to total hits in the simulation.
  \label{inel4TeV}}
\end{figure}

General characteristic distributions of B1 halo induced showers per TCT hit are shown in Fig.~\ref{dist4TeVB1}. The energy spectrum of all types of charged and neutral particle species is shown in Fig.~\ref{dist4TeVB1}~(a). We note the peak at beam energy of the protons arising from single diffractive dissociations, these are almost unperturbed beam particles loosing very little energy only. One can also see that in the 10 to around 500 GeV regime muons are the most dominant particle species, at low energy there are overwhelmingly many photons, followed by electrons. The azimuthal distribution of their energy as Fig.~\ref{dist4TeVB1}~(b) is dominated by protons which is clear from the single diffractive part. The energy of muons peak in the horizontal plane. The transverse radial distribtion shows that only muons reach out and most of the energy is confined to r~=~150~cm.  
\begin{figure}[!htb]
\begin{center}
\includegraphics[width=0.49\textwidth]{figures/4TeV/Ekin_BH_4TeV_B1_20MeV.pdf}
\includegraphics[width=0.49\textwidth]{figures/4TeV/PhiEnDist_BH_4TeV_B1_20MeV.pdf}
\includegraphics[width=0.49\textwidth]{figures/4TeV/RadNDist_BH_4TeV_B1_20MeV.pdf}
\includegraphics[width=0.49\textwidth]{figures/4TeV/RadEnDist_BH_4TeV_B1_20MeV.pdf}
%\includegraphics[width=0.49\textwidth]{figures/4TeV/XYNMuons_BH_4TeV_B1_20MeV.pdf} %% EMPTY PLOT
\end{center}
\begin{picture} (0.,0.)
\setlength{\unitlength}{1.0cm}
\small{
    \put ( 4.,7.35){(a)}
    \put ( 12.4,7.35){(b)}
    \put ( 4.,1.){(c)}
    \put ( 12.4,1.){(d)}}
\end{picture}
\vspace{-0.6cm}
 \caption{Halo induced background by Beam 1 at the interface plane. (a) shows the energy distribution, (b) is the energy in $\phi$, (c) is the radial transverse distribution $r$ and (d) the energy in $r$.
  \label{dist4TeVB1}}
\end{figure}

Comparing B1 to B2 halo induced distributions as in Fig.~\ref{comp4TeVB1B2}, one can see all of the shapes are quite similar. The energy spectrum of all particles (a), the enery of all particles (b) of muons (c) and protons (d) in $\phi$. One can recognise that there can be more energy expected from B2 than from B1 per beam-gas interaction. Since the normalisation to an absolute scale (not shown here) is the same for B1 and B2, simulation predicts a higher rate from B2 as well.
%% \begin{figure}[!htb]
%% \begin{center}
%% \includegraphics[width=0.411\textwidth]{figures/4TeV/Ekin_BH_4TeV_B2_20MeV.pdf}
%% \includegraphics[width=0.411\textwidth]{figures/4TeV/PhiEnDist_BH_4TeV_B2_20MeV.pdf}
%% \includegraphics[width=0.411\textwidth]{figures/4TeV/RadNDist_BH_4TeV_B2_20MeV.pdf}
%% \includegraphics[width=0.411\textwidth]{figures/4TeV/RadEnDist_BH_4TeV_B2_20MeV.pdf}
%% \end{center}
%% \begin{picture} (0.,0.)
%% \setlength{\unitlength}{1.0cm}
%% \small{
%%     \put ( 4.,7.35){(a)}
%%     \put ( 12.4,7.35){(b)}
%%     \put ( 4.,1.){(c)}
%%     \put ( 12.4,1.){(d)}}
%% \end{picture}
%% \vspace{-0.6cm}
%%  \caption{Halo induced background by Beam 2 at the interface plane. (a) shows the energy distribution, (b) is the energy in $\phi$, (c) is the radial transverse distribution $r$ and (d) the energy in $r$.
%%   \label{dist4TeVB2}}
%% \end{figure}


\begin{figure}[!htb]
\begin{center}
\includegraphics[width=0.4\textwidth]{figures/4TeV/compB1B2/perTCThit/ratioEkinAll.pdf}
%\includegraphics[width=0.4\textwidth]{figures/4TeV/compB1B2/perTCThit/ratioEkinMuons.pdf}
\includegraphics[width=0.4\textwidth]{figures/4TeV/compB1B2/perTCThit/ratioPhiEnAll.pdf}
\includegraphics[width=0.4\textwidth]{figures/4TeV/compB1B2/perTCThit/ratioPhiEnMuons.pdf}
\includegraphics[width=0.4\textwidth]{figures/4TeV/compB1B2/perTCThit/ratioPhiEnProtons.pdf}
\end{center}
\begin{picture} (0.,0.)
\setlength{\unitlength}{1.0cm}
\small{
    \put ( 4.,9.05){(a)}
    \put ( 12.4,9.05){(b)}
    \put ( 4.,1.){(c)}
    \put ( 12.4,1.){(d)}}
\end{picture}
\vspace{-0.6cm}
 \caption{Comparison of B1 and B2 halo shower distributions at the interface plane. The numbers in the ratio plot is the integral of the ratio curve and indicates by how much nominator or denominator is larger. The error bars indicate statistical uncertainties only.
  \label{comp4TeVB1B2}}
\end{figure}


% --------------------------------------------------------------------------------------------
\subsection{Run I: 4 TeV Beam-Gas}

Characteristic distributions induced by beam-gas interactions are shown in Fig.~\ref{dist4TeVBGbs}. The energy spectra, Fig.~\ref{dist4TeVBGbs}~(a), highlights a typcial shape for beam-gas interaction: in contrast to beam-halo showers it does not have the single diffractive peak at beam energy. Beam-gas induced protons show a much smoother increase up to beam-energy than in the halo case. It is still due to the same physics process, single diffractive dissociation but since there is a much less localised source of interactions on the beamline as in the halo case (where the source are the TCTs) the peak is not as pronounced as for the halos.
One immediate difference one notices is there is on average a factor 10 or more particles and energy produced by a beam-gas interaction than by a TCT interaction. One can clearly see that comparing the energy of protons at small radii, essentially the first bin of Fig.~\ref{dist4TeVB1}~(d) and Fig.~\ref{dist4TeVBGbs}~(d).



\begin{figure}[!htb]
\begin{center}
\includegraphics[width=0.49\textwidth]{figures/4TeV/bs_20MeV/Ekin_BG_4TeV_20MeV_bs.pdf}
\includegraphics[width=0.49\textwidth]{figures/4TeV/bs_20MeV/PhiEnDist_BG_4TeV_20MeV_bs.pdf}
\includegraphics[width=0.49\textwidth]{figures/4TeV/bs_20MeV/RadNDist_BG_4TeV_20MeV_bs.pdf}
\includegraphics[width=0.49\textwidth]{figures/4TeV/bs_20MeV/RadEnDist_BG_4TeV_20MeV_bs.pdf}
\end{center}
\begin{picture} (0.,0.)
\setlength{\unitlength}{1.0cm}
\small{
    \put ( 4.,7.35){(a)}
    \put ( 12.4,7.35){(b)}
    \put ( 4.,1.){(c)}
    \put ( 12.4,1.){(d)}}
\end{picture}
\vspace{-0.6cm}
 \caption{Beam-gas induced background at the interface plane. (a) shows the energy distribution, (b) is the energy in $\phi$, (c) is the transverse radial distribution $r$ and (d) the energy in $r$.
  \label{dist4TeVBGbs}}
\end{figure}
% --------------------------------------------------------------------------------------------
\subsection{New simulation techniques}
Previous studies as in Ref.~\cite{nimPaperRod} used methods relaying on approximations of either of the beam or on its trajectory. One approximation is that the transverse beam size was neglected. In particular, just before the triplet the beamsize is very large and addtional interactions in that location could contribute to shower particles at the interface plane. We investigate how the additional improvements, i.e.~the inclusion of the crossing angle and beam size, effect previous simulations. 

\subsubsection{Setup to include the beam size and comparison \label{beamgasBS}}
Two cases were simulated in \fluka, with exactly the same setup for the 2012 Run I scenario, see Table~\ref{paramsRun12}, but using a different input file for positions at which beam-gas interactions are sampled.

The new input file was created by dumping positions of the trajectory with different starting positions. The ideal orbit goes through (0,0) in (x,y) at the IP. Assuming a gaussian distribution of the beam particles one can produce matched phase space coordinates in the transverse plane, as shown in Fig.~\ref{ip1_gauss}, at the IP where the optical functions are $\alpha = 0$, $\beta = \beta^* = 60$~cm. Using a normalised coordinate system, the phase space coordinates were calculated as in Eq.~\ref{eq1} and used as initial seeds in \fluka~to create the trajectory. 1000 trajectories were created and randomly 10 sampling positions per longitudinal coordinate were chosen. The final input file to \fluka~is visualised in Fig.~\ref{BGASflukaInp}. One would expect the beam size to be largest after the D1 at the entrance of the triplet in Fig.~\ref{ip1_gauss}~(a). Looking back to Fig.~\ref{nominalLHC_layout} we identify first a squeeze in x (red bottom box), then in y (purple top box) and then in x again and indeed this is also what we see if the beamsize is calculated theoretically using the MadX $\beta-$functions\footnote{given by $\sigma_{x,y} = \sqrt{\epsilon_{geo} \cdot \beta_{x,y}}$ with $\epsilon_{\textrm{geo}} = \frac{ \epsilon_{\textrm{n}}}{\gamma_{\textrm{rel}}}$} as shown in Fig.~\ref{twissfileBS}.

\begin{equation} \label{eq1}
  \begin{split}
x = & \, \sqrt{\beta \epsilon} \cdot X \\
x' = & \sqrt{\frac{\epsilon}{\beta}} \cdot X' - \frac{\alpha X}{\sqrt{\beta \epsilon}}
  \end{split}
\end{equation}

with $\epsilon$ being the emittance, $\alpha, \beta$ and $\gamma$ the usual twiss parameters from the definition of the emittance as conservative in $\epsilon = \gamma x^2 + \beta x'^2 + 2 \alpha x x'$, and $X$ and $X'$ satisfying the circle equation, $X^2 + X'^2 = 1$. 


\begin{figure}%[!htb]
\begin{center}
\includegraphics[width=0.9\textwidth]{figures/IP1_gauss.pdf}
\includegraphics[width=0.9\textwidth]{figures/twiss_gauss.pdf}
\end{center}
%% \begin{picture} (0.,0.)
%% \setlength{\unitlength}{1.0cm}
%% \small{
%%     \put ( 4.,7.35){(a)}
%%     \put ( 12.4,7.35){(b)}
%%     \put ( 4.,1.){(c)}
%%     \put ( 12.4,1.){(d)}}
%% \end{picture}
\vspace{-0.6cm}
 \caption{Matched phase space coordinates at IP1 in x and y (top) and at an example position (bottom, here TCTH). The rings indicate in $\sigma$ the gaussian distribution.
  \label{ip1_gauss}}
\end{figure}


\begin{figure}[!htb]
\begin{center}
\includegraphics[width=0.44\textwidth]{figures/inputFluka6500GeV_yBGAS.pdf}
\includegraphics[width=0.44\textwidth]{figures/xBGAS10z1.pdf}
\end{center}
\begin{picture} (0.,0.)
\setlength{\unitlength}{1.0cm}
\small{
    \put ( 4.,1.){(a)}
    \put ( 12.4,1.){(b)}
}
\end{picture}
\vspace{-0.6cm}
 \caption{Positions as sampled in \fluka~with variations of the transverse beam size vertically (a) shown for full range until the arc and horizontally (b) shown for the first 85~m from the IP without arc.
  \label{BGASflukaInp}}
\end{figure}

%% %Muons, Protons
A global view on particle distributions in $\phi$ shows that the former peaks, in particular when looking at Fig.~\ref{bsRatioPhiAll} in the vertical plane at $\pm\frac{\pi}{2}$ is ``washed out'' but else no major differences can be observed. 

The distributions per species, muons and protons, are shown in Fig.~\ref{bsRatioPhiMP} and one can observe clearly in the proton distributions that the shoulders are a little wider and instead the peak is not as high as when no beam size was taken into account. For muons one can conclude it does not change anything significantly.

Having a closer look to proton distributions we investigate if an increase is visible where the beam size is indeed large just downstram the interface plane, from 22.6 to 59~m. Comparing the shapes per z-region of all particles and protons only one can see most of the energy comes from protons as visible from Fig.~\ref{bsZ2}. A direct comparison of the shapes in Fig.~\ref{bsZ} reveals all differences are almost entirely due to protons.

\begin{figure}%[!htb]
\begin{center}
  \includegraphics[width=0.85\textwidth]{figures/twiss_b1_sigma_IR1Right_4TeV.pdf}
\end{center}
\vspace{-0.6cm}
 \caption{Beam sizes in IR1 in horizontal and vertical plane. The red lines indicate the longitudinal s-sections: from 22.6~m to 59~m it contains triplet, 59~m to 153~m is where the beampipes are split and the D1 sits, in 153--269~m is the D2 and goes up to the end of the LSS1 and at 269~m the arc starts and is shown up to 550~m.
  \label{twissfileBS}}
\end{figure}


\begin{figure}%[!htb]
\begin{center}
  \includegraphics[width=0.411\textwidth]{figures/4TeV/beamsizeRatio/ratioPhiNAll.pdf}
  \includegraphics[width=0.411\textwidth]{figures/4TeV/beamsizeRatio/ratioPhiEnAll.pdf}
\end{center}
\vspace{-0.6cm}
 \caption{4 TeV azimuthal distributions of all particles and their energy.
  \label{bsRatioPhiAll}}
\end{figure}

\begin{figure}%[!htb]
\begin{center}
  \includegraphics[width=0.411\textwidth]{figures/4TeV/beamsizeRatio/ratioPhiNProtonsE100.pdf}
  \includegraphics[width=0.411\textwidth]{figures/4TeV/beamsizeRatio/ratioPhiEnProtons.pdf}
  \includegraphics[width=0.411\textwidth]{figures/4TeV/beamsizeRatio/ratioPhiNMuonsE100.pdf}
  \includegraphics[width=0.411\textwidth]{figures/4TeV/beamsizeRatio/ratioPhiEnMuons.pdf}
\end{center}
\vspace{-0.6cm}
 \caption{4 TeV azimuthal distributions of multiplicity of high energy protons (top) and muons (bottom) and all proton and muon energies.
  \label{bsRatioPhiMP}}
\end{figure}

\begin{figure}%[!htb]
\begin{center}
  \includegraphics[width=0.452\textwidth]{figures/4TeV/bs_20MeV/PhiEnAllZ_BG_4TeV_20MeV_bs.pdf}
  \includegraphics[width=0.452\textwidth]{figures/4TeV/bs_20MeV/PhiEnProtZ_BG_4TeV_20MeV_bs.pdf}
\end{center}
\begin{picture} (0.,0.)
\setlength{\unitlength}{1.0cm}
\small{
    \put ( 4.,2.){all}
    \put ( 12.4,2.){protons}}
\end{picture}
\vspace{-0.6cm}
 \caption{4 TeV azimuthal energy distribution of all particles (left) and protons (right) in different sections as depicted in Fig.~\ref{twissfileBS}.
  \label{bsZ2}}
\end{figure}

\begin{figure}%[!htb]
\begin{center}
  \includegraphics[width=0.24\textwidth]{figures/4TeV/beamsizeRatio/ratioPhiEnAllZ1.pdf}
  \includegraphics[width=0.24\textwidth]{figures/4TeV/beamsizeRatio/ratioPhiEnAllZ2.pdf}
  \includegraphics[width=0.24\textwidth]{figures/4TeV/beamsizeRatio/ratioPhiEnAllZ3.pdf}
  \includegraphics[width=0.24\textwidth]{figures/4TeV/beamsizeRatio/ratioPhiEnAllZ4.pdf}
  \includegraphics[width=0.24\textwidth]{figures/4TeV/beamsizeRatio/ratioPhiEnPrZ1.pdf}
  \includegraphics[width=0.24\textwidth]{figures/4TeV/beamsizeRatio/ratioPhiEnPrZ2.pdf}
  \includegraphics[width=0.24\textwidth]{figures/4TeV/beamsizeRatio/ratioPhiEnPrZ3.pdf}
  \includegraphics[width=0.24\textwidth]{figures/4TeV/beamsizeRatio/ratioPhiEnPrZ4.pdf}
\end{center}
\vspace{-0.6cm}
 \caption{Azimuthal energy distributions of all particles (top) and protons (bottom) in different z-regions.
  \label{bsZ}}
\end{figure}

%% \begin{figure}%%[!htb]
%% \begin{center}
%%   \includegraphics[width=0.411\textwidth]{figures/4TeV/beamsizeRatio/ratioPhiEnMuE100.pdf}
%%   \includegraphics[width=0.411\textwidth]{figures/4TeV/beamsizeRatio/ratioPhiEnMuonsZ1R1E100.pdf}
%%   \includegraphics[width=0.411\textwidth]{figures/4TeV/beamsizeRatio/ratioPhiEnMuonsZ2R1E100.pdf}
%%   \includegraphics[width=0.411\textwidth]{figures/4TeV/beamsizeRatio/ratioPhiEnMuonsZ3R1E100.pdf}
%% \end{center}
%% \vspace{-0.6cm}
%%  \caption{4 TeV azimuthal distributions of high-energy muons for different z-sections as in bottom of Fig.~\ref{twissfileBS}.
%%   \label{bsRatioPhiEnMu}}
%% \end{figure}

\subsubsection{Simulations with crossing angle}
The motivation to introduce a crossing angle in the machine is to avoid parasitic interactions of the beam while they travel in the same beam pipe in the interaction region. A small crossing angle allows for a quasi head-on collision of two bunches while other bunches are kept separated. The amount of the crossing angle is given by other beam-beam effects which one wants to suppress and is trade-off between maximising luminosity and keeping the beam stable. The plane in which the angle is introduced is chosen such that one can compensate partially another long-range beam-beam effect resulting in either a positive or negative tune shift. While in IR1 the crossing angle is in the vertical plane, it is in the horizontal plane in IR5.

We can study the crossing angle effect on background in IR1 using the 3.5~TeV simulation data of~\cite{nimPaperRod}, thus halo and beam-gas induced, without any crossing angle included in the simulations and the 4 TeV halo data with a crossing angle of 290~$\mu$rad. For this comparison, we show the distributions at the interface plane per TCT hit. 

We find that the beam-gas data in Fig.~\ref{xingCompBG} exhibit a clear signature of the improved simulation technique including the crossing angle. There is a general accumulattion in the upper part of the vertical plane; the beam is directed downwards but as the sampling position is still in the upper hemisphere the peak of particles is around $+ \frac{\pi}{2}$. Muon distributions are not significantly affected as other effects are more dominating the azimuthal distribution (like magnetic fields effects). 

In contrast, the beam-halo data in Fig.~\ref{xingCompBH} do not exhibit clear differences when comparing 3.5~TeV to 4 TeV simulations. As only effect we find that the distributions without crossing angle are slightly more symmetric in $\phi$. The reason for that is that particles of all forward angles are created at the TCTs and secondaries loose the initial crossing angle of the beam protons.

\begin{figure}
\begin{center}
  \includegraphics[width=0.4\textwidth]{figures/4TeV/compBG_3p5_vs_4TeV/ratioPhiEnAll.pdf}
  \includegraphics[width=0.4\textwidth]{figures/4TeV/compBG_3p5_vs_4TeV/ratioPhiEnPhotons.pdf}
  \includegraphics[width=0.4\textwidth]{figures/4TeV/compBG_3p5_vs_4TeV/ratioPhiEnMuons.pdf}
  \includegraphics[width=0.4\textwidth]{figures/4TeV/compBG_3p5_vs_4TeV/ratioPhiEnMuE100.pdf}
\end{center}
\vspace{-0.6cm}
 \caption{beam-gas data: Clear effect of crossing angle visible showing accumulations of particles at around $\frac{\pi}{2}$ in the vertical plane. The distribution of muons and in particular high energy muons are not significantly affected.
  \label{xingCompBG}}
\end{figure}

\begin{figure}
\begin{center}
  \includegraphics[width=0.24\textwidth]{figures/compBHB1_3p5vs4TeV/ratioPhiEnAll.pdf}
  \includegraphics[width=0.24\textwidth]{figures/compBHB1_3p5vs4TeV/ratioPhiEnPhotons.pdf}
  \includegraphics[width=0.24\textwidth]{figures/compBHB1_3p5vs4TeV/ratioPhiEnMuons.pdf}
  \includegraphics[width=0.24\textwidth]{figures/compBHB1_3p5vs4TeV/ratioPhiEnMuE100.pdf}

  \includegraphics[width=0.24\textwidth]{figures/4TeV/compB2_3p5vs4TeV/ratioPhiEnAll.pdf}
  \includegraphics[width=0.24\textwidth]{figures/4TeV/compB2_3p5vs4TeV/ratioPhiEnPhotons.pdf}
  \includegraphics[width=0.24\textwidth]{figures/4TeV/compB2_3p5vs4TeV/ratioPhiEnMuons.pdf}
  \includegraphics[width=0.24\textwidth]{figures/4TeV/compB2_3p5vs4TeV/ratioPhiEnMuE100.pdf}
\end{center}
\vspace{-0.6cm}
 \caption{B2 beam-halo data: Energy enhancement between $-\frac{\pi}{2}$ and $-\pi$, enhancement of photons in horizontal plane?? More muons around $-\frac{\pi}{2}$? No significant different for 100 GeV muons, even less energy around $\frac{\pi}{2}$.
  \label{xingCompBH}}
\end{figure}

\begin{figure}
\begin{center}
  \includegraphics[width=0.24\textwidth]{figures/compBHB1_4TeV_vs_6.5TeV/ratioEkinAll.pdf}
  \includegraphics[width=0.24\textwidth]{figures/compBHB1_4TeV_vs_6.5TeV/ratioPhiEnAll.pdf}
  \includegraphics[width=0.24\textwidth]{figures/compBHB1_4TeV_vs_6.5TeV/ratioPhiEnPhotons.pdf}
  \includegraphics[width=0.24\textwidth]{figures/compBHB1_4TeV_vs_6.5TeV/ratioPhiEnMuons.pdf}
%  \includegraphics[width=0.24\textwidth]{figures/compBHB1_4TeV_vs_6.5TeV/ratioPhiEnMuE100.pdf}

  \includegraphics[width=0.24\textwidth]{figures/compBHB2_4TeV_vs_6.5TeV/ratioEkinAll.pdf}
  \includegraphics[width=0.24\textwidth]{figures/compBHB2_4TeV_vs_6.5TeV/ratioPhiEnAll.pdf}
  %\includegraphics[width=0.24\textwidth]{figures/compBHB2_4TeV_vs_6.5TeV/ratioPhiEnPhotons.pdf}
  \includegraphics[width=0.24\textwidth]{figures/compBHB2_4TeV_vs_6.5TeV/ratioPhiEnMuons.pdf}
  \includegraphics[width=0.24\textwidth]{figures/compBHB2_4TeV_vs_6.5TeV/ratioPhiEnMuE100.pdf}
\end{center}
\vspace{-0.6cm}
 \caption{Beam-halo data: B1 (top) B2 (bottom)
  \label{valXingBH}}
\end{figure}

\subsubsection{Re-normalising beam gas events with pressure profile}

The gas densities of 2012 runs is shown in Fig.~\ref{pressure2012} for IR1 in the long straight section (LSS1) based on measurements during LHC fill 2736 and interpolated between vacuum gauges using VASCO (VAcuum Simulation COde). The most dominant molecules were simulated, H$_2$, CH$_4$, CO, CO$_2$. These molecules are decomposed to the total atomic component the interaction probability as function of the location is shown on the bottom of Fig.~\ref{pressure2012}.

The interaction probability per gas species $i$ and per seconds is given by 
\begin{equation} \label{eq2}
p_{\mathrm{int},i} = \sigma_{i} \cdot \rho_{i}(s) \cdot \frac{1}{T_{\mathrm{rev}}}
\end{equation}
with $\sigma_i$ the cross-section of the atom, $T_{\mathrm{rev}}$ the revolution period of the beam. We used the inelastic cross-sections for H, C and O as indicated in Ref.~\cite{nimPaperRod} as they are expected to be very similar to those at 4~TeV.

To obtain the normalisation per bin we applied the same method as it has been done for 3.5~TeV \cite{nimPaperRod}, scaling with the beam intensity and the total interaction probability. The method is shown in the top plot of Fig.~\ref{normed}, in green the total interaction probability up to the arc (the pressure has been extended to the arc by taking closest value as constant), in black the number of muons per beam-gas interaction and in red the multiplication of both scaled to the maximum beam intensity of that fill (2 $\times$ 10$^{14}$ protons). One can see the result is mainly shaped by the pressure map. The bottom plot shows again the number of muons per second together with high energy muons. 


\begin{figure}[!htb]
\begin{center}
  \includegraphics[width=0.75\textwidth]{figures/4TeV/LSS1_B1_Fill2736_Finala_pressure.pdf}
  \includegraphics[width=0.75\textwidth]{figures/4TeV/atomicDensities4TeV.pdf}
\end{center}
\vspace{-0.6cm}
 \caption{Gas densities in LSS1 shown for the most common molecules (top) and split into atomic components to indicate the interaction probability (bottom).
  \label{pressure2012}}
\end{figure}

\begin{figure}[!htb]
\begin{center}
  \includegraphics[width=0.95\textwidth]{figures/4TeV/flatvsprofile.pdf}
  \includegraphics[width=0.95\textwidth]{figures/4TeV/muonrates.pdf}
\end{center}
\vspace{-0.6cm}
 \caption{Origin of muon production normalised to the pressure shown in Fig.~\ref{pressure2012}. of all energies (black line) and muons with an energy above 100 GeV (filled red).
  \label{normed}}
\end{figure}
% --------------------------------------------------------------------------------------------
\subsection{Run I: Off-momentum induced showers at 4 TeV}

Off-momentum particles are ususally efficiently cleaned in IR3, but as off-amplitude particles leak from IR7, they represent as well a source of background leaking into the experimental areas. Even if one knows already from measurements, regular off-momentum loss maps are taken for the purpose of validating collimator settings, one can see it is not a dominant source. Measured off-momenta are in the range of
\begin{equation}
  \delta = \frac{\Delta \mathrm{p}}{\mathrm{p}} = 1.6 10^{-3}
\end{equation}

\begin{figure}[!htb]
\begin{center}
  \includegraphics[width=0.49\textwidth]{figures/4TeV/offmom/20MeV/Ekin_offmin500Hz_4TeV_B2_20MeV.pdf}
  \includegraphics[width=0.49\textwidth]{figures/4TeV/offmom/20MeV/PhiEnDist_offmin500Hz_4TeV_B2_20MeV.pdf}
  \includegraphics[width=0.49\textwidth]{figures/4TeV/offmom/20MeV/RadNDist_offmin500Hz_4TeV_B2_20MeV.pdf}
  \includegraphics[width=0.49\textwidth]{figures/4TeV/offmom/20MeV/RadEnDist_offmin500Hz_4TeV_B2_20MeV.pdf}
\end{center}
\vspace{-0.6cm}
 \caption{Off-momentum induced particle distributions.
  \label{offmom4TeV}}
\end{figure}

\begin{figure}[!htb]
\begin{center}
  \includegraphics[width=0.49\textwidth]{figures/4TeV/offmom/20MeV/Ekin_offplus500Hz_4TeV_B2_20MeV.pdf}
  \includegraphics[width=0.49\textwidth]{figures/4TeV/offmom/20MeV/PhiEnDist_offplus500Hz_4TeV_B2_20MeV.pdf}
  \includegraphics[width=0.49\textwidth]{figures/4TeV/offmom/20MeV/RadNDist_offplus500Hz_4TeV_B2_20MeV.pdf}
  \includegraphics[width=0.49\textwidth]{figures/4TeV/offmom/20MeV/RadEnDist_offplus500Hz_4TeV_B2_20MeV.pdf}
\end{center}
\vspace{-0.6cm}
 \caption{Off-momentum particles generated by a positive frequency shift induce particle distributions.
  \label{offmom4TeV}}
\end{figure}


\begin{figure}[!htb]
\begin{center}
  \includegraphics[width=0.30\textwidth]{figures/4TeV/offmom/comppm500Hz/ratioEkinAll.pdf}
  \includegraphics[width=0.30\textwidth]{figures/4TeV/offmom/comppm500Hz/ratioEkinMuons.pdf}
  \includegraphics[width=0.30\textwidth]{figures/4TeV/offmom/comppm500Hz/ratioEkinProtons.pdf}
  \includegraphics[width=0.30\textwidth]{figures/4TeV/offmom/comppm500Hz/ratioEkinNeutrons.pdf}
\end{center}
\vspace{-0.6cm}
 \caption{Off-momentum particles generated by a positive frequency shift induce particle distributions.
  \label{compPM_ekin}}
\end{figure}

\begin{figure}[!htb]
\begin{center}
  \includegraphics[width=0.30\textwidth]{figures/4TeV/offmom/comppm500Hz/ratioPhiEnAll.pdf}
  \includegraphics[width=0.30\textwidth]{figures/4TeV/offmom/comppm500Hz/ratioPhiEnMuons.pdf}
  \includegraphics[width=0.30\textwidth]{figures/4TeV/offmom/comppm500Hz/ratioPhiEnMuE100.pdf}
  \includegraphics[width=0.30\textwidth]{figures/4TeV/offmom/comppm500Hz/ratioPhiEnProtons.pdf}
\end{center}
\vspace{-0.6cm}
 \caption{Off-momentum particles generated by a positive frequency shift induce particle distributions.
  \label{compPM_phien}}
\end{figure}

\begin{figure}[!htb]
\begin{center}
  \includegraphics[width=0.30\textwidth]{figures/4TeV/offmom/comppm500Hz/ratioRadEnAll.pdf}
  \includegraphics[width=0.30\textwidth]{figures/4TeV/offmom/comppm500Hz/ratioRadEnMuons.pdf}
  \includegraphics[width=0.30\textwidth]{figures/4TeV/offmom/comppm500Hz/ratioRadEnProtons.pdf}
  \includegraphics[width=0.30\textwidth]{figures/4TeV/offmom/comppm500Hz/ratioRadEnPhotons.pdf}
\end{center}
\vspace{-0.6cm}
 \caption{Off-momentum particles generated by a positive frequency shift induce particle distributions.
  \label{compPM_raden}}
\end{figure}

% --------------------------------------------------------------------------------------------
\subsection{Comparison of Run I background sources}
\begin{figure}
\begin{center}
  \includegraphics[width=0.42\textwidth]{figures/4TeV/compAllBKG/EkinAll.pdf}
  \includegraphics[width=0.42\textwidth]{figures/4TeV/compAllBKG/PhiEnAll.pdf}
  \includegraphics[width=0.42\textwidth]{figures/4TeV/compAllBKG/EkinMuons.pdf}
  \includegraphics[width=0.42\textwidth]{figures/4TeV/compAllBKG/PhiEnMuons.pdf}
  \includegraphics[width=0.42\textwidth]{figures/4TeV/compAllBKG/EkinProtons.pdf}
  \includegraphics[width=0.42\textwidth]{figures/4TeV/compAllBKG/PhiEnProtons.pdf}
  \includegraphics[width=0.42\textwidth]{figures/4TeV/compAllBKG/EkinPhotons.pdf}
  \includegraphics[width=0.42\textwidth]{figures/4TeV/compAllBKG/PhiEnPhotons.pdf}
\end{center}
\vspace{-0.6cm}
 \caption{comparison of all background sources at 4 TeV.
  \label{compPM_raden}}
\end{figure}

% --------------------------------------------------------------------------------------------
\subsection{Run II: 6.5 TeV Beam-Halo}

The same method is used as in the 4 TeV simulations changing to the Run II 2015 scenario, with $\beta^* = 80$~cm, see also Tab.~\ref{paramsRun12}. The input distributions are shown for B1 and B2 in Fig.~\ref{inel6.5}. One can already see that the B1 depth distribution in the TCTV is much shallower than at 4~TeV in Fig.~\ref{inel4TeV} most of the hits are in the very first bin. This is very likely due to the different collimator settings and optics. Characteristic distributions are shown in Fig.~\ref{dist6500GeVB2}, and a comparison to B1 distributions are in Fig.~\ref{compBHB1B2run2}. One can see that in contrast to 4 TeV data, B1 produces more shower particles and energy than B2. A detailed discussion on the comparison to 4~TeV beam-halo results is shown in Sec.~\ref{compRunI+II}.

\begin{figure}[!htb]
\begin{center}
\includegraphics[width=0.4\textwidth]{figures/inelposition_sum_HALOB1.pdf}
\includegraphics[width=0.4\textwidth]{figures/inelposition_sum_HALOB2.pdf}
\end{center}
\begin{picture} (0.,0.)
\setlength{\unitlength}{1.0cm}
\small{
    \put ( 4.,1.){(a)}
    \put ( 12.4,1.){(b)}}
\end{picture}
\vspace{-0.6cm}
 \caption{Depth of the inelastic interaction of the halo hits with the collimator material for B1 and B2 6.5 TeV hits.
  \label{inel6.5}}
\end{figure}


\begin{figure}[!htb]
\begin{center}
\includegraphics[width=0.49\textwidth]{figures/BH_run2/Ekin_BH_6500GeV_haloB2_20MeV.pdf}
\includegraphics[width=0.49\textwidth]{figures/BH_run2/PhiEnDist_BH_6500GeV_haloB2_20MeV.pdf}
%\includegraphics[width=0.49\textwidth]{figures/BH_run2/PhiNDist_BH_6500GeV_haloB2_20MeV.pdf}
\includegraphics[width=0.49\textwidth]{figures/BH_run2/RadNDist_BH_6500GeV_haloB2_20MeV.pdf}
\includegraphics[width=0.49\textwidth]{figures/BH_run2/RadEnDist_BH_6500GeV_haloB2_20MeV.pdf}
\includegraphics[width=0.49\textwidth]{figures/BH_run2/OrigYZMuons_BH_6500GeV_haloB2_20MeV.pdf}
\includegraphics[width=0.49\textwidth]{figures/BH_run2/OrigXYMuons_BH_6500GeV_haloB2_20MeV.pdf}
\end{center}
\begin{picture} (0.,0.)
\setlength{\unitlength}{1.0cm}
\small{
    \put ( 4.,7.35){(a)}
    \put ( 12.4,7.35){(b)}
    \put ( 4.,1.){(c)}
    \put ( 12.4,1.){(d)}}
\end{picture}
\vspace{-0.6cm}
 \caption{B2 halo induced background at the interface plane. The distributions exhibit the same features as at 4 TeV.
  \label{dist6500GeVB2}}
\end{figure}


\begin{figure}[!htb]
\begin{center}
  \includegraphics[width=0.411\textwidth]{figures/BH_run2/perTCThit/ratioEkinAll.pdf}
  \includegraphics[width=0.411\textwidth]{figures/BH_run2/perTCThit/ratioPhiEnAll.pdf}
  \includegraphics[width=0.411\textwidth]{figures/BH_run2/perTCThit/ratioPhiEnMuons.pdf}
  \includegraphics[width=0.411\textwidth]{figures/BH_run2/perTCThit/ratioPhiEnProtons.pdf}
\end{center}
%% \begin{picture} (0.,0.)
%% \setlength{\unitlength}{1.0cm}
%% \small{
%%     \put ( 4.,7.35){(a)}
%%     \put ( 12.4,7.35){(b)}
%%     \put ( 4.,1.){(c)}
%%     \put ( 12.4,1.){(d)}}
%% \end{picture}
\vspace{-0.6cm}
 \caption{Comparison of B1/B2 halo induced distributions per TCT hit.
  \label{compBHB1B2run2}}
\end{figure}



\subsection{Run II: 6.5 TeV Beam-Gas}

This case has been simulated with the same method as for Run I scenario, using the extension of the beam size. The beam optics and collimator settings were different in the 2015 Run II, as detailed in Tab.~\ref{paramsRun12}. Characteristic distributions are shown in Fig.~\ref{bg6500}.

\begin{figure}[!htb]
\begin{center}
  \includegraphics[width=0.49\textwidth]{figures/6500GeV/20MeV/Ekin_BG_6500GeV_flat_20MeV_bs.pdf}
  \includegraphics[width=0.49\textwidth]{figures/6500GeV/20MeV/PhiEnDist_BG_6500GeV_flat_20MeV_bs.pdf}
  \includegraphics[width=0.49\textwidth]{figures/6500GeV/20MeV/RadNDist_BG_6500GeV_flat_20MeV_bs.pdf}
  \includegraphics[width=0.49\textwidth]{figures/6500GeV/20MeV/RadEnDist_BG_6500GeV_flat_20MeV_bs.pdf}
\end{center}
\vspace{-0.6cm}
 \caption{Characteristic beam-gas induced distributions at 6.5~TeV per BG interaction using the more realistic model of the beam size.
  \label{bg6500}}
\end{figure}


\subsubsection{Comparison of Run II background sources}
\begin{figure}
\begin{center}
  \includegraphics[width=0.42\textwidth]{figures/6500GeV/compAllBKG/EkinAll.pdf}
  \includegraphics[width=0.42\textwidth]{figures/6500GeV/compAllBKG/PhiEnAll.pdf}
  \includegraphics[width=0.42\textwidth]{figures/6500GeV/compAllBKG/EkinMuons.pdf}
  \includegraphics[width=0.42\textwidth]{figures/6500GeV/compAllBKG/PhiEnMuons.pdf}
  \includegraphics[width=0.42\textwidth]{figures/6500GeV/compAllBKG/EkinProtons.pdf}
  \includegraphics[width=0.42\textwidth]{figures/6500GeV/compAllBKG/PhiEnProtons.pdf}
  \includegraphics[width=0.42\textwidth]{figures/6500GeV/compAllBKG/EkinPhotons.pdf}
  \includegraphics[width=0.42\textwidth]{figures/6500GeV/compAllBKG/PhiEnPhotons.pdf}
\end{center}
\vspace{-0.6cm}
 \caption{comparison of all background sources at 6.5~TeV.
  \label{compAllBKG_6.5}}
\end{figure}

