\section{Simulation Results for LHC's Run I and Run II}

The two background sources were investigated for several run scenarios. During Run I, we had two beam energies, 3.5 TeV in 2010--11 and 4~TeV in 2012\footnote{A detailed overview of parameters in Run 1 can be found in~\cite{ParametersRun1}} While the 3.5~TeV background data has been presented in detail in~\cite{nimPaperRod}, we focus on further developments made since that paper and use these improvements for 4 TeV and Run II simulation cases. Beam-halo simulations were improved compared to~\cite{nimPaperRod} by considering a crossing angle in the simulation as it has been present in the machine. Another improvement concerns beam-gas simulations for which the transverse beam size has been taken into account. 
\subsection{Sensitivity Studies}
We investigate how the additional improvements, i.e.~the inclusion of the crossing angle and beam size, effect previous simulations.

\subsubsection{Crossing Angle}
The motivation to introduce a crossing angle in the machine is to avoid parasitic interactions of the beam while they travel in the same beam pipe in the interaction region. A small crossing angle allows for a quasi head-on collision of two bunches while other bunches are kept separated. The amount of the crossing angle is given by other beam-beam effects which one wants to suppress and is trade-off between maximising luminosity and keeping the beam stable. The plane in which the angle is introduced is chosen such that one can compensate partially another long-range beam-beam effect resulting in either a positive or negative tune shift. While in IR1 the crossing angle is in the vertical plane, it is in the horizontal plane in IR5.

We can study the crossing angle effect on background in IR1 using the 3.5~TeV halo simulation data of~\cite{nimPaperRod} which did not have any crossing angle included and the 4 TeV halo data with a crossing angle of 290~$\mu$rad. For this comparison, we plot the distributions at the interface plane per TCT hit. 


\begin{figure}[!htb]
\begin{center}
\includegraphics[width=0.495\textwidth]{figures/}

\end{center}
\begin{picture} (0.,0.)
\setlength{\unitlength}{1.0cm}
\small{
    \put ( 4.,1.){(a)}
    \put ( 12.4,1.){(b)}
}
\end{picture}
\vspace{-0.6cm}
 \caption{$\phi$ distribution of .
  \label{compTCT5INOUT}}
\end{figure}



\subsubsection{Beam Size}

\subsection{Run I: 4 TeV Beam-Halo}
\subsection{Run I: 4 TeV Beam-Gas}

\subsection{Run II: 6.5 TeV Beam-Halo}
\subsection{Run II: 6.5 TeV Beam-Gas}
