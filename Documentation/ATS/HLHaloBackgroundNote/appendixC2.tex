\clearpage
\begin{figure}
\begin{center}
  \includegraphics[width=0.42\textwidth]{figures/4TeV/compAllBKG/EkinAll.pdf}
  \includegraphics[width=0.42\textwidth]{figures/4TeV/compAllBKG/PhiEnAll.pdf}
%  \includegraphics[width=0.42\textwidth]{figures/4TeV/compAllBKG/EkinMuons.pdf}
%  \includegraphics[width=0.42\textwidth]{figures/4TeV/compAllBKG/PhiEnMuons.pdf}
  \includegraphics[width=0.42\textwidth]{figures/4TeV/compAllBKG/EkinProtons.pdf}
  \includegraphics[width=0.42\textwidth]{figures/4TeV/compAllBKG/PhiEnProtons.pdf}
  \includegraphics[width=0.42\textwidth]{figures/4TeV/compAllBKG/EkinPhotons.pdf}
  \includegraphics[width=0.42\textwidth]{figures/4TeV/compAllBKG/PhiEnPhotons.pdf}
\end{center}
\vspace{-0.6cm}
 \caption{Comparison of all background sources at 4 TeV normalised per interaction showing energy spectrum and energy in $\phi$.
  \label{fig:compAllBKG_perInt1}}
\end{figure}

\begin{figure}
\begin{center}
  \includegraphics[width=0.42\textwidth]{figures/4TeV/compAllBKG/RadNAll.pdf}
  \includegraphics[width=0.42\textwidth]{figures/4TeV/compAllBKG/RadEnAll.pdf}
  \includegraphics[width=0.42\textwidth]{figures/4TeV/compAllBKG/RadNMuons.pdf}
  \includegraphics[width=0.42\textwidth]{figures/4TeV/compAllBKG/RadEnMuons.pdf}
  \includegraphics[width=0.42\textwidth]{figures/4TeV/compAllBKG/RadNProtons.pdf}
  \includegraphics[width=0.42\textwidth]{figures/4TeV/compAllBKG/RadEnProtons.pdf}
  \includegraphics[width=0.42\textwidth]{figures/4TeV/compAllBKG/RadNPhotons.pdf}
  \includegraphics[width=0.42\textwidth]{figures/4TeV/compAllBKG/RadEnPhotons.pdf}
\end{center}
\vspace{-0.6cm}
 \caption{Comparison of all background sources at 4 TeV normalised per interaction showing radial distributions and energy in $r$.
  \label{fig:compAllBKG_perInt2}}
\end{figure}

\begin{figure}
\begin{center}
  \includegraphics[width=0.42\textwidth]{figures/4TeV/reweighted/cv78_EkinAll.pdf}
  \includegraphics[width=0.42\textwidth]{figures/4TeV/reweighted/cv78_PhiEnAll.pdf}
%  \includegraphics[width=0.42\textwidth]{figures/4TeV/reweighted/cv78_EkinMuons.pdf}
%  \includegraphics[width=0.42\textwidth]{figures/4TeV/reweighted/cv78_PhiEnMuons.pdf}
  \includegraphics[width=0.42\textwidth]{figures/4TeV/reweighted/cv78_EkinProtons.pdf}
  \includegraphics[width=0.42\textwidth]{figures/4TeV/reweighted/cv78_PhiEnProtons.pdf}
  \includegraphics[width=0.42\textwidth]{figures/4TeV/reweighted/cv78_EkinPhotons.pdf}
  \includegraphics[width=0.42\textwidth]{figures/4TeV/reweighted/cv78_PhiEnPhotons.pdf}
\end{center}
\vspace{-0.6cm}
 \caption{comparison of all background sources at 4 TeV normalised to a rate.
  \label{compAllBKG4TeV_rates}}
\end{figure}

\begin{figure}
\begin{center}
  \includegraphics[width=0.42\textwidth]{figures/4TeV/reweighted/cv78_RadNAll.pdf}
  \includegraphics[width=0.42\textwidth]{figures/4TeV/reweighted/cv78_RadEnAll.pdf}
  \includegraphics[width=0.42\textwidth]{figures/4TeV/reweighted/cv78_RadNMuons.pdf}
  \includegraphics[width=0.42\textwidth]{figures/4TeV/reweighted/cv78_RadEnMuons.pdf}
  \includegraphics[width=0.42\textwidth]{figures/4TeV/reweighted/cv78_RadNProtons.pdf}
  \includegraphics[width=0.42\textwidth]{figures/4TeV/reweighted/cv78_RadEnProtons.pdf}
  \includegraphics[width=0.42\textwidth]{figures/4TeV/reweighted/cv78_RadNPhotons.pdf}
  \includegraphics[width=0.42\textwidth]{figures/4TeV/reweighted/cv78_RadEnPhotons.pdf}
\end{center}
\vspace{-0.6cm}
 \caption{comparison of all background sources at 4 TeV normalised to a rate.
  \label{compAllBKG4TeV_rates2}}
\end{figure}

\begin{figure}
\begin{center}
  \includegraphics[width=0.8\textwidth]{figures/cv87_allenergies_OrigZAll.pdf}
%  \includegraphics[width=0.8\textwidth]{figures/cv87_allenergies_OrigZMuon.pdf}
\end{center}
\vspace{-0.6cm}
 \caption{
  \label{fig:OrigZMuonAllEn2}} 
\end{figure}



\begin{figure}
\begin{center}
  \includegraphics[width=0.42\textwidth]{figures/6500GeV/reweighted/cv78_EkinAll.pdf}
  \includegraphics[width=0.42\textwidth]{figures/6500GeV/reweighted/cv78_PhiEnAll.pdf}
%  \includegraphics[width=0.42\textwidth]{figures/6500GeV/reweighted/cv78_EkinMuons.pdf}
 % \includegraphics[width=0.42\textwidth]{figures/6500GeV/reweighted/cv78_PhiEnMuons.pdf}
  \includegraphics[width=0.42\textwidth]{figures/6500GeV/reweighted/cv78_EkinProtons.pdf}
  \includegraphics[width=0.42\textwidth]{figures/6500GeV/reweighted/cv78_PhiEnProtons.pdf}
 \includegraphics[width=0.42\textwidth]{figures/6500GeV/reweighted/cv78_EkinPhotons.pdf}
 \includegraphics[width=0.42\textwidth]{figures/6500GeV/reweighted/cv78_PhiEnPhotons.pdf}
\end{center}
\vspace{-0.6cm}
 \caption{comparison of all background sources at 6.5~TeV.
  \label{compAllBKG_6.52}}
\end{figure}

\begin{figure}%[!htb]
\centering
\includegraphics[width=0.45\textwidth]{figures/compBGreweighted/ratioEkinAll.pdf}
%\includegraphics[width=0.45\textwidth]{figures/compBGreweighted/ratioEkinMuons.pdf}
\includegraphics[width=0.45\textwidth]{figures/compBGreweighted/ratioPhiEnAll.pdf}
%\includegraphics[width=0.45\textwidth]{figures/compBGreweighted/ratioPhiEnMuons.pdf}
\caption{Reweighted beam-gas distributions in the 2012 Run I and 2015 Run II scenario for all particles and muons showing the energy spectrum (top) and the azimuthal distribution (bottom).
  \label{fig:compBGreweighted12}}
\end{figure}




\begin{figure}%[!htb]
\centering
\includegraphics[width=0.45\textwidth]{figures/compBGreweighted/ratioRadNAll.pdf}
\includegraphics[width=0.45\textwidth]{figures/compBGreweighted/ratioRadNMuons.pdf}
\includegraphics[width=0.45\textwidth]{figures/compBGreweighted/ratioRadEnAll.pdf}
\includegraphics[width=0.45\textwidth]{figures/compBGreweighted/ratioRadEnMuons.pdf}
\caption{Reweighted beam-gas distributions in the 2012 Run I and 2015 Run II scenario for all particles and muons showing radial positions and energy in $r$.
  \label{fig:compBGreweighted2}}
\end{figure}


\begin{figure}
\begin{center}
  \includegraphics[width=0.49\textwidth]{figures/compBHB1_4TeV_vs_6p5TeV/normalised/ratioEkinAll.pdf}
%  \includegraphics[width=0.49\textwidth]{figures/compBHB1_4TeV_vs_6p5TeV/normalised/ratioEkinMuons.pdf}
  \includegraphics[width=0.49\textwidth]{figures/compBHB1_4TeV_vs_6p5TeV/normalised/ratioPhiNAll.pdf}
  \includegraphics[width=0.49\textwidth]{figures/compBHB1_4TeV_vs_6p5TeV/normalised/ratioPhiEnAll.pdf}
%  \includegraphics[width=0.49\textwidth]{figures/compBHB1_4TeV_vs_6p5TeV/normalised/ratioPhiEnMuons.pdf}
  \includegraphics[width=0.49\textwidth]{figures/compBHB1_4TeV_vs_6p5TeV/normalised/ratioRadEnAll.pdf}

\end{center}
\vspace{-0.6cm}
 \caption{Comparison of halo induced background at 4 and 6.5~TeV in the azimuthal distributions of all particles at the interface plane (top) and high-energy muons and protons (bottom) and their energy.
  \label{compBHB1run1run22}}
\end{figure}


\begin{figure}%[!htb]
\begin{center}
  \includegraphics[width=0.49\textwidth]{figures/compBHB2_4TeV_vs_6p5TeV/normalised/ratioEkinAll.pdf}
  \includegraphics[width=0.49\textwidth]{figures/compBHB2_4TeV_vs_6p5TeV/normalised/ratioPhiEnAll.pdf}
  \includegraphics[width=0.49\textwidth]{figures/compBHB2_4TeV_vs_6p5TeV/normalised/ratioPhiEnMuons.pdf}
  \includegraphics[width=0.49\textwidth]{figures/compBHB2_4TeV_vs_6p5TeV/normalised/ratioRadEnMuons.pdf}
\end{center}
\vspace{-0.6cm}
 \caption{Comparison of halo induced background at 4 and 6.5~TeV in the azimuthal distributions of all particles at the interface plane (top) and high-energy muons and protons (bottom) and their energy.
  \label{compBHB2run1run2}}
\end{figure}



\begin{figure}
\begin{center}
  \includegraphics[width=0.42\textwidth]{figures/HLRunII/cv78_EkinAll.pdf}
  \includegraphics[width=0.42\textwidth]{figures/HLRunII/cv78_PhiEnAll.pdf}
%  \includegraphics[width=0.42\textwidth]{figures/HLRunII/cv78_EkinMuons.pdf}
%  \includegraphics[width=0.42\textwidth]{figures/HLRunII/cv78_PhiEnMuons.pdf}
  \includegraphics[width=0.42\textwidth]{figures/HLRunII/cv78_EkinProtons.pdf}
  \includegraphics[width=0.42\textwidth]{figures/HLRunII/cv78_PhiEnProtons.pdf}
  \includegraphics[width=0.42\textwidth]{figures/HLRunII/cv78_EkinPhotons.pdf}
  \includegraphics[width=0.42\textwidth]{figures/HLRunII/cv78_PhiEnPhotons.pdf}
\end{center}
\vspace{-0.6cm}
 \caption{Comparison of beam-gas (BG) in Run II and beam-halo in HL using the baseline layout (TCT5s in, \twosigmaret~settings) and round beam optics.
  \label{fig:hlrun22}}
\end{figure}


\begin{figure}
\begin{center}

\includegraphics[width=0.495\textwidth]{figures/HL/tct5inrd/RadNDist_BH_HL_tct5inrdB2_20MeV.pdf}
\includegraphics[width=0.495\textwidth]{figures/HL/tct5inrd/RadEnDist_BH_HL_tct5inrdB2_20MeV.pdf}
\end{center}
\vspace{-0.6cm}
 \caption{B2 induced particle distributions at the interface plane.}
  \label{tct5inrdb2retr2}
\end{figure}



\begin{figure}
\centering
\includegraphics[width=0.4\textwidth]{figures/HL/compINOUTB1_retracted/perTCThit/ratioEkinAll}
\includegraphics[width=0.4\textwidth]{figures/HL/compINOUTB1_retracted/perTCThit/ratioEkinMuons}
\includegraphics[width=0.4\textwidth]{figures/HL/compINOUTB1_retracted/perTCThit/ratioPhiNAll}
\includegraphics[width=0.4\textwidth]{figures/HL/compINOUTB1_retracted/perTCThit/ratioPhiNMuons}
\includegraphics[width=0.4\textwidth]{figures/HL/compINOUTB1_retracted/perTCThit/ratioPhiEnAll}
\includegraphics[width=0.4\textwidth]{figures/HL/compINOUTB1_retracted/perTCThit/ratioPhiEnMuons}
 \caption{Distributions of all particles (left) and muons (right) and their energy in the two cases, TCT4s only and TCT5s in, for B1.
  \label{fig:compInOutB1_perTCThit}}
\end{figure}




\begin{figure}
\begin{center}
\includegraphics[width=0.4\textwidth]{figures/HL/compINOUTB2_retracted/perTCThit/ratioEkinAll}
\includegraphics[width=0.4\textwidth]{figures/HL/compINOUTB2_retracted/perTCThit/ratioEkinMuons}
\includegraphics[width=0.4\textwidth]{figures/HL/compINOUTB2_retracted/perTCThit/ratioPhiNAll}
\includegraphics[width=0.4\textwidth]{figures/HL/compINOUTB2_retracted/perTCThit/ratioPhiNMuons}
\includegraphics[width=0.4\textwidth]{figures/HL/compINOUTB2_retracted/perTCThit/ratioPhiEnAll}
\includegraphics[width=0.4\textwidth]{figures/HL/compINOUTB2_retracted/perTCThit/ratioPhiEnMuons}
\end{center}
\vspace{-0.6cm}
 \caption{Azimuthal distribution of all particles and muons (top) and their energy (bottom).
  \label{fig:compInOutB2}}
\end{figure}


\begin{figure}
\centering
\includegraphics[width=0.43\textwidth]{figures/HL/compNomRetrCollSett/normalised/ratioEkinAll.pdf}
\includegraphics[width=0.43\textwidth]{figures/HL/compNomRetrCollSett/normalised/ratioPhiEnAll.pdf}
 \caption{Comparison of halo-induced background showers with nominal and \twosigmaret~collimator settings. Overall halo reduction of a factor 2 to 3 can be obtained considering also the IR7 to IR1 conversion factors\label{fig:compNomRetrSett2}.}
\end{figure}
