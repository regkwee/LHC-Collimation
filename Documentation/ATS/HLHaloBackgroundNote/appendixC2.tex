\clearpage
\begin{figure}
\begin{center}
  \includegraphics[width=0.42\textwidth]{figures/4TeV/compAllBKG/EkinAll.pdf}
  \includegraphics[width=0.42\textwidth]{figures/4TeV/compAllBKG/PhiEnAll.pdf}
%  \includegraphics[width=0.42\textwidth]{figures/4TeV/compAllBKG/EkinMuons.pdf}
%  \includegraphics[width=0.42\textwidth]{figures/4TeV/compAllBKG/PhiEnMuons.pdf}
  \includegraphics[width=0.42\textwidth]{figures/4TeV/compAllBKG/EkinProtons.pdf}
  \includegraphics[width=0.42\textwidth]{figures/4TeV/compAllBKG/PhiEnProtons.pdf}
  \includegraphics[width=0.42\textwidth]{figures/4TeV/compAllBKG/EkinPhotons.pdf}
  \includegraphics[width=0.42\textwidth]{figures/4TeV/compAllBKG/PhiEnPhotons.pdf}
\end{center}
\vspace{-0.6cm}
 \caption{Comparison of all background sources at 4 TeV normalised per interaction showing energy spectrum and energy in $\phi$.
  \label{fig:compAllBKG_perInt1}}
\end{figure}

\begin{figure}
\begin{center}
  \includegraphics[width=0.42\textwidth]{figures/4TeV/compAllBKG/RadNAll.pdf}
  \includegraphics[width=0.42\textwidth]{figures/4TeV/compAllBKG/RadEnAll.pdf}
  \includegraphics[width=0.42\textwidth]{figures/4TeV/compAllBKG/RadNMuons.pdf}
  \includegraphics[width=0.42\textwidth]{figures/4TeV/compAllBKG/RadEnMuons.pdf}
  \includegraphics[width=0.42\textwidth]{figures/4TeV/compAllBKG/RadNProtons.pdf}
  \includegraphics[width=0.42\textwidth]{figures/4TeV/compAllBKG/RadEnProtons.pdf}
  \includegraphics[width=0.42\textwidth]{figures/4TeV/compAllBKG/RadNPhotons.pdf}
  \includegraphics[width=0.42\textwidth]{figures/4TeV/compAllBKG/RadEnPhotons.pdf}
\end{center}
\vspace{-0.6cm}
 \caption{Comparison of all background sources at 4 TeV normalised per interaction showing radial distributions and energy in $r$.
  \label{fig:compAllBKG_perInt2}}
\end{figure}

\begin{figure}
\begin{center}
  \includegraphics[width=0.42\textwidth]{figures/4TeV/reweighted/app/cv78_EkinAll.pdf}
  \includegraphics[width=0.42\textwidth]{figures/4TeV/reweighted/app/cv78_PhiEnAll.pdf}
%  \includegraphics[width=0.42\textwidth]{figures/4TeV/reweighted/app/cv78_EkinMuons.pdf}
%  \includegraphics[width=0.42\textwidth]{figures/4TeV/reweighted/app/cv78_PhiEnMuons.pdf}
  \includegraphics[width=0.42\textwidth]{figures/4TeV/reweighted/app/cv78_EkinProtons.pdf}
  \includegraphics[width=0.42\textwidth]{figures/4TeV/reweighted/app/cv78_PhiEnProtons.pdf}
  \includegraphics[width=0.42\textwidth]{figures/4TeV/reweighted/app/cv78_EkinPhotons.pdf}
  \includegraphics[width=0.42\textwidth]{figures/4TeV/reweighted/app/cv78_PhiEnPhotons.pdf}
\end{center}
\vspace{-0.6cm}
 \caption{Comparison of background sources at 4 TeV at the interface plane normalised to a rate (halo is from IR7 cleaning, thus betatron halo).
  \label{compAllBKG4TeV_rates}}
\end{figure}

\begin{figure}
\begin{center}
  \includegraphics[width=0.42\textwidth]{figures/4TeV/reweighted/app/cv78_RadNAll.pdf}
  \includegraphics[width=0.42\textwidth]{figures/4TeV/reweighted/app/cv78_RadEnAll.pdf}
  \includegraphics[width=0.42\textwidth]{figures/4TeV/reweighted/app/cv78_RadNMuons.pdf}
  \includegraphics[width=0.42\textwidth]{figures/4TeV/reweighted/app/cv78_RadEnMuons.pdf}
  \includegraphics[width=0.42\textwidth]{figures/4TeV/reweighted/app/cv78_RadNProtons.pdf}
  \includegraphics[width=0.42\textwidth]{figures/4TeV/reweighted/app/cv78_RadEnProtons.pdf}
  \includegraphics[width=0.42\textwidth]{figures/4TeV/reweighted/app/cv78_RadNPhotons.pdf}
  \includegraphics[width=0.42\textwidth]{figures/4TeV/reweighted/app/cv78_RadEnPhotons.pdf}
\end{center}
\vspace{-0.6cm}
 \caption{comparison of all background sources at 4 TeV at the inteface plane normalised to a rate (halo is from IR7 cleaning, thus betatron halo).
  \label{compAllBKG4TeV_rates2}}
\end{figure}




\begin{figure}
\begin{center}
  \includegraphics[width=0.42\textwidth]{figures/6500GeV/reweighted/app/cv78_EkinAll.pdf}
  \includegraphics[width=0.42\textwidth]{figures/6500GeV/reweighted/app/cv78_PhiEnAll.pdf}
%  \includegraphics[width=0.42\textwidth]{figures/6500GeV/reweighted/app/cv78_EkinMuons.pdf}
 % \includegraphics[width=0.42\textwidth]{figures/6500GeV/reweighted/app/cv78_PhiEnMuons.pdf}
  \includegraphics[width=0.42\textwidth]{figures/6500GeV/reweighted/app/cv78_EkinProtons.pdf}
  \includegraphics[width=0.42\textwidth]{figures/6500GeV/reweighted/app/cv78_PhiEnProtons.pdf}
 \includegraphics[width=0.42\textwidth]{figures/6500GeV/reweighted/app/cv78_EkinPhotons.pdf}
 \includegraphics[width=0.42\textwidth]{figures/6500GeV/reweighted/app/cv78_PhiEnPhotons.pdf}
\end{center}
\vspace{-0.6cm}
 \caption{Comparison of background sources at 6.5~TeV normalised to typical rates in Run~II 2015.
  \label{compAllBKG_6.52}}
\end{figure}




\begin{figure}
\begin{center}
  \includegraphics[width=0.492\textwidth]{figures/6500GeV/20MeV/PhiEnMu_BG_6500GeV_flat_20GeV_bs.pdf}
  \includegraphics[width=0.492\textwidth]{figures/BH_run2/b1/PhiEnMu_BH_6500GeV_haloB1_20MeV.pdf}      
\end{center}
\vspace{-0.6cm}
 \caption{Run~II 2015: Muons sorted by radius in beam-gas (left) and betatron halo of B1 (right) at the inteface plane.
  \label{fig:PhiEnMu}}
\end{figure}



\begin{figure}
\begin{center}
  \includegraphics[width=0.492\textwidth]{figures/6500GeV/20MeV/PhiEnMuPM_BG_6500GeV_flat_20GeV_bs.pdf}
    \includegraphics[width=0.492\textwidth]{figures/BH_run2/b1/PhiEnMuPM_BH_6500GeV_haloB1_20MeV.pdf}            
\end{center}
\vspace{-0.6cm}
 \caption{Run~II 2015: Energy of muons separated by their charge in beam-gas (left) and betatron halo (right) induced showers at the interface plane. A clear effect of the combination dipole is visible. 
  \label{fig:PhiEnMuPM}}
\end{figure}


\begin{figure}
\begin{center}
  \includegraphics[width=0.492\textwidth]{figures/6500GeV/20MeV/RadNMuons_BG_6500GeV_flat_20GeV_bs.pdf}
  \includegraphics[width=0.492\textwidth]{figures/BH_run2/b1/RadNMuons_BH_6500GeV_haloB1_20MeV.pdf}      
\end{center}
\vspace{-0.6cm}
 \caption{Radii of muons of different energies.
  \label{fig:PhiEnMuComp}}
\end{figure}

% -------------- HL

\begin{figure}
\begin{center}
\includegraphics[width=0.495\textwidth]{figures/HL/tct5inrd/RadNDist_BH_HL_tct5inrdB2_20MeV.pdf}
\includegraphics[width=0.495\textwidth]{figures/HL/tct5inrd/RadEnDist_BH_HL_tct5inrdB2_20MeV.pdf}
\end{center}
\vspace{-0.6cm}
 \caption{B2 betatron halo induced particle distributions at the interface plane for the HL-LHC scenario with baseline settings (.}
  \label{tct5inrdb2retr2}
\end{figure}



\begin{figure}
\centering
\includegraphics[width=0.4\textwidth]{figures/HL/compINOUTB1_retracted/perTCThit/ratioEkinAll}
\includegraphics[width=0.4\textwidth]{figures/HL/compINOUTB1_retracted/perTCThit/ratioEkinMuons}
\includegraphics[width=0.4\textwidth]{figures/HL/compINOUTB1_retracted/perTCThit/ratioPhiNAll}
\includegraphics[width=0.4\textwidth]{figures/HL/compINOUTB1_retracted/perTCThit/ratioPhiNMuons}
\includegraphics[width=0.4\textwidth]{figures/HL/compINOUTB1_retracted/perTCThit/ratioPhiEnAll}
\includegraphics[width=0.4\textwidth]{figures/HL/compINOUTB1_retracted/perTCThit/ratioPhiEnMuons}
 \caption{Distributions of all particles (left) and muons (right) and their energy in the two cases, TCT4s only and TCT5s in, for B1.
  \label{fig:compInOutB1_perTCThit}}
\end{figure}




\begin{figure}
\begin{center}
\includegraphics[width=0.4\textwidth]{figures/HL/compINOUTB2_retracted/perTCThit/ratioEkinAll}
\includegraphics[width=0.4\textwidth]{figures/HL/compINOUTB2_retracted/perTCThit/ratioEkinMuons}
\includegraphics[width=0.4\textwidth]{figures/HL/compINOUTB2_retracted/perTCThit/ratioPhiNAll}
\includegraphics[width=0.4\textwidth]{figures/HL/compINOUTB2_retracted/perTCThit/ratioPhiNMuons}
\includegraphics[width=0.4\textwidth]{figures/HL/compINOUTB2_retracted/perTCThit/ratioPhiEnAll}
\includegraphics[width=0.4\textwidth]{figures/HL/compINOUTB2_retracted/perTCThit/ratioPhiEnMuons}
\end{center}
\vspace{-0.6cm}
 \caption{Azimuthal distribution of all particles and muons (top) and their energy (bottom).
  \label{fig:compInOutB2}}
\end{figure}


\begin{figure}
\centering
\includegraphics[width=0.43\textwidth]{figures/HL/compNomRetrCollSett/normalised/ratioEkinAll.pdf}
\includegraphics[width=0.43\textwidth]{figures/HL/compNomRetrCollSett/normalised/ratioPhiEnAll.pdf}
 \caption{Comparison of halo-induced background showers with nominal and \twosigmaret~collimator settings. Overall halo reduction of a factor 2 to 3 can be obtained considering also the IR7 to IR1 conversion factors\label{fig:compNomRetrSett2}.}
\end{figure}


% --------- comparison of Run I to HL

\begin{figure}
  \begin{center}
    \includegraphics[width=0.495\textwidth]{figures/4TeV/bs_20MeV/XYNCharZoom_BG_4TeV_20MeV_bs.pdf}
    \includegraphics[width=0.495\textwidth]{figures/4TeV/bs_20MeV/XYNNeutronsE10_BG_4TeV_20MeV_bs.pdf}
    \includegraphics[width=0.495\textwidth]{figures/4TeV/haloB1_20MeV/XYNNeutronsE10_BH_4TeV_B1_20MeV.pdf}
    \includegraphics[width=0.495\textwidth]{figures/4TeV/haloB2_20MeV/XYNNeutronsE10_BH_4TeV_B2_20MeV.pdf}
    \includegraphics[width=0.495\textwidth]{figures/4TeV/haloB1_20MeV/XYNPhotonsZoom_BH_4TeV_B1_20MeV.pdf}
    \includegraphics[width=0.495\textwidth]{figures/4TeV/haloB2_20MeV/XYNPhotonsZoom_BH_4TeV_B2_20MeV.pdf}
    \includegraphics[width=0.495\textwidth]{figures/4TeV/haloB1_20MeV/XYNCharZoom_BH_4TeV_B1_20MeV.pdf}
    \includegraphics[width=0.495\textwidth]{figures/4TeV/haloB2_20MeV/XYNCharZoom_BH_4TeV_B2_20MeV.pdf}
    
    %% \includegraphics[width=0.495\textwidth]{figures/HL/tct5otrd/XYNNeutronsE10_BH_HL_tct5otrdB2_20MeV.pdf}
    %% \includegraphics[width=0.495\textwidth]{figures/HL/tct5inrd/XYNNeutronsE10_BH_HL_tct5inrdB1_20MeV.pdf}
    %% \includegraphics[width=0.495\textwidth]{figures/HL/tct5inrd/XYNCharZoom_BH_HL_tct5inrdB1_20MeV.pdf}
    %% \includegraphics[width=0.495\textwidth]{figures/HL/tct5inrd/XYNElecPosi_BH_HL_tct5inrdB1_20MeV.pdf}
    %% \includegraphics[width=0.495\textwidth]{figures/HL/tct5inrd/XYNProtonsE10_BH_HL_tct5inrdB1_20MeV.pdf}
    %% \includegraphics[width=0.495\textwidth]{figures/HL/tct5inrd/XYNPhotonsE10_BH_HL_tct5inrdB1_20MeV.pdf}
\end{center}
\vspace{-0.6cm}
 \caption{FLUKA particle distributions in the (xy)-plane at the interface plane normalised to 1/cm$^{2}$/TCT hit for various scenarios (see plot title), for charged particles (top left), and others (see description top right corner).
  \label{fig:XYNPho}}
\end{figure}

\begin{figure}
  \begin{center}
%    \includegraphics[width=0.495\textwidth]{figures/BH_run2/b1/XYNCharZoom_BH_6500GeV_haloB1_20MeV.pdf}
%    \includegraphics[width=0.495\textwidth]{figures/BH_run2/b2/XYNCharZoom_BH_6500GeV_haloB2_20MeV.pdf}  
    %\includegraphics[width=0.49\textwidth]{figures/4TeV/bs_20MeV/XYNCharZoom_BG_4TeV_20MeV_bs.pdf}
    %\includegraphics[width=0.49\textwidth]{figures/4TeV/bs_20MeV/XYNNeutronsE10_BG_4TeV_20MeV_bs.pdf}
    %\includegraphics[width=0.495\textwidth]{figures/4TeV/haloB1_20MeV/XYNCharZoom_BH_4TeV_B1_20MeV.pdf}
    \includegraphics[width=0.495\textwidth]{figures/HL/tct5inrd/XYNCharZoom_BH_HL_tct5inrdB1_20MeV.pdf}
    \includegraphics[width=0.495\textwidth]{figures/HL/tct5otrd/XYNCharZoom_BH_HL_tct5otrdB1_20MeV.pdf} 
    \includegraphics[width=0.495\textwidth]{figures/HL/tct5otrd/XYNNeutronsE10_BH_HL_tct5otrdB1_20MeV.pdf}
    \includegraphics[width=0.495\textwidth]{figures/HL/tct5inrd/XYNNeutronsE10_BH_HL_tct5inrdB1_20MeV.pdf}
    \includegraphics[width=0.495\textwidth]{figures/HL/tct5otrd/XYNPhotonsE10_BH_HL_tct5otrdB1_20MeV.pdf}
    \includegraphics[width=0.495\textwidth]{figures/HL/tct5inrd/XYNPhotonsE10_BH_HL_tct5inrdB1_20MeV.pdf}
    \includegraphics[width=0.495\textwidth]{figures/HL/tct5otrd/XYNPhotonsE10_BH_HL_tct5otrdB2_20MeV.pdf}
    \includegraphics[width=0.495\textwidth]{figures/HL/tct5inrd/XYNPhotonsE10_BH_HL_tct5inrdB2_20MeV.pdf}

%    \includegraphics[width=0.495\textwidth]{figures/HL/tct5inrd/XYNNeutronsE10_BH_HL_tct5inrdB2_20MeV.pdf}
\end{center}
\vspace{-0.6cm}
 \caption{FLUKA particle distributions in the (xy)-plane at the interface plane normalised to 1/cm$^{2}$/TCT hit for various scenarios (see plot title), for charged particles (top left and bottom), and others (see description top right corner).
  \label{fig:XYN}}
\end{figure}

