\newpage
\section{Evolution of Background in LHC and Comparison to HL--LHC\label{evolut}}

We focus on the evolution of beam-gas and halo showers in IR1 for the Run~I, II and HL scenario highlighting muons as they are usually the most important particles for measuring background from the machine side. Other selected particle distributions can be found in the App.~\ref{evolutApp}.

\subsection{Comparison of background sources in 2012 and 2015}


We analyse properties of muons created in beam-gas and halo interactions. A shape comparison is made in Fig.~\ref{fig:compAllBKG_muons} for different background sources simulated for the Run~I at 4~TeV showing the energy spectrum of muons and the azimuthal distribution of their energy. One notices in for both plots, that sources from the TCT impacts as halo and offmomentum induced are very similar distributions at the interface plane. The beam-gas shape of the left figure has a few distinct features, it starts ealier at low energy and has one a dint at 10 to 100~GeV, but beam-gas overtakes the tctimpacts for the energies just as close as the beam energy. The other figure exhibits another feature: Muons from beam-gas give highest contributions at $0$ and $\pm$, but lowest in between, which means their energy is mostly horizontally distributed.\\

\begin{figure}%[htp]
\begin{center}
  \includegraphics[width=0.42\textwidth]{figures/4TeV/compAllBKG/EkinMuons.pdf}
  \includegraphics[width=0.42\textwidth]{figures/4TeV/compAllBKG/PhiEnMuons.pdf}
\end{center}
\vspace{-0.6cm}
 \caption{Comparison of all background sources at 4 TeV normalised per interaction showing energy spectrum and energy in $\phi$ for muons.
  \label{fig:compAllBKG_muons}}
\end{figure}
To visualise the spatial distribution of muons, we show them for beam-gas (left) and halo on the TCTs (right) in Fig.~\ref{fig:XYNMuons} for the same scenario at 4~TeV. One can recognise a geometrical effect of the beginning of the ATLAS cavern, a rectangular shape with a higher particle flux inside. At $s \approx 28~m$, a concrete shielding starts blocking particles leaving air only up to $\pm 100~$cm in $x$ and $y$. The last quadrupole of the triplet (MQXA) before the IP extends up to around $r \approx 46~$ (including tank and vacuum vessel) which is also visible in that figure. While muons in beam-gas collisions are created inside the vacuum tube and most of them stay inside, muons from halo showers are rather shielded by the vessel and higher rates can be found outside.


\begin{figure} %[htp]
  \centering
  \includegraphics[width=0.495\textwidth]{figures/XYNMuons_BG_4TeV_20MeV_bs.pdf}
  %\includegraphics[width=0.495\textwidth]{figures/XYNMuons_BG_6500GeV_flat_20GeV_bs.pdf}  
  \includegraphics[width=0.495\textwidth]{figures/XYNMuons_BH_4TeV_B1_20MeV.pdf}
  %\includegraphics[width=0.495\textwidth]{figures/XYNMuons_BH_6500GeV_haloB1_20MeV.pdf}
  \caption{Spatial distribution of muons of all energies in an beam-gas (left) and beam-halo (right) scenario for a 4~TeV beam in Run I 2012. At 6.5~TeV the distributions look similar, see Fig.~\ref{fig:XYNMuons2}.
    \label{fig:XYNMuons}}
\end{figure}


We normalise the results to rates towards ATLAS, using Eq.~\ref{eqNormHalo}, the run parameteres as listed in Tab.~\ref{paramsRun12} and IR7 leakages from Tab.~\ref{leakageFactorsIR7} and present the rates for 4~TeV in Fig.~\ref{compAllBKG_run12}. In all cases, beam-gas contributions are significantly higher than from those from halos. At 4~TeV, halo contributions are below a percent-level for particles and around 2~\% for energy produced from what is produced in beam-gas. This increases by about an order of magnitude for Run~II, where halo-muons form 20~\% and 25~\% of beam-gas muons. However, in absolute terms beam-gas rates are lower than in 2012, as mentioned due to the improved vacuum stability. One can also read off that figure that at 4~TeV, B2 produced double the amount of B1, while in 2015 both contributions are almost equal.

What is new in 6.5~TeV is that the muon rates produced in the arc follow now a profile with spikes (at transition regions of the bending magnets) while rates at 3.5 and 4~TeV in the arc are dominated by the assumption of contstant pressure (for 3.5~TeV pressures and analysis see Ref.~\cite{nimPaperRod}). Another interesting thing to note are the rates within the triplet (up to $s =~50~$m). The one at 6.5~TeV is above the one at 4~TeV, although anywhere else it is on the lowest level.

The rates are shown in Fig.~\ref{fig:compBGreweighted1} to compare directly the 4 and 6.5~TeV case of reweighted beam-gas. One clearly sees the shape has changed, the ratio plot on the bottom of the left figure around shows that from 20 to 150~GeV there were fewer muons, up to a factor 5 less at 6.5~TeV then at 4~TeV. In the right figure one can see that the energy difference carried by muons was reduced thereby almost factor 4. The shape is clearly different: while most of muons are horizontally distributed at 4~TeV, they show at 6.5~TeV now peaks in the vertical plane as well. This regular structure must have it origins in the magnetic fields. 
% --------------------------------------------------------------------------------------------

\begin{figure}
\begin{center}
  \includegraphics[width=0.49\textwidth]{figures/4TeV/reweighted/cv78_EkinMuons.pdf}
  \includegraphics[width=0.49\textwidth]{figures/4TeV/reweighted/cv78_PhiEnMuons.pdf}
  \includegraphics[width=0.49\textwidth]{figures/6500GeV/reweighted/cv78_EkinMuons.pdf}
  \includegraphics[width=0.49\textwidth]{figures/6500GeV/reweighted/cv78_PhiEnMuons.pdf}
\end{center}
\vspace{-0.6cm}
 \caption{Rate comparison of halo and beam-gas at 4~TeV (top) and 6.5~TeV (bottom). The numbers in the legend indicate the fraction to the beam-gas data of the respective scenario. They are formed from the integral of the distributions.
  \label{compAllBKG_run12}}
\end{figure}

\begin{figure}%[!htb]
\centering

\includegraphics[width=0.45\textwidth]{figures/compBGreweighted/ratioEkinMuons.pdf}
\includegraphics[width=0.45\textwidth]{figures/compBGreweighted/ratioPhiEnMuons.pdf}
\caption{Reweighted beam-gas distributions in the 2012 Run I and 2015 Run II scenario for muons showing the energy spectrum (left) and the azimuthal distribution of the energy (right).
  \label{fig:compBGreweighted1}}
\end{figure}


\begin{figure}
\begin{center}
%  \includegraphics[width=0.8\textwidth]{figures/cv87_allenergies_OrigZAll.pdf}
  \includegraphics[width=0.8\textwidth]{figures/cv87_allenergies_OrigZMuon.pdf}
\end{center}
\vspace{-0.6cm}
 \caption{Evolution of muon rates. 
  \label{fig:OrigZMuon}} 
\end{figure}




\begin{figure}
\begin{center}

  \includegraphics[width=0.49\textwidth]{figures/compBHB1_4TeV_vs_6p5TeV/normalised/ratioEkinMuons.pdf}
  \includegraphics[width=0.49\textwidth]{figures/compBHB1_4TeV_vs_6p5TeV/normalised/ratioPhiEnMuons.pdf}

\end{center}
\vspace{-0.6cm}
 \caption{Comparison of halo induced background at 4 and 6.5~TeV in the azimuthal distributions of all particles at the interface plane (top) and high-energy muons and protons (bottom) and their energy.
  \label{compBHB1run1run2}}
\end{figure}


\begin{figure}
\centering
  \includegraphics[width=0.49\textwidth]{figures/compBHB2_4TeV_vs_6p5TeV/normalised/ratioEkinMuons.pdf}
  \includegraphics[width=0.49\textwidth]{figures/compBHB2_4TeV_vs_6p5TeV/normalised/ratioPhiEnMuons.pdf}
 \caption{Comparison of halo induced background at 4 and 6.5~TeV in the azimuthal distributions of all particles at the interface plane (top) and high-energy muons and protons (bottom) and their energy.
  \label{compBHB1run1run2}}
\end{figure}




\subsection{Run II vs HL--LHC}

\begin{figure}
\begin{center}
  \includegraphics[width=0.42\textwidth]{figures/HLRunII/cv78_EkinMuons.pdf}
  \includegraphics[width=0.42\textwidth]{figures/HLRunII/cv78_PhiEnMuons.pdf}
\end{center}
\vspace{-0.6cm}
 \caption{Comparison of beam-gas (BG) in Run II and beam-halo in HL using the baseline layout (TCT5s in, \twosigmaret~settings) and round beam optics.
  \label{fig:hlrun2}}
\end{figure}
