\newpage
\section{Evolution of Background in LHC and Comparison to HL--LHC\label{evolut}}

While several sources of background had been studied in the past, we focus here on the evolution of the most dominant ones, local beam-gas and beam-halo, in IR1. 


\subsection{IR7 TCP to TCT conversion factors in beam-halo simulations}
The approach to normalise the halo induced distributions is to scale the distribution per TCT hit up by the initial particle flux (thus the total number of protons initially in the machine, the product of the number of bunches, $N_{\mathrm{bunch}}$, times the intensity per bunch, $I_{\mathrm{bunch}}$ per time unit $\tau$ for which we assume a stable beam operation scenario of 100 hours beam life time) multiplied by a sixtrack conversion factor, which takes into account the efficiency of the collimator settings and is calculated by the ratio of absorbed protons on the TCTs at IR1 or 5 and those absorbed by the TCPs in IR7 per simulation case h and v. For TCPs,  Using the absorbed protons at IR7 primaries makes it easier to compare BLM losses in IR7, also losses on secondaries or tertiaries do not change dramatically, i.e. an order of magnitude, when having them considered in the denominator as well to obtain IR7 leakage factors.

\begin{equation} \label{eq3}
\frac{N_{\mathrm{bunch}} \cdot I_{\mathrm{bunch}}}{\tau_{\mathrm{stable \, beam}}} \times \Bigg[ \frac{1}{2} \cdot \Big( \frac{N_{\mathrm{TCTV}} + N_{\mathrm{TCTH}}}{N_{\mathrm{TCP.IR7}}}\Big)_{\mathrm{h}} + \Big( \frac{N_{\mathrm{TCTV}} + N_{\mathrm{TCTH}}}{N_{\mathrm{TCP.IR7}}} \Big)_{\mathrm{v}}  \Bigg]
\end{equation}



\subsection{Comparison of Run I background sources}
We investigate how the different sources of background compare to each other. We show them per interaction for the Run I case in Fig.~\ref{fig:compAllBKG_perInt1} and normalised to expected rates for stable beam operation in Fig~\ref{compAllBKG4TeV_rates}.

\begin{figure}
\begin{center}
  \includegraphics[width=0.42\textwidth]{figures/4TeV/compAllBKG/EkinAll.pdf}
  \includegraphics[width=0.42\textwidth]{figures/4TeV/compAllBKG/PhiEnAll.pdf}
  \includegraphics[width=0.42\textwidth]{figures/4TeV/compAllBKG/EkinMuons.pdf}
  \includegraphics[width=0.42\textwidth]{figures/4TeV/compAllBKG/PhiEnMuons.pdf}
  \includegraphics[width=0.42\textwidth]{figures/4TeV/compAllBKG/EkinProtons.pdf}
  \includegraphics[width=0.42\textwidth]{figures/4TeV/compAllBKG/PhiEnProtons.pdf}
  \includegraphics[width=0.42\textwidth]{figures/4TeV/compAllBKG/EkinPhotons.pdf}
  \includegraphics[width=0.42\textwidth]{figures/4TeV/compAllBKG/PhiEnPhotons.pdf}
\end{center}
\vspace{-0.6cm}
 \caption{Comparison of all background sources at 4 TeV normalised per interaction showing energy spectrum and energy in $\phi$.
  \label{fig:compAllBKG_perInt1}}
\end{figure}

%% \begin{figure}
%% \begin{center}
%%   \includegraphics[width=0.42\textwidth]{figures/4TeV/compAllBKG/RadNAll.pdf}
%%   \includegraphics[width=0.42\textwidth]{figures/4TeV/compAllBKG/RadEnAll.pdf}
%%   \includegraphics[width=0.42\textwidth]{figures/4TeV/compAllBKG/RadNMuons.pdf}
%%   \includegraphics[width=0.42\textwidth]{figures/4TeV/compAllBKG/RadEnMuons.pdf}
%%   \includegraphics[width=0.42\textwidth]{figures/4TeV/compAllBKG/RadNProtons.pdf}
%%   \includegraphics[width=0.42\textwidth]{figures/4TeV/compAllBKG/RadEnProtons.pdf}
%%   \includegraphics[width=0.42\textwidth]{figures/4TeV/compAllBKG/RadNPhotons.pdf}
%%   \includegraphics[width=0.42\textwidth]{figures/4TeV/compAllBKG/RadEnPhotons.pdf}
%% \end{center}
%% \vspace{-0.6cm}
%%  \caption{Comparison of all background sources at 4 TeV normalised per interaction showing radial distributions and energy in $r$.
%%   \label{fig:compAllBKG_perInt2}}
%% \end{figure}


\begin{figure}
\begin{center}
  \includegraphics[width=0.42\textwidth]{figures/4TeV/reweighted/cv78_EkinAll.pdf}
  \includegraphics[width=0.42\textwidth]{figures/4TeV/reweighted/cv78_PhiEnAll.pdf}
  \includegraphics[width=0.42\textwidth]{figures/4TeV/reweighted/cv78_EkinMuons.pdf}
  \includegraphics[width=0.42\textwidth]{figures/4TeV/reweighted/cv78_PhiEnMuons.pdf}
  \includegraphics[width=0.42\textwidth]{figures/4TeV/reweighted/cv78_EkinProtons.pdf}
  \includegraphics[width=0.42\textwidth]{figures/4TeV/reweighted/cv78_PhiEnProtons.pdf}
  \includegraphics[width=0.42\textwidth]{figures/4TeV/reweighted/cv78_EkinPhotons.pdf}
  \includegraphics[width=0.42\textwidth]{figures/4TeV/reweighted/cv78_PhiEnPhotons.pdf}
\end{center}
\vspace{-0.6cm}
 \caption{comparison of all background sources at 4 TeV normalised to a rate.
  \label{compAllBKG4TeV_rates}}
\end{figure}

%% \begin{figure}
%% \begin{center}
%%   \includegraphics[width=0.42\textwidth]{figures/4TeV/reweighted/cv78_RadNAll.pdf}
%%   \includegraphics[width=0.42\textwidth]{figures/4TeV/reweighted/cv78_RadEnAll.pdf}
%%   \includegraphics[width=0.42\textwidth]{figures/4TeV/reweighted/cv78_RadNMuons.pdf}
%%   \includegraphics[width=0.42\textwidth]{figures/4TeV/reweighted/cv78_RadEnMuons.pdf}
%%   \includegraphics[width=0.42\textwidth]{figures/4TeV/reweighted/cv78_RadNProtons.pdf}
%%   \includegraphics[width=0.42\textwidth]{figures/4TeV/reweighted/cv78_RadEnProtons.pdf}
%%   \includegraphics[width=0.42\textwidth]{figures/4TeV/reweighted/cv78_RadNPhotons.pdf}
%%   \includegraphics[width=0.42\textwidth]{figures/4TeV/reweighted/cv78_RadEnPhotons.pdf}
%% \end{center}
%% \vspace{-0.6cm}
%%  \caption{comparison of all background sources at 4 TeV normalised to a rate.
%%   \label{compAllBKG4TeV_rates}}
%% \end{figure}

% --------------------------------------------------------------------------------------------
\subsection{Comparison of Run II background sources}
\begin{figure}
\begin{center}
  \includegraphics[width=0.42\textwidth]{figures/6500GeV/reweighted/cv78_EkinAll.pdf}
  \includegraphics[width=0.42\textwidth]{figures/6500GeV/reweighted/cv78_PhiEnAll.pdf}
  \includegraphics[width=0.42\textwidth]{figures/6500GeV/reweighted/cv78_EkinMuons.pdf}
  \includegraphics[width=0.42\textwidth]{figures/6500GeV/reweighted/cv78_PhiEnMuons.pdf}
  \includegraphics[width=0.42\textwidth]{figures/6500GeV/reweighted/cv78_EkinProtons.pdf}
  \includegraphics[width=0.42\textwidth]{figures/6500GeV/reweighted/cv78_PhiEnProtons.pdf}
 \includegraphics[width=0.42\textwidth]{figures/6500GeV/reweighted/cv78_EkinPhotons.pdf}
 \includegraphics[width=0.42\textwidth]{figures/6500GeV/reweighted/cv78_PhiEnPhotons.pdf}
\end{center}
\vspace{-0.6cm}
 \caption{comparison of all background sources at 6.5~TeV.
  \label{compAllBKG_6.5}}
\end{figure}

\begin{figure}
\begin{center}
  \includegraphics[width=0.8\textwidth]{figures/cv87_allenergies_OrigZAll.pdf}
  \includegraphics[width=0.8\textwidth]{figures/cv87_allenergies_OrigZMuon.pdf}
\end{center}
\vspace{-0.6cm}
 \caption{
  \label{fig:OrigZMuonAllEn}} 
\end{figure}

% --------------------------------------------------------------------------------------------
\subsection{Beam-gas rate comparisons in Run I and Run II}
\begin{figure}[!htb]
\centering
\includegraphics[width=0.45\textwidth]{figures/compBGreweighted/ratioEkinAll.pdf}
\includegraphics[width=0.45\textwidth]{figures/compBGreweighted/ratioEkinMuons.pdf}
\includegraphics[width=0.45\textwidth]{figures/compBGreweighted/ratioPhiEnAll.pdf}
\includegraphics[width=0.45\textwidth]{figures/compBGreweighted/ratioPhiEnMuons.pdf}
\caption{Reweighted beam-gas distributions in the 2012 Run I and 2015 Run II scenario for all particles and muons showing the energy spectrum (top) and the azimuthal distribution (bottom).
  \label{fig:compBGreweighted1}}
\end{figure}

\begin{figure}%[!htb]
\centering
\includegraphics[width=0.45\textwidth]{figures/compBGreweighted/ratioRadNAll.pdf}
\includegraphics[width=0.45\textwidth]{figures/compBGreweighted/ratioRadNMuons.pdf}
\includegraphics[width=0.45\textwidth]{figures/compBGreweighted/ratioRadEnAll.pdf}
\includegraphics[width=0.45\textwidth]{figures/compBGreweighted/ratioRadEnMuons.pdf}
\caption{Reweighted beam-gas distributions in the 2012 Run I and 2015 Run II scenario for all particles and muons showing radial positions and energy in $r$.
  \label{fig:compBGreweighted2}}
\end{figure}

%% \subsection{Comparison of halo hit distributions}


%% \begin{figure}
%% \begin{center}
%% \includegraphics[width=0.495\textwidth]{figures/inelposition_sum_impacts_real_HL_TCT5IN_nomColl_haloB1.pdf}
%% %%\includegraphics[width=0.495\textwidth]{figures/inelposition_sum_impacts_real_tct5inb1_crabs.pdf}
%% \end{center}
%%  \caption{Depth distribution of TCT hits when using the nominal collimator settings (from the LHC Design report).
%%   \label{inelHLtct5inNomCrab}}
%% \end{figure}



\subsection{Comparison of different scenarios \label{compRun1Run2}}




\subsubsection{2012 4 TeV vs. 2015 6.5~TeV}
\begin{figure}
\begin{center}
  \includegraphics[width=0.4\textwidth]{figures/compBHB1_4TeV_vs_6p5TeV/normalised/ratioEkinAll.pdf}
%  \includegraphics[width=0.4\textwidth]{figures/compBHB1_4TeV_vs_6p5TeV/normalised/ratioEkinMuons.pdf}
  \includegraphics[width=0.4\textwidth]{figures/compBHB1_4TeV_vs_6p5TeV/normalised/ratioPhiEnAll.pdf}
  \includegraphics[width=0.4\textwidth]{figures/compBHB1_4TeV_vs_6p5TeV/normalised/ratioPhiEnMuons.pdf}
%  \includegraphics[width=0.4\textwidth]{figures/compBHB1_4TeV_vs_6p5TeV/normalised/ratioRadEnAll.pdf}
  \includegraphics[width=0.4\textwidth]{figures/compBHB1_4TeV_vs_6p5TeV/normalised/ratioRadEnMuons.pdf}
\end{center}
\vspace{-0.6cm}
 \caption{Comparison of halo induced background at 4 and 6.5~TeV in the azimuthal distributions of all particles at the interface plane (top) and high-energy muons and protons (bottom) and their energy.
  \label{compBHB1run1run2}}
\end{figure}

\begin{figure}%[!htb]
\begin{center}
  \includegraphics[width=0.49\textwidth]{figures/compBHB2_4TeV_vs_6p5TeV/normalised/ratioEkinAll.pdf}
  \includegraphics[width=0.49\textwidth]{figures/compBHB2_4TeV_vs_6p5TeV/normalised/ratioPhiEnAll.pdf}
  \includegraphics[width=0.49\textwidth]{figures/compBHB2_4TeV_vs_6p5TeV/normalised/ratioPhiEnMuons.pdf}
  \includegraphics[width=0.49\textwidth]{figures/compBHB2_4TeV_vs_6p5TeV/normalised/ratioRadEnMuons.pdf}
\end{center}
\vspace{-0.6cm}
 \caption{Comparison of halo induced background at 4 and 6.5~TeV in the azimuthal distributions of all particles at the interface plane (top) and high-energy muons and protons (bottom) and their energy.
  \label{compBHB2run1run2}}
\end{figure}



\begin{figure}%[!htb]
\begin{center}
  \includegraphics[width=0.49\textwidth]{figures/compBG_4TeV_vs_6.5TeV/ratioEkinAll.pdf}
  \includegraphics[width=0.49\textwidth]{figures/compBG_4TeV_vs_6.5TeV/ratioPhiEnAll.pdf}
  \includegraphics[width=0.49\textwidth]{figures/compBG_4TeV_vs_6.5TeV/ratioPhiNMuE100.pdf}
  \includegraphics[width=0.49\textwidth]{figures/compBG_4TeV_vs_6.5TeV/ratioPhiEnMuE100.pdf}
\end{center}
\vspace{-0.6cm}
 \caption{Comparison BG interactions in the azimuthal distributions of all particles at the interface plane (top) and high-energy muons (bottom) and their energy at 4 and 6.5 TeV.
  \label{compBGrun1run2}}
\end{figure}


\subsection{Run II vs HL--LHC}

