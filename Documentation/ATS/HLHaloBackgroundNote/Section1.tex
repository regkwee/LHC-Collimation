\section{Introduction}

%\section{Particle losses in the LHC machine and sources of background to the experiments}

% machine losses
The LHC has finished its first run and is at the end of its second run of operation. Proton-proton collisions collected by the two large experiments ATLAS and CMS have led to the long thought discovery of the higgs particle. This motivates an upgrade of the LHC with even higher luminosities, HL-LHC, where the design study that has turned successfully into a project planning to take up major constructions in 2024.

In this report particle losses and background sources to experiments are discussed in all of these scenarios ranging from 4 to 6.5 TeV to several in HL-LHC using different collimator settings and beam optics. Simulations have been performed of the most important sources of beam induced background using new, improved techniques. Any interaction of beam particles with anything but beam particles of the other beam showering towards the experiment can contribute to background.

Depending on how often they appear they are judged either critical to managable for operation. Often it is hard to precisely predict the rate as the rate depends on machine conditions and settings, like the vacuum quality and settings of the collimators which defines the leakage to the experimental areas. Certainly, also very basic beam properties are crucial for background, e.g.~the beam life time directly correlates with the rates observable at the experiments. There were also luminosity-dependent phenomena observed. In such a complex machine like the LHC with various filling schemes and different beam optics to meet various physics purposes at increasing beam energies, there is a good chance to encounter background phenomena which have not been known from other machines, like in ATLAS who observed ``after-glow" or ``ghost-charge"~\cite{ATLAS_JINST_13}. That is why the experiments, even if the background is judged to be not critical for operation, are motivated to study their origin and characteristica. As experiments would like to analyse rare interaction processes, also background that does not appear at a high rate is of high interest. This enables them to study e.g.~how many high energy particles reach which part of the detector. This is vital for them to substract the background from signals that may mean a new particle. 

One can differentiate between several background sources depending on how they are created. Once typical characteristic distributions are known, it is a matter of normalisation which encounters the specific running conditions to determine the rate. In this note, we focus on the study of shape-properties of background and investigate three sources that can create background at the experiments. We investigate them in a typical scenarios for physics runs during Run I and II.

A well-known source is if beam protons hit residual gas molecules inelastically producing a stray of secondaries and unwanted signals in the detector. The rate of such beam-gas events is tightly connected to the vacuum quality and thus has to be permanently monitored by both, the experiments and machine operators.

Another source of steady losses in the machine are halo protons, beam particles with increased betatron amplitude (due to dynamical effects within the bunch) moving outwards of the bunch. Eventually they get lost before they reach one of the IP's (interaction point). In order to avoid losses in cold areas (where they can cause magnets to quench when depositing too much energy in the supraconductiong coils) an entire cleaning insertion in IR7 is dedicated to these unavoidable steady losses. Most of what is cleaned in IR7 is even further reduced by tertiary collimators close to the experiments, each IR where the bunches cross, is equipped with such collimators. Anything that leaks from this system can produce a background shower for the experiment when interacting with the collimator material. 

A very similar station is installed in IR3 to clean off-momentum particles. They had never been simulated as background before, we report here for the first time on their characteristica and how they compare to the almost traditional background sources of beam-gas and beam-halo. 

Our simulations are mostly for IR1, in particular shower distributions are shown for the machine-detector interface in IR1 only as the layout of the beamline is identical to IR5. Of course there are other differences, like the path of the beam is different from injection to IP5, so is the crossing plane of the beam. Therefore, we highlight the differences of collimator losses in IR5. We also look in IR7 for HL-LHC cases.
