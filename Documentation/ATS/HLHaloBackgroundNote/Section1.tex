\section{Introduction}


The LHC has finished its first run and is half way through its second run of operation. Proton-proton collisions collected by the two high-luminosity experiments ATLAS and CMS have led to the long thought discovery of the Higgs particle~\cite{Aad20121,Chatrchyan201230}. To extend the discovery potential of the LHC, an upgrade is planned with even higher luminosities, HL-LHC~\cite{hl-lhc-prel-design}, and it is foreseen to take up major constructions in 2024.

In this report particle losses and background sources to experiments are discussed in all of these scenarios ranging from 4 to 6.5 TeV in LHC to 7~TeV in HL-LHC using different collimator settings and beam optics. Simulations have been performed for two of the most important sources of beam-induced background using new, improved techniques. Any interaction of beam particles with anything upstream the interaction point (IP), causing shower towards the experiment, can contribute to background.

Depending on how often signals from such interactions appear they are judged either critical or managable for machine and detector operation. Both, machine and experiments, have to assess, if an increased risk of damage to the machine or to one of its detectors exist for a given machine configuration. If no such risk is present, and the backround rate is managable, experimentalists are able to extract useful signals from their detector. Often it is hard to precisely predict the machine-induced background rate for experiments as the it depends on machine conditions and settings, like the vacuum quality and settings of the collimators, which define the leakage to the experimental areas. Certainly, also very basic beam properties are crucial for background, e.g.~the beam lifetime directly correlates with the rates observable at the experiments and the lifetime is not easy to control or predict. In such a complex machine like the LHC with various filling schemes and different beam optics to meet various physics purposes at increasing beam energies, there is a good chance to encounter background phenomena which have not been well known from other machines, like in ATLAS where ``after-glow" or ``ghost-charge"~\cite{ATLAS_JINST_13} were observed. That is why the experiments, even if the background is judged not to be critical for operation in the current run, are motivated to study their origin and characteristics. Certain sources may become an issue in the future when machine parameters change. As experiments would like to analyse rare interaction processes, also background that does not appear at a high rate is of high interest. This enables them to study e.g.~how many high energy particles reach which part of the detector. This is vital for them to substract the background from signals that may mean a new particle. 

One can differentiate between several background sources depending on how they are created. Once typical characteristic distributions are known, it is a matter of normalisation which accounts for the specific running conditions to determine the rate. In this report, we present both, the study of shapes of background distributions at a virtual machine-detector interface plane and rate estimates of two of the most important sources close by that can create background at the experiments. We discuss three sources here.

A well-known source is if beam protons hit residual gas molecules inelastically upstream and in the vicinity of the experiment producing a shower of secondaries and unwanted signals in the detector. The rate of such beam-gas events is tightly connected to the vacuum quality and thus has to be permanently monitored by both, the experiments and machine operators.

Another source of steady losses in the machine are protons diffusing out to high transverse amplitudes (they form a halo of the beam). Eventually they get lost before they reach one of the interaction points (IPs). In order to avoid losses in cold areas (where they can cause magnets to quench when depositing too much energy in the superconducting coils) an entire cleaning insertion in IR7 is dedicated to these unavoidable steady losses~\cite{LHCDesignRep,assmann05chamonix}. Halo particles escaping IR7 are further reduced by tertiary collimators (TCTs) close to the experiments. Each IR where the bunches cross, is equipped with such collimators. Anything that leaks from this system can produce a background shower for the experiment when interacting with the collimator material. 

A very similar system is installed in IR3 to clean off-momentum halo particles. These particles, when they deviate from the nominal beam energy, will be bent and focused differently and run on a different orbit given by the dispersion function. The leakage out of IR3 impacts as well on the TCTs in the experimental IRs and contribute to background at the experiments. They had never been simulated as background before, we report here for the first time on their characteristics and how they compare to the other two background sources of local beam-gas and betatron halo. 

Secondary particle distributions are evaluated at an artificial interface plane between the machine and detector, which is defined at 22.6~m from IP1. This is the main output of the studies presented in this report, and can be used as input for further background studies by the experiments within the detectors. We either highlight general particle distributions or focus on muons as experiments are usually interested in these particles. The IR1 layout of the beamline is identical to that in IR5, but there are other differences, like the path length from cleaning insertions to IP1, the crossing plane is chosen horizontally in IR5 and vertically in IR1 (but the crossing angle in IR1 and IR5 are the same). Therefore, we study also collimator losses in IR5.

We also present details of several scenarios simulated with the final HL-LHC collimation layout. We investigate how much the shapes of particle distributions at the interface plane differ from the present LHC and what can be expected in terms of rates. As the stored beam energy increases from 360~MJ to 675~MJ, all scenarios of beam losses are more critical and one has to review dominant sources. This report highlights the evolution of background sources including future trends for the upgrade of the LHC.

%In Sect.~\ref{simSetup} we described the simulation techniques used in Sect.~\ref{run1run2} for the results for Run~1 and Run~2 as well as for scenarios in HL--LHC in Sect.~\ref{hllhcResults}. We discuss the evolution of background sources in Sect.~\ref{evolut} and give conclusions in Sect.~\ref{last}. 
