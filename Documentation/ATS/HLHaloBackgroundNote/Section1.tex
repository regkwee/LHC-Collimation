\section{Introduction}
The High Luminosity (HL) LHC is a major upgrade project to produce in total 3000~\ifb~integrated luminosity for each, ATLAS and CMS, starting installation around 2023~\cite{HLLHCWebRef}. In particular, the upgrade plans affect the experimental insertion regions (IR) IR1 and IR5, housing ATLAS and CMS respectively, to reach higher luminosities. The optics have been re-designed (ATS -- achromatic telescopic squeeze -- optics~\cite{ATSref}) and require e.g. larger-aperture magnets for squeezing the optics to 15~cm in the horizontal and vertical plane at the high-luminosity interaction points (IPs) 1 and 5 in order to achieve the luminosity goal. New possible aperture bottlenecks arise due to HL-LHC layout changes, and the quadrupoles Q5 and Q4 may no longer be sufficiently protected, see also Fig.~\ref{layoutRod}. To address this, the collimation system~\cite{LHCDesignRep,collRef} will be upgraded in the experimental IRs. While these upgrades are in detailed in~\cite{layoutProcRod}, the focus of this note is on upgrade plans for the incoming beam. The new layout forsees to place in cell 5 a vertical and horizontal tertiary collimator (TCT5s for TCTH.5 and TCTV.5). As they are further away from the exisiting horizontal and vertical collimators (TCT4s that are TCTH.4 and TCTV.4), as illustrated in Fig.~\ref{layoutRod}, they could help reducing beam-induced halo background. This note presents a first estimate of the halo induced background reduction based on simulations.

\begin{figure*}[!tbh]
    \centering
    \includegraphics[height=3.3cm,width=\textwidth]{figures/TUPTY067f1}
    \vspace{-0.6cm}
    \caption{Planned HL-LHC layout~\cite{layoutProcRod} for the incoming beam in the experimental insertion region of ATLAS (IP1) with tertiary collimator pairs (TCT4s and TCT5s) highlighted above the beamline. TCT4s are at around 131~m and TCT5s at 213~m. The layout in IR5 is identical.}
    \label{layoutRod}
\end{figure*}
