\section{Introduction}

%\section{Particle losses in the LHC machine and sources of background to the experiments}

% machine losses
The LHC has finished its first run and is in the middle of its second run of operation. Proton-proton collisions collected by the two large experiments ATLAS and CMS have led to the long thought discovery of the higgs particle. This motivates an upgrade of the LHC with even higher luminosities, HL-LHC, the design study that has become a project for which the process for a Technical Design Report this year has been launched. \\

In this report particle losses and sources to experiments are discussed in all of these scenarios. Simulations have been performed of the most important sources of beam induced background using new, improved techniques. Any interaction of beam particles with anything but beam particles of the other beam showering towards the experiment can contribute to background. A well-known source is if beam protons hit residual gas molecules inelastically producing a stray of secondaries and unwanted signals in the detector. Depending on how often they appear they are judged either critical to managable for operation. Even if they are not critical, experiments want to know what they look like, what the characteristica are in order to substract them from signals that may mean a new particle. The rate of such beam-gas events is tightly connected to the vacuum quality and because of that ... \\

Another source of steady losses in the machine is equally of great interest: halo protons. These are beam particles which increase their betatron amplitude due to dynamical effects within the bunch, moving outwards of the bunch, and eventually they get lost before they reach one of the IP's (interaction point). An entire cleaning insertion is dedicated to these unavoidable steady losses in IR7 to protect the cold magnets and minimise downtime. Most of what is cleaned in IR7 is even further reduced by tertiary collimators close to the experiments. Anything that leaks from this system can produce a background shower for the experiment when interacting with the collimator material. 

A very similar station is installed in IR3 to clean particles with an off-momentum. They have not yet been simulated as background before, we report here for the first time on their characteristica and how they compare to the almost traditional background sources. \\

This in not a complete review on all sources only those that are of interest to the experiments. We have added a scenario in HL-LHC that discusses a different type of scenario, a worst case failure of crab cavities. There are more kinds of background that are not discussed here like elastically scattered protons or protons emerging from single diffractive dissociation that contribute to background shower at other locations than the place of creation. This is left for future study.

% steady background


%% Beam particles are lost throughout the entire machine, most of them are steady losses, foreseen to happen and unavoidable due to beam dynamics. Unless the amount of lost particles is small, the beam remains stable and under normal conditions these particles are cleaned by the LHC collimation system~\cite{collRef}. The two insertions in the LHC, one with large betatron amplitude (IR7) and one for off-momentum protons (IR3) have been in detailed described elsewhere, e.g.~\cite{roderikSimMeasPaper}. 


%% % background sources
%% Several sources in the machine contribute to background signals in the experiments (often also refered to as \textit{MIB} - machine induced background). Two of these sources are most relevant for the experiments and are studied in this paper: One are induced by beam protons interacting with residual gas molecules, refered to as \textit{beam-gas}, while for the other one beam halo protons leaking out of the cleaning system interact with tertiary collimators around $\pm$150~m away from the IP on each side of the incoming beam and produce particle showers streaming towards the experiment. We discuss two types background in this paper, beam-gas interactions and beam-halo showers (short halo). Other sources are e.g.~cross-talk are left for future studies where scattered protons in an elastic or inelastic interaction process at an IP is intercepted in another TCT where it creates a shower towards the respective experiment. 

%% \subsection{Beam-Gas}

%% Beam-gas interactions can be divided into different classes depending on whether they collisions are close to the experiment (\textit{local beam-gas}) or far out (\textit{global beam-gas}) and whether it was an inelastic or elastic interaction. Here, the focus is on local beam-gas simulations as local pressure peaks are found to correlate well with observed background in the experiment~\cite{nimPaperRod}.

%% \subsection{Beam-Halo}
%% Beam particles in the accelerator oscillate around an ideal orbit but diffuse out the beam core due to various beam dynamics effects (e.g.~particle-particle scattering within a bunch, interactions between colliding bunches, scattering with residual gas-molecules) forming the \textit{beam halo} and unavoidable losses. The task of the collimation system is to safely remove these halo particles in two dedicated insertion regions, IR7 for betatron and IR3 for momentum cleaning~\cite{LHCDesignRep,collRef}. Tertiary collimators (TCTs) in the experimental IR's complement the IR7/3 collimators and are installed to provide local protection of the inner triplet magnets. Halo protons leaking from IR7/3, should mostly be stopped by the TCTs. While interacting with the collimator material, they produce particle showers that contribute to the machine-induced halo background. 

%% Note, in the experiments the term halo is often used as muon-halo since these particles will not be stopped but rather pass the detector even at very large (transverse) radii. The high energetic halo induced muons at large radii are of particular interest to the experiments as they will produce fake jet signals in the calorimeter and must be indentified in order to substract them from any jet-based physics analysis. 

%% The topology of this background source and their frequencey is therefore extensively studied in the following chapters.

%% \subsection{(Off-momentum losses and others)}

%% The High Luminosity (HL) LHC is a major upgrade project to produce in total 3000~\ifb~integrated luminosity for each, ATLAS and CMS, starting installation around 2023~\cite{HLLHCWebRef}. In particular, the upgrade plans affect the experimental insertion regions (IR) IR1 and IR5, housing ATLAS and CMS respectively, to reach higher luminosities. The optics have been re-designed (ATS -- achromatic telescopic squeeze -- optics~\cite{ATSref}) and require e.g. larger-aperture magnets for squeezing the optics to 15~cm in the horizontal and vertical plane at the high-luminosity interaction points (IPs) 1 and 5 in order to achieve the luminosity goal. New possible aperture bottlenecks arise due to HL-LHC layout changes, and the quadrupoles Q5 and Q4 may no longer be sufficiently protected, see also Fig.~\ref{layoutRod}. To address this, the collimation system~\cite{LHCDesignRep,collRef} will be upgraded in the experimental IRs. While these upgrades are in detailed in~\cite{layoutProcRod}, the focus of this note is on upgrade plans for the incoming beam. The new layout forsees to place in cell 5 a vertical and horizontal tertiary collimator (TCT5s for TCTH.5 and TCTV.5). As they are further away from the exisiting horizontal and vertical collimators (TCT4s that are TCTH.4 and TCTV.4), as illustrated in Fig.~\ref{layoutRod}, they could help reducing beam-induced halo background. This note presents a first estimate of the halo induced background reduction based on simulations.

%% \begin{figure*}[!tbh]
%%     \centering
%%     \includegraphics[height=3.3cm,width=\textwidth]{figures/TUPTY067f1}
%%     \vspace{-0.6cm}
%%     \caption{Planned HL-LHC layout~\cite{layoutProcRod} for the incoming beam in the experimental insertion region of ATLAS (IP1) with tertiary collimator pairs (TCT4s and TCT5s) highlighted above the beamline. TCT4s are at around 131~m and TCT5s at 213~m. The layout in IR5 is identical.}
%%     \label{layoutRod}
%% \end{figure*}
